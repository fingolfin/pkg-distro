\Chapter{What is XGAP?}

In this chapter you find the answer to the above question beginning from a
short overview up to a description of the technical concept.

%%%%%%%%%%%%%%%%%%%%%%%%%%%%%%%%%%%%%%%%%%%%%%%%%%%%%%%%%%%%%%%%%%%%%%%%%%%%%
\Section{Basics}

The idea of \XGAP~is that \GAP~should be able to control graphics. A
graphical user interface is sometimes easier to use than a text and
command oriented one and there are mathematical applications for which it
can be quite useful to visualize objects with computer graphics.

On the other hand it is not sensible to change the whole concept and user
interface of {\GAP} because it is not advisable to put all the facilities of
{\GAP} into a menu system. So \XGAP~is a separate C program running under the
X Window System, which starts up a \GAP~job and allows normal command
execution within a window. Note that the online help of {\GAP} is
available,  however it will appear in a separate window.

In addition there is a library written in \GAP, which makes it
possible to open new windows, display graphics, control menus and do
other graphical user communication in \GAP~via the separate C part.

Built on those ``simple'' windows and graphic objects are other libraries
which display graphs and posets in a window and allow the user to move
vertices around, select them and invoke \GAP~functions on mathematical
objects which belong to the graphic objects.

One ``application'' of these libraries is a program to display subgroup
lattices interactively. So \XGAP~works as a front end for mathematical
operations on subgroup lattices. It is possible to ``switch'' between the
graphics and the \GAP~commands. This means that you can for example use the 
graphically selected vertices resp. subgroups to do your own calculations
in the command window. You can then display your results again as vertices
in the lattice.

Of course there are other applications possible and the libraries are
developed with code reusage in mind.

%%%%%%%%%%%%%%%%%%%%%%%%%%%%%%%%%%%%%%%%%%%%%%%%%%%%%%%%%%%%%%%%%%%%%%%%%%%%%
\Section{What you can do with XGAP}

{\XGAP} graphic sheets work graphic object oriented. This means that the
basic graphic objects are not pixels but lines, rectangles, circles and so
on. Although technically everything on the screen consists of pixels
{\XGAP} remembers the structure of your graphics via higher objects. This
has advantages as well as disadvantages. Do not expect to be able to
place pixel images into your {\XGAP} graphic sheets. That is as of now
*not possible* with {\XGAP} and probably will never be, because it is not
the idea of the design. 

What you can do is create, move around and change lines, circles, text and
so forth in graphic sheets. Your programs can communicate with the user via
graphical user interfaces like mouse, menus, dialogs, and so on.

It is very easy to link this graphical environment with your programs in
the mathematical environment of {\GAP}. So you can very quickly implement
visualizations of the mathematical objects you study. The user can select
objects, choose functions from menus and ask for more information with a
few mouse clicks. 

A good example for this approach is the implementation of the interactive
Todd-Coxeter-Algorithm to compute coset tables in finitely presented
groups. It uses the graphical features of {\XGAP} to give the user quick
and easy access to the algorithm by a few mouse clicks. This program was
written by Ludger Hippe in Aachen using {\XGAP3} and is currently ported to
{\XGAP4} and extended by Volkmar Felsch.

Another nice little example is in the `examples' subdirectory in the
{\XGAP} distribution. It was written by Thomas Breuer (Aachen) to
demonstrate the features of {\XGAP}. The user gets a small window with a
puzzle and can solve it using the mouse. You can test this example by
starting {\XGAP} and `Read'ing the file `pkg/xgap/examples/puzzle.g'.
You can do this by using

\begintt
gap> ReadPkg("xgap","examples/puzzle.g");
gap> p := Puzzle(4,4);
\endtt

You do not have to invent the wheel many times. For certain mathematical
concepts like graphs, posets or lattices {\XGAP} provides implementations
which can  
be adapted to your special situation. You can use those parts of the code
you like and substitute the other parts to adapt the behaviour of the user
interface to your wishes.


%%%%%%%%%%%%%%%%%%%%%%%%%%%%%%%%%%%%%%%%%%%%%%%%%%%%%%%%%%%%%%%%%%%%%%%%%%%%%
\Section{How does it work?}

{\XGAP} consists of a C program `xgap' (in the following `xgap' in
typewriter style refers to this C part) separate from {\GAP}, and of some
libraries written in the {\GAP} language. `xgap' is started by the
user and launches a {\GAP} job in the background. It then talks 
to this {\GAP} job. Especially it displays all the output which comes from
{\GAP} in the communication window and feeds everything the user types in
this window into the {\GAP} job. 

But there is also some communication with the {\GAP} job about the graphics
that should be displayed. Because {\GAP} has no concept of putting graphics
on the screen, this part is done by `xgap'. Therefore there is a protocol
between the {\GAP} part of {\XGAP} running in the {\GAP} session and `xgap' 
which is embedded in the input/output stream. The user does not notice
this. `xgap' stores all windows and graphic objects and does all the work
necessary for displaying windows and managing user communication and so on.

The {\GAP} part of {\XGAP} also stores all graphical information, but in
form of {\GAP} objects. The user can inspect all these structures and use
them in own programs. Changes in these structures are transmitted through
the communications protocol to `xgap' and are eventually displayed on the
screen. User actions like mouse clicks or keyboard events are caught by
`xgap' and transmitted to the {\GAP} job via function calls that are
``typed in'' as if the user had typed them. So the library can work on them 
and change the {\GAP} objects accordingly.

Technically, {\XGAP} is a package and one of the first commands that
is executed automatically within the {\GAP} session is a
`RequirePackage("xgap")' call. This reads in the extra {\XGAP}
libraries. They are found in the `pkg/xgap/lib' subdirectory, normally in the
main {\GAP} directory. The files contain the following:

\beginitems
`window.g' & basic definitions for the communications protocol

`sheet.g[di]' & graphic sheets and their operations 

`color.g[di]' & color information

`font.g[di]'  & text font information

`menu.g[di]'  & menus, dialogs, popups, and user communication

`gobject.g[di]'& low level graphic objects and their operations

`poset.g[di]' & graphic graphs and graphic posets

`ilatgrp.g[di]'& graphic subgroup lattices

`meataxe.g[di]'& support to display submodule lattices calculated within
the C MeatAxe
\enditems

The user normally does not need to know this or the details of it. However, 
it is important to understand that the program `xgap' is highly machine or
at least operating system dependent. There is no generic way to access
graphics across different platforms up to now. {\XGAP} should run on all
variants of Unix with the X Window System Version 11 Release 5 or
higher. As of now {\XGAP} does not run on Microsoft Windows or MacOS. It is 
also definitely *not* easily ported there, because some important features
that are used within {\XGAP} are missing there (such as pseudo terminals).
There are currently no plans underways to do work in this direction.
However, portations are very welcome! If you would like to put some work
into this, please contact Max Neunh\accent127offer (email:
`neunhoef@mcs.st-and.ac.uk'). 

%%%%%%%%%%%%%%%%%%%%%%%%%%%%%%%%%%%%%%%%%%%%%%%%%%%%%%%%%%%%%%%%%%%%%%%%%%%%%
\Section{Historical Remarks and Acknowledgements}

A first program for drawing a diagram showing the lattice of subgroups
of  a finite  group  that  had  been   calculated by  a  computer  was
implemented by H. J\accent127urgensen in 1965 (see \cite{FJ65}).
% [quote K. Ferber, H. Juergensen,
%A   programme for drawing    a  lattice.  p.  83 -   86  of J.  Leech,
%ed. Computational Problems in Abstract Algebra, 1970]

The  design of {\XGAP} was strongly  influenced and  in fact triggered by
the  QUOTPIC system of  Derek Holt  and  Sarah Rees (see \cite{HR91})
%[quote  D.Holt, S.
%Rees, A  Graphics System for Displaying  Finite Quotients  of Finitely
%Presented Groups  p. 113 - 126  of L. Finkelstei,  W.M. Kantor, Groups
%and  Computation, DIMACS 1991]
%
which allows to depict graphically knowledge about normal subgroups of
a finitely presented group found by a variety of methods for the
investigation of finitely presented groups.  It seemed most desirable
to allow to depict in a similar way the even wider variety of
information on subgroups of groups that can be obtained by a system
such as {\GAP}.

Beginning 1993,  Frank Celler developed the  idea of an interface from
{\GAP}   to graphic systems that  allowed  to actually write commands for
graphical   tasks  in  the {\GAP}  language   and  together with  Susanne
Keitemeier  (see \cite{SK95}) wrote a first version
of   programs in {\XGAP}  for   drawing diagrams  representing posets  of
subgroups of finite and finitely presented groups.  We most gratefully
acknowledge  the  help of  Sarah Rees in  implementing the interactive
lattice functions and in beta testing the {\GAP3} version of {\XGAP}.

In 1998, Thomas Breuer, Frank Celler, Joachim Neub\accent127user and Max
Neunh\accent127offer planned the new concepts for the {\GAP4}
version. The implementation and portation to {\GAP4} was done mainly by Max
Neunh\accent127offer in 1998 and 1999. Michael Ringe added the link to his
MeatAxe programs. We like to thank all those who have adapted the {\GAP} 
library to the needs of the new {\XGAP}, in particular Alexander Hulpke who 
has been extremely helpful with this task.

