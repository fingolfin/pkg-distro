%%%%%%%%%%%%%%%%%%%%%%%%%%%%%%%%%%%%%%%%%%%%%%%%%%%%%%%%%%%%%%%%%%%%%%%%%
%%
%W  install.tex            GAP documentation            Christian Sievers
%%
%Y  2003 - 2012
%%

%%%%%%%%%%%%%%%%%%%%%%%%%%%%%%%%%%%%%%%%%%%%%%%%%%%%%%%%%%%%%%%%%%%%%%%%%
\Chapter{Installing and loading the FGA package}

\atindex{Installing and loading the FGA package}{@Installing %
                        and loading the {\FGA} package|indexit}
%%%%%%%%%%%%%%%%%%%%%%%%%%%%%%%%%%%%%%%%%%%%%%%%%%%%%%%%%%%%%%%%%%%%%%%%%
\Section{Installing the FGA package}\null

\atindex{Installing the FGA package}{@Installing %
                                      the {\FGA} package|indexit}
The installation of the {\FGA} package follows standard {\GAP} rules.
So the standard method is to unpack the archive into the `pkg'
directory  of your {\GAP} distribution.  This will create an `fga'
subdirectory. 

For other non-standard options please see Chapter~"ref:Installing a
GAP Package" in the {\GAP} Reference Manual.

%To create the documentation, go into the `doc' directory and type
%`make_doc'.

%%%%%%%%%%%%%%%%%%%%%%%%%%%%%%%%%%%%%%%%%%%%%%%%%%%%%%%%%%%%%%%%%%%%%%%%%
\Section{Loading the FGA package}\null

\atindex{Loading the FGA package}{@loading the {\FGA} package|indexit}
The {\FGA} package is configured to autoload, so its functionality is
usually available once {\GAP} is started.

If {\GAP} does not autoload, you can request the package with the
`LoadPackage' command like this:

\testexamplefalse
\beginexample
gap> LoadPackage( "fga" );
-----------------------------------------------------------------------------
Loading  FGA 1.2.0 (Free Group Algorithms)
by Christian Sievers (c.sievers@tu-bs.de).
-----------------------------------------------------------------------------
true
\endexample

You will not see the banner if {\FGA} has already been loaded.

The `LoadPackage' command and ways to disable autoloading are
described in Section~"ref:Loading a GAP Package" in the {\GAP}
Reference Manual.

%%%%%%%%%%%%%%%%%%%%%%%%%%%%%%%%%%%%%%%%%%%%%%%%%%%%%%%%%%%%%%%%%%%%%%%%%
%%
%E
