%%%%%%%%%%%%%%%%%%%%%%%%%%%%%%%%%%%%%%%%%%%%%%%%%%%%%%%%%%%%%%%%%%%%%%%%%
%%
%W  intro.tex          Radiroot documentation             Andreas Distler
%%
%H  $Id$
%%
%Y  2005
%%

%%%%%%%%%%%%%%%%%%%%%%%%%%%%%%%%%%%%%%%%%%%%%%%%%%%%%%%%%%%%%%%%%%%%%%%%%
\Chapter{Introduction}

This package provides functionality to deal with one of the fundamental
problems in algebra. The roots of a rational polynomial shall be
expressed by radicals. This means one is only allowed to use the four
basic operations $(+, -, \cdot \,,\div)$ and to extract roots. For
example, a radical expression for the roots of the polynomial $x^4 -
x^3 - x^2 + x + 1$ is 
$$
\matrix{
\frac{1}{4} + \frac{1}{4}\sqrt{-3} + \frac{1}{2}\sqrt{\frac{7}{2} +
  \frac{1}{2}\sqrt{-3}}.
}
$$

There are formulas to solve the general equation $x^n+
a_{n-1}x^{n-1}+ \dots + a_1x+a_0 = 0$ up to degree 4. For higher
degrees such formulae do not exist (\cite{Abel26}). It was \'Evariste Galois
(1811 -- 1832) who discovered that there exists a radical expression
for the roots if and only if the Galois group of the polynomial - initially a
permutation group on the roots - is solvable \cite{Galois97}. But the task
itself was impractical in his days. This package is the first public tool
which provides a practical method for solving a polynomial algebraically. The
implementation is based on Galois' ideas and the algorithm is described in
\cite{Distler05}. 

The package can provide the result in various forms. As a default an
expression is given in a similar way as in the example
above. Alternatively, a file containing the roots might be created
which is readable by Maple \cite{Maple10}. In {\GAP} itself some
information deduced during the computation is available.

The user should be aware that radical expressions can get very complicated
even for polynomials of small degree. Especially because the algorithm
will find an irreducible radical expression. That means one gets a
root of the given polynomial for every choice of a value of the radicals in
the expression. Moreover it is not the aim of this package to give a
simplest expression, in any sense.

In Chapter 2 the methods provided by this package are listed and
explained.

Chapter 3 gives details about the info class of this package. See
Section "ref:Info Functions" in the {\GAP} reference manual for
general information about info classes.

While the installation of the package follows standard {\GAP} rules
the Chapter 4 contains information about external programs required by
{\Radiroot} in its default setup.

% In its default use the package creates a LaTex file for the radical
%expression and displays the according dvi-file. Therefore you need a Latex
%compiler and the dvi viewer xdvi, to use the main functionality.  

This package uses the interface in the package \Alnuth, to
factorise polynomials over algebraic number fields. This functionality must
be available to use the functions in {\Radiroot}.  

%%%%%%%%%%%%%%%%%%%%%%%%%%%%%%%%%%%%%%%%%%%%%%%%%%%%%%%%%%%%%%%%%%
\Section{License}

This package is free software; you can redistribute it and/or modify
it under the terms of the GNU General Public License as published by
the Free Software Foundation; version 2 of the License.

This program is distributed in the hope that it will be useful, but
WITHOUT ANY WARRANTY; without even the implied warranty of
MERCHANTABILITY or FITNESS FOR A PARTICULAR PURPOSE. See the GNU
General Public License for more details.

%%%%%%%%%%%%%%%%%%%%%%%%%%%%%%%%%%%%%%%%%%%%%%%%%%%%%%%%%%%%%%%%%%%%%%%%%
%%
%E
