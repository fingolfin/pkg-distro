% generated by GAPDoc2LaTeX from XML source (Frank Luebeck)
\documentclass[a4paper,11pt]{report}

\usepackage{a4wide}
\sloppy
\pagestyle{myheadings}
\usepackage{amssymb}
\usepackage[utf8]{inputenc}
\usepackage{makeidx}
\makeindex
\usepackage{color}
\definecolor{FireBrick}{rgb}{0.5812,0.0074,0.0083}
\definecolor{RoyalBlue}{rgb}{0.0236,0.0894,0.6179}
\definecolor{RoyalGreen}{rgb}{0.0236,0.6179,0.0894}
\definecolor{RoyalRed}{rgb}{0.6179,0.0236,0.0894}
\definecolor{LightBlue}{rgb}{0.8544,0.9511,1.0000}
\definecolor{Black}{rgb}{0.0,0.0,0.0}

\definecolor{linkColor}{rgb}{0.0,0.0,0.554}
\definecolor{citeColor}{rgb}{0.0,0.0,0.554}
\definecolor{fileColor}{rgb}{0.0,0.0,0.554}
\definecolor{urlColor}{rgb}{0.0,0.0,0.554}
\definecolor{promptColor}{rgb}{0.0,0.0,0.589}
\definecolor{brkpromptColor}{rgb}{0.589,0.0,0.0}
\definecolor{gapinputColor}{rgb}{0.589,0.0,0.0}
\definecolor{gapoutputColor}{rgb}{0.0,0.0,0.0}

%%  for a long time these were red and blue by default,
%%  now black, but keep variables to overwrite
\definecolor{FuncColor}{rgb}{0.0,0.0,0.0}
%% strange name because of pdflatex bug:
\definecolor{Chapter }{rgb}{0.0,0.0,0.0}
\definecolor{DarkOlive}{rgb}{0.1047,0.2412,0.0064}


\usepackage{fancyvrb}

\usepackage{mathptmx,helvet}
\usepackage[T1]{fontenc}
\usepackage{textcomp}


\usepackage[
            pdftex=true,
            bookmarks=true,        
            a4paper=true,
            pdftitle={Written with GAPDoc},
            pdfcreator={LaTeX with hyperref package / GAPDoc},
            colorlinks=true,
            backref=page,
            breaklinks=true,
            linkcolor=linkColor,
            citecolor=citeColor,
            filecolor=fileColor,
            urlcolor=urlColor,
            pdfpagemode={UseNone}, 
           ]{hyperref}

\newcommand{\maintitlesize}{\fontsize{50}{55}\selectfont}

% write page numbers to a .pnr log file for online help
\newwrite\pagenrlog
\immediate\openout\pagenrlog =\jobname.pnr
\immediate\write\pagenrlog{PAGENRS := [}
\newcommand{\logpage}[1]{\protect\write\pagenrlog{#1, \thepage,}}
%% were never documented, give conflicts with some additional packages

\newcommand{\GAP}{\textsf{GAP}}

%% nicer description environments, allows long labels
\usepackage{enumitem}
\setdescription{style=nextline}

%% depth of toc
\setcounter{tocdepth}{1}





%% command for ColorPrompt style examples
\newcommand{\gapprompt}[1]{\color{promptColor}{\bfseries #1}}
\newcommand{\gapbrkprompt}[1]{\color{brkpromptColor}{\bfseries #1}}
\newcommand{\gapinput}[1]{\color{gapinputColor}{#1}}


\begin{document}

\logpage{[ 0, 0, 0 ]}
\begin{titlepage}
\mbox{}\vfill

\begin{center}{\maintitlesize \textbf{\textsf{ToricVarieties}\mbox{}}}\\
\vfill

\hypersetup{pdftitle=\textsf{ToricVarieties}}
\markright{\scriptsize \mbox{}\hfill \textsf{ToricVarieties} \hfill\mbox{}}
{\Huge \textbf{A \textsf{GAP} package for handling toric varieties.\mbox{}}}\\
\vfill

{\Huge Version 2012.12.22\mbox{}}\\[1cm]
{October 2012\mbox{}}\\[1cm]
\mbox{}\\[2cm]
{\Large \textbf{Sebastian Gutsche\\
    \mbox{}}}\\
\hypersetup{pdfauthor=Sebastian Gutsche\\
    }
\mbox{}\\[2cm]
\begin{minipage}{12cm}\noindent
 \\
\\
 This manual is best viewed as an \textsc{HTML} document. An \textsc{offline} version should be included in the documentation subfolder of the package. \\
\\
 \end{minipage}

\end{center}\vfill

\mbox{}\\
{\mbox{}\\
\small \noindent \textbf{Sebastian Gutsche\\
    }  Email: \href{mailto://sebastian.gutsche@rwth-aachen.de} {\texttt{sebastian.gutsche@rwth-aachen.de}}\\
  Homepage: \href{http://wwwb.math.rwth-aachen.de/~gutsche} {\texttt{http://wwwb.math.rwth-aachen.de/\texttt{\symbol{126}}gutsche}}\\
  Address: \begin{minipage}[t]{8cm}\noindent
 Lehrstuhl B f{\"u}r Mathematik, RWTH Aachen, Templergraben 64, 52056 Aachen,
Germany \end{minipage}
}\\
\end{titlepage}

\newpage\setcounter{page}{2}
{\small 
\section*{Copyright}
\logpage{[ 0, 0, 1 ]}
 {\copyright} 2011-2012 by Sebastian Gutsche

 This package may be distributed under the terms and conditions of the GNU
Public License Version 2. \mbox{}}\\[1cm]
{\small 
\section*{Acknowledgements}
\logpage{[ 0, 0, 2 ]}
 \mbox{}}\\[1cm]
\newpage

\def\contentsname{Contents\logpage{[ 0, 0, 3 ]}}

\tableofcontents
\newpage

 \index{\textsf{ToricVarieties}}   
\chapter{\textcolor{Chapter }{Introduction}}\label{intro}
\logpage{[ 1, 0, 0 ]}
\hyperdef{L}{X7DFB63A97E67C0A1}{}
{
  
\section{\textcolor{Chapter }{What is the goal of the \textsf{ToricVarieties} package?}}\label{WhyToricVarieties}
\logpage{[ 1, 1, 0 ]}
\hyperdef{L}{X82D29B587A1E08FF}{}
{
  \textsf{ToricVarieties} provides data structures to handle toric varieties by their commutative
algebra structure and by their combinatorics. For combinatorics, it uses the \textsf{Convex} package. Its goal is to provide a suitable framework to work with toric
varieties. All combinatorial structures mentioned in this manual are the ones
from \textsf{Convex}. }

  }

   
\chapter{\textcolor{Chapter }{Installation of the \textsf{ToricVarieties} Package}}\label{install}
\logpage{[ 2, 0, 0 ]}
\hyperdef{L}{X7EC76C1D7F46724F}{}
{
  To install this package just extract the package's archive file to the \textsf{GAP} \texttt{pkg} directory.

 By default the \textsf{ToricVarieties} package is not automatically loaded by \textsf{GAP} when it is installed. You must load the package with \\
\\
 \texttt{LoadPackage}( "ToricVarieties" ); \\
\\
 before its functions become available.

 Please, send me an e-mail if you have any questions, remarks, suggestions,
etc. concerning this package. Also, I would be pleased to hear about
applications of this package and about any suggestions for new methods to add
to the package. \\
\\
\\
 Sebastian Gutsche  }

   
\chapter{\textcolor{Chapter }{Toric varieties}}\label{Varieties}
\logpage{[ 3, 0, 0 ]}
\hyperdef{L}{X866558FA7BC3F2C8}{}
{
  
\section{\textcolor{Chapter }{Toric variety: Category and Representations}}\label{ToricVariety:Category}
\logpage{[ 3, 1, 0 ]}
\hyperdef{L}{X8108B9978021989B}{}
{
  

\subsection{\textcolor{Chapter }{IsToricVariety}}
\logpage{[ 3, 1, 1 ]}\nobreak
\hyperdef{L}{X7A99B0697F11DEB1}{}
{\noindent\textcolor{FuncColor}{$\triangleright$\ \ \texttt{IsToricVariety({\mdseries\slshape M})\index{IsToricVariety@\texttt{IsToricVariety}}
\label{IsToricVariety}
}\hfill{\scriptsize (Category)}}\\
\textbf{\indent Returns:\ }
\texttt{true} or \texttt{false}



 The \textsf{GAP} category of a toric variety. }

 }

 
\section{\textcolor{Chapter }{Toric varieties: Properties}}\label{ToricVarieties:Properties}
\logpage{[ 3, 2, 0 ]}
\hyperdef{L}{X81C5B56F7A5E912E}{}
{
  

\subsection{\textcolor{Chapter }{IsNormalVariety}}
\logpage{[ 3, 2, 1 ]}\nobreak
\hyperdef{L}{X7D6DF89D7836A3D8}{}
{\noindent\textcolor{FuncColor}{$\triangleright$\ \ \texttt{IsNormalVariety({\mdseries\slshape vari})\index{IsNormalVariety@\texttt{IsNormalVariety}}
\label{IsNormalVariety}
}\hfill{\scriptsize (property)}}\\
\textbf{\indent Returns:\ }
\texttt{true} or \texttt{false}



 Checks if the toric variety \mbox{\texttt{\mdseries\slshape vari}} is a normal variety. }

 

\subsection{\textcolor{Chapter }{IsAffine}}
\logpage{[ 3, 2, 2 ]}\nobreak
\hyperdef{L}{X7E2687347A75468E}{}
{\noindent\textcolor{FuncColor}{$\triangleright$\ \ \texttt{IsAffine({\mdseries\slshape vari})\index{IsAffine@\texttt{IsAffine}}
\label{IsAffine}
}\hfill{\scriptsize (property)}}\\
\textbf{\indent Returns:\ }
\texttt{true} or \texttt{false}



 Checks if the toric variety \mbox{\texttt{\mdseries\slshape vari}} is an affine variety. }

 

\subsection{\textcolor{Chapter }{IsProjective}}
\logpage{[ 3, 2, 3 ]}\nobreak
\hyperdef{L}{X7EC041A77E7E46D2}{}
{\noindent\textcolor{FuncColor}{$\triangleright$\ \ \texttt{IsProjective({\mdseries\slshape vari})\index{IsProjective@\texttt{IsProjective}}
\label{IsProjective}
}\hfill{\scriptsize (property)}}\\
\textbf{\indent Returns:\ }
\texttt{true} or \texttt{false}



 Checks if the toric variety \mbox{\texttt{\mdseries\slshape vari}} is a projective variety. }

 

\subsection{\textcolor{Chapter }{IsComplete}}
\logpage{[ 3, 2, 4 ]}\nobreak
\hyperdef{L}{X7D689F21828A4278}{}
{\noindent\textcolor{FuncColor}{$\triangleright$\ \ \texttt{IsComplete({\mdseries\slshape vari})\index{IsComplete@\texttt{IsComplete}}
\label{IsComplete}
}\hfill{\scriptsize (property)}}\\
\textbf{\indent Returns:\ }
\texttt{true} or \texttt{false}



 Checks if the toric variety \mbox{\texttt{\mdseries\slshape vari}} is a complete variety. }

 

\subsection{\textcolor{Chapter }{IsSmooth}}
\logpage{[ 3, 2, 5 ]}\nobreak
\hyperdef{L}{X86CBF5497EC15CFC}{}
{\noindent\textcolor{FuncColor}{$\triangleright$\ \ \texttt{IsSmooth({\mdseries\slshape vari})\index{IsSmooth@\texttt{IsSmooth}}
\label{IsSmooth}
}\hfill{\scriptsize (property)}}\\
\textbf{\indent Returns:\ }
\texttt{true} or \texttt{false}



 Checks if the toric variety \mbox{\texttt{\mdseries\slshape vari}} is a smooth variety. }

 

\subsection{\textcolor{Chapter }{HasTorusfactor}}
\logpage{[ 3, 2, 6 ]}\nobreak
\hyperdef{L}{X87B517958002AE71}{}
{\noindent\textcolor{FuncColor}{$\triangleright$\ \ \texttt{HasTorusfactor({\mdseries\slshape vari})\index{HasTorusfactor@\texttt{HasTorusfactor}}
\label{HasTorusfactor}
}\hfill{\scriptsize (property)}}\\
\textbf{\indent Returns:\ }
\texttt{true} or \texttt{false}



 Checks if the toric variety \mbox{\texttt{\mdseries\slshape vari}} has a torus factor. }

 

\subsection{\textcolor{Chapter }{HasNoTorusfactor}}
\logpage{[ 3, 2, 7 ]}\nobreak
\hyperdef{L}{X83AF576586EFA7A6}{}
{\noindent\textcolor{FuncColor}{$\triangleright$\ \ \texttt{HasNoTorusfactor({\mdseries\slshape vari})\index{HasNoTorusfactor@\texttt{HasNoTorusfactor}}
\label{HasNoTorusfactor}
}\hfill{\scriptsize (property)}}\\
\textbf{\indent Returns:\ }
\texttt{true} or \texttt{false}



 Checks if the toric variety \mbox{\texttt{\mdseries\slshape vari}} has no torus factor. }

 

\subsection{\textcolor{Chapter }{IsOrbifold}}
\logpage{[ 3, 2, 8 ]}\nobreak
\hyperdef{L}{X78CBF9007E82E5DF}{}
{\noindent\textcolor{FuncColor}{$\triangleright$\ \ \texttt{IsOrbifold({\mdseries\slshape vari})\index{IsOrbifold@\texttt{IsOrbifold}}
\label{IsOrbifold}
}\hfill{\scriptsize (property)}}\\
\textbf{\indent Returns:\ }
\texttt{true} or \texttt{false}



 Checks if the toric variety \mbox{\texttt{\mdseries\slshape vari}} has an orbifold, which is, in the toric case, equivalent to the simpliciality
of the fan. }

 }

 
\section{\textcolor{Chapter }{Toric varieties: Attributes}}\label{ToricVarieties:Attributes}
\logpage{[ 3, 3, 0 ]}
\hyperdef{L}{X7AA03F947802BFA6}{}
{
  

\subsection{\textcolor{Chapter }{AffineOpenCovering}}
\logpage{[ 3, 3, 1 ]}\nobreak
\hyperdef{L}{X82AB95E9870A50A6}{}
{\noindent\textcolor{FuncColor}{$\triangleright$\ \ \texttt{AffineOpenCovering({\mdseries\slshape vari})\index{AffineOpenCovering@\texttt{AffineOpenCovering}}
\label{AffineOpenCovering}
}\hfill{\scriptsize (attribute)}}\\
\textbf{\indent Returns:\ }
a list



 Returns a torus invariant affine open covering of the variety \mbox{\texttt{\mdseries\slshape vari}}. The affine open cover is computed out of the cones of the fan. }

 

\subsection{\textcolor{Chapter }{CoxRing}}
\logpage{[ 3, 3, 2 ]}\nobreak
\hyperdef{L}{X78F279E67EE26EBF}{}
{\noindent\textcolor{FuncColor}{$\triangleright$\ \ \texttt{CoxRing({\mdseries\slshape vari})\index{CoxRing@\texttt{CoxRing}}
\label{CoxRing}
}\hfill{\scriptsize (attribute)}}\\
\textbf{\indent Returns:\ }
a ring



 Returns the Cox ring of the variety \mbox{\texttt{\mdseries\slshape vari}}. The actual method requires a string with a name for the variables. A method
for computing the Cox ring without a variable given is not implemented. You
will get an error. }

 

\subsection{\textcolor{Chapter }{ListOfVariablesOfCoxRing}}
\logpage{[ 3, 3, 3 ]}\nobreak
\hyperdef{L}{X87C2F08C84052EB9}{}
{\noindent\textcolor{FuncColor}{$\triangleright$\ \ \texttt{ListOfVariablesOfCoxRing({\mdseries\slshape vari})\index{ListOfVariablesOfCoxRing@\texttt{ListOfVariablesOfCoxRing}}
\label{ListOfVariablesOfCoxRing}
}\hfill{\scriptsize (attribute)}}\\
\textbf{\indent Returns:\ }
a list



 Returns a list of the variables of the cox ring of the variety \mbox{\texttt{\mdseries\slshape vari}}. }

 

\subsection{\textcolor{Chapter }{ClassGroup}}
\logpage{[ 3, 3, 4 ]}\nobreak
\hyperdef{L}{X872024617ADBE423}{}
{\noindent\textcolor{FuncColor}{$\triangleright$\ \ \texttt{ClassGroup({\mdseries\slshape vari})\index{ClassGroup@\texttt{ClassGroup}}
\label{ClassGroup}
}\hfill{\scriptsize (attribute)}}\\
\textbf{\indent Returns:\ }
a module



 Returns the class group of the variety \mbox{\texttt{\mdseries\slshape vari}} as factor of a free module. }

 

\subsection{\textcolor{Chapter }{PicardGroup}}
\logpage{[ 3, 3, 5 ]}\nobreak
\hyperdef{L}{X854A7BDA84D12EEC}{}
{\noindent\textcolor{FuncColor}{$\triangleright$\ \ \texttt{PicardGroup({\mdseries\slshape vari})\index{PicardGroup@\texttt{PicardGroup}}
\label{PicardGroup}
}\hfill{\scriptsize (attribute)}}\\
\textbf{\indent Returns:\ }
a module



 Returns the Picard group of the variety \mbox{\texttt{\mdseries\slshape vari}} as factor of a free module. }

 

\subsection{\textcolor{Chapter }{TorusInvariantDivisorGroup}}
\logpage{[ 3, 3, 6 ]}\nobreak
\hyperdef{L}{X8025428782FE96E1}{}
{\noindent\textcolor{FuncColor}{$\triangleright$\ \ \texttt{TorusInvariantDivisorGroup({\mdseries\slshape vari})\index{TorusInvariantDivisorGroup@\texttt{TorusInvariantDivisorGroup}}
\label{TorusInvariantDivisorGroup}
}\hfill{\scriptsize (attribute)}}\\
\textbf{\indent Returns:\ }
a module



 Returns the subgroup of the Weil divisor group of the variety \mbox{\texttt{\mdseries\slshape vari}} generated by the torus invariant prime divisors. This is always a finitely
generated free module over the integers. }

 

\subsection{\textcolor{Chapter }{MapFromCharacterToPrincipalDivisor}}
\logpage{[ 3, 3, 7 ]}\nobreak
\hyperdef{L}{X86539CAC7DFFA60B}{}
{\noindent\textcolor{FuncColor}{$\triangleright$\ \ \texttt{MapFromCharacterToPrincipalDivisor({\mdseries\slshape vari})\index{MapFromCharacterToPrincipalDivisor@\texttt{MapFromCharacterToPrincipalDivisor}}
\label{MapFromCharacterToPrincipalDivisor}
}\hfill{\scriptsize (attribute)}}\\
\textbf{\indent Returns:\ }
a morphism



 Returns a map which maps an element of the character group into the torus
invariant Weil group of the variety \mbox{\texttt{\mdseries\slshape vari}}. This has to viewn as an help method to compute divisor classes. }

 

\subsection{\textcolor{Chapter }{Dimension}}
\logpage{[ 3, 3, 8 ]}\nobreak
\hyperdef{L}{X7E6926C6850E7C4E}{}
{\noindent\textcolor{FuncColor}{$\triangleright$\ \ \texttt{Dimension({\mdseries\slshape vari})\index{Dimension@\texttt{Dimension}}
\label{Dimension}
}\hfill{\scriptsize (attribute)}}\\
\textbf{\indent Returns:\ }
an integer



 Returns the dimension of the variety \mbox{\texttt{\mdseries\slshape vari}}. }

 

\subsection{\textcolor{Chapter }{DimensionOfTorusfactor}}
\logpage{[ 3, 3, 9 ]}\nobreak
\hyperdef{L}{X7843850F8735A926}{}
{\noindent\textcolor{FuncColor}{$\triangleright$\ \ \texttt{DimensionOfTorusfactor({\mdseries\slshape vari})\index{DimensionOfTorusfactor@\texttt{DimensionOfTorusfactor}}
\label{DimensionOfTorusfactor}
}\hfill{\scriptsize (attribute)}}\\
\textbf{\indent Returns:\ }
an integer



 Returns the dimension of the torus factor of the variety \mbox{\texttt{\mdseries\slshape vari}}. }

 

\subsection{\textcolor{Chapter }{CoordinateRingOfTorus}}
\logpage{[ 3, 3, 10 ]}\nobreak
\hyperdef{L}{X8514A91A7CEA7092}{}
{\noindent\textcolor{FuncColor}{$\triangleright$\ \ \texttt{CoordinateRingOfTorus({\mdseries\slshape vari})\index{CoordinateRingOfTorus@\texttt{CoordinateRingOfTorus}}
\label{CoordinateRingOfTorus}
}\hfill{\scriptsize (attribute)}}\\
\textbf{\indent Returns:\ }
a ring



 Returns the coordinate ring of the torus of the variety \mbox{\texttt{\mdseries\slshape vari}}. This method is not implemented, you need to call it with a second argument,
which is a list of strings for the variables of the ring. }

 

\subsection{\textcolor{Chapter }{IsProductOf}}
\logpage{[ 3, 3, 11 ]}\nobreak
\hyperdef{L}{X876EF14B845E275E}{}
{\noindent\textcolor{FuncColor}{$\triangleright$\ \ \texttt{IsProductOf({\mdseries\slshape vari})\index{IsProductOf@\texttt{IsProductOf}}
\label{IsProductOf}
}\hfill{\scriptsize (attribute)}}\\
\textbf{\indent Returns:\ }
a list



 If the variety \mbox{\texttt{\mdseries\slshape vari}} is a product of 2 or more varieties, the list contain those varieties. If it
is not a product or at least not generated as a product, the list only
contains the variety itself. }

 

\subsection{\textcolor{Chapter }{CharacterLattice}}
\logpage{[ 3, 3, 12 ]}\nobreak
\hyperdef{L}{X81E2EA227D69040A}{}
{\noindent\textcolor{FuncColor}{$\triangleright$\ \ \texttt{CharacterLattice({\mdseries\slshape vari})\index{CharacterLattice@\texttt{CharacterLattice}}
\label{CharacterLattice}
}\hfill{\scriptsize (attribute)}}\\
\textbf{\indent Returns:\ }
a module



 The method returns the character lattice of the variety \mbox{\texttt{\mdseries\slshape vari}}, computed as the containing grid of the underlying convex object, if it
exists. }

 

\subsection{\textcolor{Chapter }{TorusInvariantPrimeDivisors}}
\logpage{[ 3, 3, 13 ]}\nobreak
\hyperdef{L}{X84594CD787F6BA94}{}
{\noindent\textcolor{FuncColor}{$\triangleright$\ \ \texttt{TorusInvariantPrimeDivisors({\mdseries\slshape vari})\index{TorusInvariantPrimeDivisors@\texttt{TorusInvariantPrimeDivisors}}
\label{TorusInvariantPrimeDivisors}
}\hfill{\scriptsize (attribute)}}\\
\textbf{\indent Returns:\ }
a list



 The method returns a list of the torus invariant prime divisors of the variety \mbox{\texttt{\mdseries\slshape vari}}. }

 

\subsection{\textcolor{Chapter }{IrrelevantIdeal}}
\logpage{[ 3, 3, 14 ]}\nobreak
\hyperdef{L}{X78BB13787BA1C31C}{}
{\noindent\textcolor{FuncColor}{$\triangleright$\ \ \texttt{IrrelevantIdeal({\mdseries\slshape vari})\index{IrrelevantIdeal@\texttt{IrrelevantIdeal}}
\label{IrrelevantIdeal}
}\hfill{\scriptsize (attribute)}}\\
\textbf{\indent Returns:\ }
an ideal



 Returns the irrelevant ideal of the cox ring of the variety \mbox{\texttt{\mdseries\slshape vari}}. }

 

\subsection{\textcolor{Chapter }{MorphismFromCoxVariety}}
\logpage{[ 3, 3, 15 ]}\nobreak
\hyperdef{L}{X78CFA34884706A16}{}
{\noindent\textcolor{FuncColor}{$\triangleright$\ \ \texttt{MorphismFromCoxVariety({\mdseries\slshape vari})\index{MorphismFromCoxVariety@\texttt{MorphismFromCoxVariety}}
\label{MorphismFromCoxVariety}
}\hfill{\scriptsize (attribute)}}\\
\textbf{\indent Returns:\ }
a morphism



 The method returns the quotient morphism from the variety of the Cox ring to
the variety \mbox{\texttt{\mdseries\slshape vari}}. }

 

\subsection{\textcolor{Chapter }{CoxVariety}}
\logpage{[ 3, 3, 16 ]}\nobreak
\hyperdef{L}{X8761518B7F4F6C58}{}
{\noindent\textcolor{FuncColor}{$\triangleright$\ \ \texttt{CoxVariety({\mdseries\slshape vari})\index{CoxVariety@\texttt{CoxVariety}}
\label{CoxVariety}
}\hfill{\scriptsize (attribute)}}\\
\textbf{\indent Returns:\ }
a variety



 The method returns the Cox variety of the variety \mbox{\texttt{\mdseries\slshape vari}}. }

 

\subsection{\textcolor{Chapter }{FanOfVariety}}
\logpage{[ 3, 3, 17 ]}\nobreak
\hyperdef{L}{X7F89CB52790F3E87}{}
{\noindent\textcolor{FuncColor}{$\triangleright$\ \ \texttt{FanOfVariety({\mdseries\slshape vari})\index{FanOfVariety@\texttt{FanOfVariety}}
\label{FanOfVariety}
}\hfill{\scriptsize (attribute)}}\\
\textbf{\indent Returns:\ }
a fan



 Returns the fan of the variety \mbox{\texttt{\mdseries\slshape vari}}. This is set by default. }

 

\subsection{\textcolor{Chapter }{CartierTorusInvariantDivisorGroup}}
\logpage{[ 3, 3, 18 ]}\nobreak
\hyperdef{L}{X7BEC4BCD7B3B3522}{}
{\noindent\textcolor{FuncColor}{$\triangleright$\ \ \texttt{CartierTorusInvariantDivisorGroup({\mdseries\slshape vari})\index{CartierTorusInvariantDivisorGroup@\texttt{CartierTorusInvariantDivisorGroup}}
\label{CartierTorusInvariantDivisorGroup}
}\hfill{\scriptsize (attribute)}}\\
\textbf{\indent Returns:\ }
a module



 Returns the the group of Cartier divisors of the variety \mbox{\texttt{\mdseries\slshape vari}} as a subgroup of the divisor group. }

 

\subsection{\textcolor{Chapter }{NameOfVariety}}
\logpage{[ 3, 3, 19 ]}\nobreak
\hyperdef{L}{X853D172E78C7D0B2}{}
{\noindent\textcolor{FuncColor}{$\triangleright$\ \ \texttt{NameOfVariety({\mdseries\slshape vari})\index{NameOfVariety@\texttt{NameOfVariety}}
\label{NameOfVariety}
}\hfill{\scriptsize (attribute)}}\\
\textbf{\indent Returns:\ }
a string



 Returns the name of the variety \mbox{\texttt{\mdseries\slshape vari}} if it has one and it is known or can be computed. }

 

\subsection{\textcolor{Chapter }{twitter}}
\logpage{[ 3, 3, 20 ]}\nobreak
\hyperdef{L}{X820E750381030706}{}
{\noindent\textcolor{FuncColor}{$\triangleright$\ \ \texttt{twitter({\mdseries\slshape vari})\index{twitter@\texttt{twitter}}
\label{twitter}
}\hfill{\scriptsize (attribute)}}\\
\textbf{\indent Returns:\ }
a ring



 This is a dummy to get immediate methods triggered at some times. It never has
a value. }

 }

 
\section{\textcolor{Chapter }{Toric varieties: Methods}}\label{ToricVarieties:Methods}
\logpage{[ 3, 4, 0 ]}
\hyperdef{L}{X866EE174808EA7F9}{}
{
  

\subsection{\textcolor{Chapter }{UnderlyingSheaf}}
\logpage{[ 3, 4, 1 ]}\nobreak
\hyperdef{L}{X7DB5B6CB86F766A5}{}
{\noindent\textcolor{FuncColor}{$\triangleright$\ \ \texttt{UnderlyingSheaf({\mdseries\slshape vari})\index{UnderlyingSheaf@\texttt{UnderlyingSheaf}}
\label{UnderlyingSheaf}
}\hfill{\scriptsize (operation)}}\\
\textbf{\indent Returns:\ }
a sheaf



 The method returns the underlying sheaf of the variety \mbox{\texttt{\mdseries\slshape vari}}. }

 

\subsection{\textcolor{Chapter }{CoordinateRingOfTorus (for a variety and a list of variables)}}
\logpage{[ 3, 4, 2 ]}\nobreak
\hyperdef{L}{X7D01CCCE78FA1FDE}{}
{\noindent\textcolor{FuncColor}{$\triangleright$\ \ \texttt{CoordinateRingOfTorus({\mdseries\slshape vari, vars})\index{CoordinateRingOfTorus@\texttt{CoordinateRingOfTorus}!for a variety and a list of variables}
\label{CoordinateRingOfTorus:for a variety and a list of variables}
}\hfill{\scriptsize (operation)}}\\
\textbf{\indent Returns:\ }
a ring



 Computes the coordinate ring of the torus of the variety \mbox{\texttt{\mdseries\slshape vari}} with the variables \mbox{\texttt{\mdseries\slshape vars}}. The argument \mbox{\texttt{\mdseries\slshape vars}} need to be a list of strings with length dimension or two times dimension. }

 

\subsection{\textcolor{Chapter }{\texttt{\symbol{92}}*}}
\logpage{[ 3, 4, 3 ]}\nobreak
\hyperdef{L}{X7857704878577048}{}
{\noindent\textcolor{FuncColor}{$\triangleright$\ \ \texttt{\texttt{\symbol{92}}*({\mdseries\slshape vari1, vari2})\index{*@\texttt{\texttt{\symbol{92}}*}}
\label{*}
}\hfill{\scriptsize (operation)}}\\
\textbf{\indent Returns:\ }
a variety



 Computes the categorial product of the varieties \mbox{\texttt{\mdseries\slshape vari1}} and \mbox{\texttt{\mdseries\slshape vari2}}. }

 

\subsection{\textcolor{Chapter }{CharacterToRationalFunction}}
\logpage{[ 3, 4, 4 ]}\nobreak
\hyperdef{L}{X80DBA6A18199A4A4}{}
{\noindent\textcolor{FuncColor}{$\triangleright$\ \ \texttt{CharacterToRationalFunction({\mdseries\slshape elem, vari})\index{CharacterToRationalFunction@\texttt{CharacterToRationalFunction}}
\label{CharacterToRationalFunction}
}\hfill{\scriptsize (operation)}}\\
\textbf{\indent Returns:\ }
a homalg element



 Computes the rational function corresponding to the character grid element \mbox{\texttt{\mdseries\slshape elem}} or to the list of integers \mbox{\texttt{\mdseries\slshape elem}}. To compute rational functions you first need to compute to coordinate ring
of the torus of the variety \mbox{\texttt{\mdseries\slshape vari}}. }

 

\subsection{\textcolor{Chapter }{CoxRing (for a variety and a string of variables)}}
\logpage{[ 3, 4, 5 ]}\nobreak
\hyperdef{L}{X80917C0C82171774}{}
{\noindent\textcolor{FuncColor}{$\triangleright$\ \ \texttt{CoxRing({\mdseries\slshape vari, vars})\index{CoxRing@\texttt{CoxRing}!for a variety and a string of variables}
\label{CoxRing:for a variety and a string of variables}
}\hfill{\scriptsize (operation)}}\\
\textbf{\indent Returns:\ }
a ring



 Computes the Cox ring of the variety \mbox{\texttt{\mdseries\slshape vari}}. \mbox{\texttt{\mdseries\slshape vars}} needs to be a string containing one variable, which will be numbered by the
method. }

 

\subsection{\textcolor{Chapter }{WeilDivisorsOfVariety}}
\logpage{[ 3, 4, 6 ]}\nobreak
\hyperdef{L}{X79474EA085374986}{}
{\noindent\textcolor{FuncColor}{$\triangleright$\ \ \texttt{WeilDivisorsOfVariety({\mdseries\slshape vari})\index{WeilDivisorsOfVariety@\texttt{WeilDivisorsOfVariety}}
\label{WeilDivisorsOfVariety}
}\hfill{\scriptsize (operation)}}\\
\textbf{\indent Returns:\ }
a list



 Returns a list of the currently defined Divisors of the toric variety. }

 

\subsection{\textcolor{Chapter }{Fan}}
\logpage{[ 3, 4, 7 ]}\nobreak
\hyperdef{L}{X80D0196B80DC94F3}{}
{\noindent\textcolor{FuncColor}{$\triangleright$\ \ \texttt{Fan({\mdseries\slshape vari})\index{Fan@\texttt{Fan}}
\label{Fan}
}\hfill{\scriptsize (operation)}}\\
\textbf{\indent Returns:\ }
a fan



 Returns the fan of the variety \mbox{\texttt{\mdseries\slshape vari}}. This is a rename for FanOfVariety. }

 }

 
\section{\textcolor{Chapter }{Toric varieties: Constructors}}\label{ToricVarieties:Constructors}
\logpage{[ 3, 5, 0 ]}
\hyperdef{L}{X7C1E65F7809F51A7}{}
{
  

\subsection{\textcolor{Chapter }{ToricVariety}}
\logpage{[ 3, 5, 1 ]}\nobreak
\hyperdef{L}{X84CA1FBC8057E3E0}{}
{\noindent\textcolor{FuncColor}{$\triangleright$\ \ \texttt{ToricVariety({\mdseries\slshape conv})\index{ToricVariety@\texttt{ToricVariety}}
\label{ToricVariety}
}\hfill{\scriptsize (operation)}}\\
\textbf{\indent Returns:\ }
a ring



 Creates a toric variety out of the convex object \mbox{\texttt{\mdseries\slshape conv}}. }

 }

 
\section{\textcolor{Chapter }{Toric varieties: Examples}}\label{ToricVarieties:Examples}
\logpage{[ 3, 6, 0 ]}
\hyperdef{L}{X802337377FDC8121}{}
{
  
\subsection{\textcolor{Chapter }{The Hirzebruch surface of index 5}}\label{Hirzebruch5Example}
\logpage{[ 3, 6, 1 ]}
\hyperdef{L}{X7F674AD387A33155}{}
{
  
\begin{Verbatim}[commandchars=!@E,fontsize=\small,frame=single,label=Example]
  !gapprompt@gap>E !gapinput@H5 := Fan( [[-1,5],[0,1],[1,0],[0,-1]],[[1,2],[2,3],[3,4],[4,1]] );E
  <A fan in |R^2>
  !gapprompt@gap>E !gapinput@H5 := ToricVariety( H5 );E
  <A toric variety of dimension 2>
  !gapprompt@gap>E !gapinput@IsComplete( H5 );E
  true
  !gapprompt@gap>E !gapinput@IsAffine( H5 );E
  false
  !gapprompt@gap>E !gapinput@IsOrbifold( H5 );E
  true
  !gapprompt@gap>E !gapinput@IsProjective( H5 );E
  true
  !gapprompt@gap>E !gapinput@TorusInvariantPrimeDivisors(H5);E
  [ <A prime divisor of a toric variety with coordinates [ 1, 0, 0, 0 ]>,
    <A prime divisor of a toric variety with coordinates [ 0, 1, 0, 0 ]>, 
    <A prime divisor of a toric variety with coordinates [ 0, 0, 1, 0 ]>,
    <A prime divisor of a toric variety with coordinates [ 0, 0, 0, 1 ]> ]
  !gapprompt@gap>E !gapinput@P := TorusInvariantPrimeDivisors(H5);E
  [ <A prime divisor of a toric variety with coordinates [ 1, 0, 0, 0 ]>,
    <A prime divisor of a toric variety with coordinates [ 0, 1, 0, 0 ]>, 
    <A prime divisor of a toric variety with coordinates [ 0, 0, 1, 0 ]>, 
    <A prime divisor of a toric variety with coordinates [ 0, 0, 0, 1 ]> ]
  !gapprompt@gap>E !gapinput@A := P[ 1 ] - P[ 2 ] + 4*P[ 3 ];E
  <A divisor of a toric variety with coordinates [ 1, -1, 4, 0 ]>
  !gapprompt@gap>E !gapinput@A;E
  <A divisor of a toric variety with coordinates [ 1, -1, 4, 0 ]>
  !gapprompt@gap>E !gapinput@IsAmple(A);E
  false
  !gapprompt@gap>E !gapinput@CoordinateRingOfTorus(H5,"x");;E
  Q[x1,x1_,x2,x2_]/( x2*x2_-1, x1*x1_-1 )
  !gapprompt@gap>E !gapinput@D:=CreateDivisor([0,0,0,0],H5);E
  <A divisor of a toric variety with coordinates 0>
  !gapprompt@gap>E !gapinput@BasisOfGlobalSections(D);E
  [ |[ 1 ]| ]
  !gapprompt@gap>E !gapinput@D:=Sum(P);E
  <A divisor of a toric variety with coordinates [ 1, 1, 1, 1 ]>
  !gapprompt@gap>E !gapinput@BasisOfGlobalSections(D);E
  [ |[ x1_ ]|, |[ x1_*x2 ]|, |[ 1 ]|, |[ x2 ]|,
    |[ x1 ]|, |[ x1*x2 ]|, |[ x1^2*x2 ]|, 
    |[ x1^3*x2 ]|, |[ x1^4*x2 ]|, |[ x1^5*x2 ]|, 
    |[ x1^6*x2 ]| ]
  !gapprompt@gap>E !gapinput@DivisorOfCharacter([1,2],H5);E
  <A principal divisor of a toric variety with coordinates [ 9, 2, 1, -2 ]>
  !gapprompt@gap>E !gapinput@BasisOfGlobalSections(last);E
  [ |[ x1_*x2_^2 ]| ]
\end{Verbatim}
}

 }

  }

   
\chapter{\textcolor{Chapter }{Toric subvarieties}}\label{Subvarieties}
\logpage{[ 4, 0, 0 ]}
\hyperdef{L}{X84370283823C138C}{}
{
  
\section{\textcolor{Chapter }{Toric subvarieties: Category and Representations}}\label{Subvarieties:Category}
\logpage{[ 4, 1, 0 ]}
\hyperdef{L}{X7A22F3137FA25458}{}
{
  

\subsection{\textcolor{Chapter }{IsToricSubvariety}}
\logpage{[ 4, 1, 1 ]}\nobreak
\hyperdef{L}{X85CA472F7A14BF8C}{}
{\noindent\textcolor{FuncColor}{$\triangleright$\ \ \texttt{IsToricSubvariety({\mdseries\slshape M})\index{IsToricSubvariety@\texttt{IsToricSubvariety}}
\label{IsToricSubvariety}
}\hfill{\scriptsize (Category)}}\\
\textbf{\indent Returns:\ }
\texttt{true} or \texttt{false}



 The \textsf{GAP} category of a toric subvariety. Every toric subvariety is a toric variety, so
every method applicable to toric varieties is also applicable to toric
subvarieties. }

 }

 
\section{\textcolor{Chapter }{Toric subvarieties: Properties}}\label{Subvarieties:Properties}
\logpage{[ 4, 2, 0 ]}
\hyperdef{L}{X826EDD1B846E74B0}{}
{
  

\subsection{\textcolor{Chapter }{IsClosed}}
\logpage{[ 4, 2, 1 ]}\nobreak
\hyperdef{L}{X81D5A4A97AA9D4B0}{}
{\noindent\textcolor{FuncColor}{$\triangleright$\ \ \texttt{IsClosed({\mdseries\slshape vari})\index{IsClosed@\texttt{IsClosed}}
\label{IsClosed}
}\hfill{\scriptsize (property)}}\\
\textbf{\indent Returns:\ }
\texttt{true} or \texttt{false}



 Checks if the subvariety \mbox{\texttt{\mdseries\slshape vari}} is a closed subset of its ambient variety. }

 

\subsection{\textcolor{Chapter }{IsOpen}}
\logpage{[ 4, 2, 2 ]}\nobreak
\hyperdef{L}{X8247435184B2DE47}{}
{\noindent\textcolor{FuncColor}{$\triangleright$\ \ \texttt{IsOpen({\mdseries\slshape vari})\index{IsOpen@\texttt{IsOpen}}
\label{IsOpen}
}\hfill{\scriptsize (property)}}\\
\textbf{\indent Returns:\ }
\texttt{true} or \texttt{false}



 Checks if a subvariety is a closed subset. }

 

\subsection{\textcolor{Chapter }{IsWholeVariety}}
\logpage{[ 4, 2, 3 ]}\nobreak
\hyperdef{L}{X7A9968777C5E19A4}{}
{\noindent\textcolor{FuncColor}{$\triangleright$\ \ \texttt{IsWholeVariety({\mdseries\slshape vari})\index{IsWholeVariety@\texttt{IsWholeVariety}}
\label{IsWholeVariety}
}\hfill{\scriptsize (property)}}\\
\textbf{\indent Returns:\ }
\texttt{true} or \texttt{false}



 Returns true if the subvariety \mbox{\texttt{\mdseries\slshape vari}} is the whole variety. }

 }

 
\section{\textcolor{Chapter }{Toric subvarieties: Attributes}}\label{Subvarieties:Attributes}
\logpage{[ 4, 3, 0 ]}
\hyperdef{L}{X790B57E084CC7198}{}
{
  

\subsection{\textcolor{Chapter }{UnderlyingToricVariety}}
\logpage{[ 4, 3, 1 ]}\nobreak
\hyperdef{L}{X7D9CEB4A878CC07C}{}
{\noindent\textcolor{FuncColor}{$\triangleright$\ \ \texttt{UnderlyingToricVariety({\mdseries\slshape vari})\index{UnderlyingToricVariety@\texttt{UnderlyingToricVariety}}
\label{UnderlyingToricVariety}
}\hfill{\scriptsize (attribute)}}\\
\textbf{\indent Returns:\ }
a variety



 Returns the toric variety which is represented by \mbox{\texttt{\mdseries\slshape vari}}. This method implements the forgetful functor subvarieties -{\textgreater}
varieties. }

 

\subsection{\textcolor{Chapter }{InclusionMorphism}}
\logpage{[ 4, 3, 2 ]}\nobreak
\hyperdef{L}{X84AE669679A57F17}{}
{\noindent\textcolor{FuncColor}{$\triangleright$\ \ \texttt{InclusionMorphism({\mdseries\slshape vari})\index{InclusionMorphism@\texttt{InclusionMorphism}}
\label{InclusionMorphism}
}\hfill{\scriptsize (attribute)}}\\
\textbf{\indent Returns:\ }
a morphism



 If the variety \mbox{\texttt{\mdseries\slshape vari}} is an open subvariety, this method returns the inclusion morphism in its
ambient variety. If not, it will fail. }

 

\subsection{\textcolor{Chapter }{AmbientToricVariety}}
\logpage{[ 4, 3, 3 ]}\nobreak
\hyperdef{L}{X87ADD8677D3DC498}{}
{\noindent\textcolor{FuncColor}{$\triangleright$\ \ \texttt{AmbientToricVariety({\mdseries\slshape vari})\index{AmbientToricVariety@\texttt{AmbientToricVariety}}
\label{AmbientToricVariety}
}\hfill{\scriptsize (attribute)}}\\
\textbf{\indent Returns:\ }
a variety



 Returns the ambient toric variety of the subvariety \mbox{\texttt{\mdseries\slshape vari}} }

 }

 
\section{\textcolor{Chapter }{Toric subvarieties: Methods}}\label{Subvarieties:Methods}
\logpage{[ 4, 4, 0 ]}
\hyperdef{L}{X7B50D40D858C8B3C}{}
{
  

\subsection{\textcolor{Chapter }{ClosureOfTorusOrbitOfCone}}
\logpage{[ 4, 4, 1 ]}\nobreak
\hyperdef{L}{X7E9E173183C04931}{}
{\noindent\textcolor{FuncColor}{$\triangleright$\ \ \texttt{ClosureOfTorusOrbitOfCone({\mdseries\slshape vari, cone})\index{ClosureOfTorusOrbitOfCone@\texttt{ClosureOfTorusOrbitOfCone}}
\label{ClosureOfTorusOrbitOfCone}
}\hfill{\scriptsize (operation)}}\\
\textbf{\indent Returns:\ }
a subvariety



 The method returns the closure of the orbit of the torus contained in \mbox{\texttt{\mdseries\slshape vari}} which corresponds to the cone \mbox{\texttt{\mdseries\slshape cone}} as a closed subvariety of \mbox{\texttt{\mdseries\slshape vari}}. }

 }

 
\section{\textcolor{Chapter }{Toric subvarieties: Constructors}}\label{Subvarieties:Constructors}
\logpage{[ 4, 5, 0 ]}
\hyperdef{L}{X7C5DB208861E7E7F}{}
{
  

\subsection{\textcolor{Chapter }{ToricSubvariety}}
\logpage{[ 4, 5, 1 ]}\nobreak
\hyperdef{L}{X851CDD807D40B7EF}{}
{\noindent\textcolor{FuncColor}{$\triangleright$\ \ \texttt{ToricSubvariety({\mdseries\slshape vari, ambvari})\index{ToricSubvariety@\texttt{ToricSubvariety}}
\label{ToricSubvariety}
}\hfill{\scriptsize (operation)}}\\
\textbf{\indent Returns:\ }
a subvariety



 The method returns the closure of the orbit of the torus contained in \mbox{\texttt{\mdseries\slshape vari}} which corresponds to the cone \mbox{\texttt{\mdseries\slshape cone}} as a closed subvariety of \mbox{\texttt{\mdseries\slshape vari}}. }

 }

  }

   
\chapter{\textcolor{Chapter }{Affine toric varieties}}\label{AffineVariety}
\logpage{[ 5, 0, 0 ]}
\hyperdef{L}{X82F418F483E4D0D6}{}
{
  
\section{\textcolor{Chapter }{Affine toric varieties: Category and Representations}}\label{AffineVariety:Category}
\logpage{[ 5, 1, 0 ]}
\hyperdef{L}{X83355FC284165BD4}{}
{
  

\subsection{\textcolor{Chapter }{IsAffineToricVariety}}
\logpage{[ 5, 1, 1 ]}\nobreak
\hyperdef{L}{X7ED0399F81CBB82D}{}
{\noindent\textcolor{FuncColor}{$\triangleright$\ \ \texttt{IsAffineToricVariety({\mdseries\slshape M})\index{IsAffineToricVariety@\texttt{IsAffineToricVariety}}
\label{IsAffineToricVariety}
}\hfill{\scriptsize (Category)}}\\
\textbf{\indent Returns:\ }
\texttt{true} or \texttt{false}



 The \textsf{GAP} category of an affine toric variety. All affine toric varieties are toric
varieties, so everything applicable to toric varieties is applicable to affine
toric varieties. }

 }

 
\section{\textcolor{Chapter }{Affine toric varieties: Properties}}\label{AffineVariety:Properties}
\logpage{[ 5, 2, 0 ]}
\hyperdef{L}{X80DB08C5837B0A49}{}
{
  Affine toric varieties have no additional properties. Remember that affine
toric varieties are toric varieties, so every property of a toric variety is a
property of an affine toric variety. }

 
\section{\textcolor{Chapter }{Affine toric varieties: Attributes}}\label{AffineVariety:Attributes}
\logpage{[ 5, 3, 0 ]}
\hyperdef{L}{X7BBE823E8205F52F}{}
{
  

\subsection{\textcolor{Chapter }{CoordinateRing}}
\logpage{[ 5, 3, 1 ]}\nobreak
\hyperdef{L}{X81BCE73D8353F9DE}{}
{\noindent\textcolor{FuncColor}{$\triangleright$\ \ \texttt{CoordinateRing({\mdseries\slshape vari})\index{CoordinateRing@\texttt{CoordinateRing}}
\label{CoordinateRing}
}\hfill{\scriptsize (attribute)}}\\
\textbf{\indent Returns:\ }
a ring



 Returns the coordinate ring of the affine toric variety \mbox{\texttt{\mdseries\slshape vari}}. The computation is mainly done in ToricIdeals package. }

 

\subsection{\textcolor{Chapter }{ListOfVariablesOfCoordinateRing}}
\logpage{[ 5, 3, 2 ]}\nobreak
\hyperdef{L}{X7F459CD178502F4E}{}
{\noindent\textcolor{FuncColor}{$\triangleright$\ \ \texttt{ListOfVariablesOfCoordinateRing({\mdseries\slshape vari})\index{ListOfVariablesOfCoordinateRing@\texttt{ListOfVariablesOfCoordinateRing}}
\label{ListOfVariablesOfCoordinateRing}
}\hfill{\scriptsize (attribute)}}\\
\textbf{\indent Returns:\ }
a list



 Returns a list containing the variables of the CoordinateRing of the variety \mbox{\texttt{\mdseries\slshape vari}}. }

 

\subsection{\textcolor{Chapter }{MorphismFromCoordinateRingToCoordinateRingOfTorus}}
\logpage{[ 5, 3, 3 ]}\nobreak
\hyperdef{L}{X82A61ED17E17D0C2}{}
{\noindent\textcolor{FuncColor}{$\triangleright$\ \ \texttt{MorphismFromCoordinateRingToCoordinateRingOfTorus({\mdseries\slshape vari})\index{MorphismFromCoordinateRingToCoordinateRingOfTorus@\texttt{Morphism}\-\texttt{From}\-\texttt{Coordinate}\-\texttt{Ring}\-\texttt{To}\-\texttt{Coordinate}\-\texttt{Ring}\-\texttt{Of}\-\texttt{Torus}}
\label{MorphismFromCoordinateRingToCoordinateRingOfTorus}
}\hfill{\scriptsize (attribute)}}\\
\textbf{\indent Returns:\ }
a morphism



 Returns the morphism between the coordinate ring of the variety \mbox{\texttt{\mdseries\slshape vari}} and the coordinate ring of its torus. This defines the embedding of the torus
in the variety. }

 

\subsection{\textcolor{Chapter }{ConeOfVariety}}
\logpage{[ 5, 3, 4 ]}\nobreak
\hyperdef{L}{X8642EA7886E09E44}{}
{\noindent\textcolor{FuncColor}{$\triangleright$\ \ \texttt{ConeOfVariety({\mdseries\slshape vari})\index{ConeOfVariety@\texttt{ConeOfVariety}}
\label{ConeOfVariety}
}\hfill{\scriptsize (attribute)}}\\
\textbf{\indent Returns:\ }
a cone



 Returns the cone ring of the affine toric variety \mbox{\texttt{\mdseries\slshape vari}}. }

 }

 
\section{\textcolor{Chapter }{Affine toric varieties: Methods}}\label{AffineVariety:Methods}
\logpage{[ 5, 4, 0 ]}
\hyperdef{L}{X8012A86C8008CC21}{}
{
  

\subsection{\textcolor{Chapter }{CoordinateRing (for affine Varieties)}}
\logpage{[ 5, 4, 1 ]}\nobreak
\hyperdef{L}{X85926C9D8411CCC1}{}
{\noindent\textcolor{FuncColor}{$\triangleright$\ \ \texttt{CoordinateRing({\mdseries\slshape vari, indet})\index{CoordinateRing@\texttt{CoordinateRing}!for affine Varieties}
\label{CoordinateRing:for affine Varieties}
}\hfill{\scriptsize (operation)}}\\
\textbf{\indent Returns:\ }
a variety



 Computes the coordinate ring of the affine toric variety \mbox{\texttt{\mdseries\slshape vari}} with indeterminates \mbox{\texttt{\mdseries\slshape indet}}. }

 

\subsection{\textcolor{Chapter }{Cone}}
\logpage{[ 5, 4, 2 ]}\nobreak
\hyperdef{L}{X822975FC7F646FE5}{}
{\noindent\textcolor{FuncColor}{$\triangleright$\ \ \texttt{Cone({\mdseries\slshape vari})\index{Cone@\texttt{Cone}}
\label{Cone}
}\hfill{\scriptsize (operation)}}\\
\textbf{\indent Returns:\ }
a cone



 Returns the cone of the variety \mbox{\texttt{\mdseries\slshape vari}}. Another name for ConeOfVariety for compatibility and shortness. }

 }

 
\section{\textcolor{Chapter }{Affine toric varieties: Constructors}}\label{AffineVariety:Constructors}
\logpage{[ 5, 5, 0 ]}
\hyperdef{L}{X846A65F77C4BEA35}{}
{
  The constructors are the same as for toric varieties. Calling them with a cone
will result in an affine variety. }

 
\section{\textcolor{Chapter }{Affine toric Varieties: Examples}}\label{AffineVariety:Examples}
\logpage{[ 5, 6, 0 ]}
\hyperdef{L}{X8068F91F7C001DE7}{}
{
  
\subsection{\textcolor{Chapter }{Affine space}}\label{AffineSpaceExampleSubsection}
\logpage{[ 5, 6, 1 ]}
\hyperdef{L}{X782DF75D8761D85B}{}
{
  
\begin{Verbatim}[commandchars=!@B,fontsize=\small,frame=single,label=Example]
  !gapprompt@gap>B !gapinput@C:=Cone( [[1,0,0],[0,1,0],[0,0,1]] );B
  <A cone in |R^3>
  !gapprompt@gap>B !gapinput@C3:=ToricVariety(C);B
  <An affine normal toric variety of dimension 3>
  !gapprompt@gap>B !gapinput@Dimension(C3);B
  3
  !gapprompt@gap>B !gapinput@IsOrbifold(C3);B
  true
  !gapprompt@gap>B !gapinput@IsSmooth(C3);B
  true
  !gapprompt@gap>B !gapinput@CoordinateRingOfTorus(C3,"x");B
  Q[x1,x1_,x2,x2_,x3,x3_]/( x3*x3_-1, x2*x2_-1, x1*x1_-1 )
  !gapprompt@gap>B !gapinput@CoordinateRing(C3,"x");B
  Q[x_1,x_2,x_3]
  !gapprompt@gap>B !gapinput@MorphismFromCoordinateRingToCoordinateRingOfTorus(C3);B
  <A monomorphism of rings>
  !gapprompt@gap>B !gapinput@C3;B
  <An affine normal smooth toric variety of dimension 3>
  !gapprompt@gap>B !gapinput@StructureDescription(C3);B
  "|A^3"
\end{Verbatim}
}

 }

  }

   
\chapter{\textcolor{Chapter }{Projective toric varieties}}\label{ProjectiveVariety}
\logpage{[ 6, 0, 0 ]}
\hyperdef{L}{X7EEBFF7883297DBC}{}
{
  
\section{\textcolor{Chapter }{Projective toric varieties: Category and Representations}}\label{ProjectiveVariety:Category}
\logpage{[ 6, 1, 0 ]}
\hyperdef{L}{X7A7AC40C780CAF75}{}
{
  

\subsection{\textcolor{Chapter }{IsProjectiveToricVariety}}
\logpage{[ 6, 1, 1 ]}\nobreak
\hyperdef{L}{X7D7A64ED86E8C558}{}
{\noindent\textcolor{FuncColor}{$\triangleright$\ \ \texttt{IsProjectiveToricVariety({\mdseries\slshape M})\index{IsProjectiveToricVariety@\texttt{IsProjectiveToricVariety}}
\label{IsProjectiveToricVariety}
}\hfill{\scriptsize (Category)}}\\
\textbf{\indent Returns:\ }
\texttt{true} or \texttt{false}



 The \textsf{GAP} category of a projective toric variety. }

 }

 
\section{\textcolor{Chapter }{Projective toric varieties: Properties}}\label{ProjectiveVariety:Properties}
\logpage{[ 6, 2, 0 ]}
\hyperdef{L}{X826634D3848FC540}{}
{
  Projective toric varieties have no additional properties. Remember that
projective toric varieties are toric varieties, so every property of a toric
variety is a property of an projective toric variety. }

 
\section{\textcolor{Chapter }{Projective toric varieties: Attributes}}\label{ProjectiveVariety:Attributes}
\logpage{[ 6, 3, 0 ]}
\hyperdef{L}{X7903BE287F71B26E}{}
{
  

\subsection{\textcolor{Chapter }{AffineCone}}
\logpage{[ 6, 3, 1 ]}\nobreak
\hyperdef{L}{X7C3748B8878B799A}{}
{\noindent\textcolor{FuncColor}{$\triangleright$\ \ \texttt{AffineCone({\mdseries\slshape vari})\index{AffineCone@\texttt{AffineCone}}
\label{AffineCone}
}\hfill{\scriptsize (attribute)}}\\
\textbf{\indent Returns:\ }
a variety



 Returns the affine cone of the projective toric variety \mbox{\texttt{\mdseries\slshape vari}}. }

 

\subsection{\textcolor{Chapter }{PolytopeOfVariety}}
\logpage{[ 6, 3, 2 ]}\nobreak
\hyperdef{L}{X791054957C1EE370}{}
{\noindent\textcolor{FuncColor}{$\triangleright$\ \ \texttt{PolytopeOfVariety({\mdseries\slshape vari})\index{PolytopeOfVariety@\texttt{PolytopeOfVariety}}
\label{PolytopeOfVariety}
}\hfill{\scriptsize (attribute)}}\\
\textbf{\indent Returns:\ }
a polytope



 Returns the polytope corresponding to the projective toric variety \mbox{\texttt{\mdseries\slshape vari}}, if it exists. }

 

\subsection{\textcolor{Chapter }{ProjectiveEmbedding}}
\logpage{[ 6, 3, 3 ]}\nobreak
\hyperdef{L}{X7CD1FB1E83F80D5F}{}
{\noindent\textcolor{FuncColor}{$\triangleright$\ \ \texttt{ProjectiveEmbedding({\mdseries\slshape vari})\index{ProjectiveEmbedding@\texttt{ProjectiveEmbedding}}
\label{ProjectiveEmbedding}
}\hfill{\scriptsize (attribute)}}\\
\textbf{\indent Returns:\ }
a list



 Returns characters for a closed embedding in an projective space for the
projective toric variety \mbox{\texttt{\mdseries\slshape vari}}. }

 }

 
\section{\textcolor{Chapter }{Projective toric varieties: Methods}}\label{ProjectiveVariety:Methods}
\logpage{[ 6, 4, 0 ]}
\hyperdef{L}{X7D9E26467C26EFE6}{}
{
  

\subsection{\textcolor{Chapter }{Polytope}}
\logpage{[ 6, 4, 1 ]}\nobreak
\hyperdef{L}{X855106007DE72898}{}
{\noindent\textcolor{FuncColor}{$\triangleright$\ \ \texttt{Polytope({\mdseries\slshape vari})\index{Polytope@\texttt{Polytope}}
\label{Polytope}
}\hfill{\scriptsize (operation)}}\\
\textbf{\indent Returns:\ }
a polytope



 Returns the polytope of the variety \mbox{\texttt{\mdseries\slshape vari}}. Another name for PolytopeOfVariety for compatibility and shortness. }

 }

 
\section{\textcolor{Chapter }{Projective toric varieties: Constructors}}\label{ProjectiveVariety:Constructors}
\logpage{[ 6, 5, 0 ]}
\hyperdef{L}{X81EFA5668447D0A4}{}
{
  The constructors are the same as for toric varieties. Calling them with a
polytope will result in an projective variety. }

 
\section{\textcolor{Chapter }{Projective toric varieties: Examples}}\label{ProjectiveVariety:Examples}
\logpage{[ 6, 6, 0 ]}
\hyperdef{L}{X8403EE76819E200F}{}
{
  
\subsection{\textcolor{Chapter }{PxP1 created by a polytope}}\label{P1P1PolytopeExampleSubsection}
\logpage{[ 6, 6, 1 ]}
\hyperdef{L}{X802DB11784310E99}{}
{
  
\begin{Verbatim}[commandchars=!@B,fontsize=\small,frame=single,label=Example]
  !gapprompt@gap>B !gapinput@P1P1 := Polytope( [[1,1],[1,-1],[-1,-1],[-1,1]] );B
  <A polytope in |R^2>
  !gapprompt@gap>B !gapinput@P1P1 := ToricVariety( P1P1 );B
  <A projective toric variety of dimension 2>
  !gapprompt@gap>B !gapinput@IsProjective( P1P1 );B
  true
  !gapprompt@gap>B !gapinput@IsComplete( P1P1 );B
  true 
  !gapprompt@gap>B !gapinput@CoordinateRingOfTorus( P1P1, "x" );B
  Q[x1,x1_,x2,x2_]/( x2*x2_-1, x1*x1_-1 )
  !gapprompt@gap>B !gapinput@IsVeryAmple( Polytope( P1P1 ) );B
  true
  !gapprompt@gap>B !gapinput@ProjectiveEmbedding( P1P1 );B
  [ |[ x1_*x2_ ]|, |[ x1_ ]|, |[ x1_*x2 ]|, |[ x2_ ]|,
  |[ 1 ]|, |[ x2 ]|, |[ x1*x2_ ]|, |[ x1 ]|, |[ x1*x2 ]| ]
  !gapprompt@gap>B !gapinput@Length( last );B
  9
\end{Verbatim}
}

 }

  }

   
\chapter{\textcolor{Chapter }{Toric morphisms}}\label{Morphisms}
\logpage{[ 7, 0, 0 ]}
\hyperdef{L}{X7FA18F537F3F5237}{}
{
  
\section{\textcolor{Chapter }{Toric morphisms: Category and Representations}}\label{Morphisms:Category}
\logpage{[ 7, 1, 0 ]}
\hyperdef{L}{X82A09A77805957FA}{}
{
  

\subsection{\textcolor{Chapter }{IsToricMorphism}}
\logpage{[ 7, 1, 1 ]}\nobreak
\hyperdef{L}{X85AC9E7B86C259AF}{}
{\noindent\textcolor{FuncColor}{$\triangleright$\ \ \texttt{IsToricMorphism({\mdseries\slshape M})\index{IsToricMorphism@\texttt{IsToricMorphism}}
\label{IsToricMorphism}
}\hfill{\scriptsize (Category)}}\\
\textbf{\indent Returns:\ }
\texttt{true} or \texttt{false}



 The \textsf{GAP} category of toric morphisms. A toric morphism is defined by a grid
homomorphism, which is compatible with the fan structure of the two varieties. }

 }

 
\section{\textcolor{Chapter }{Toric morphisms: Properties}}\label{Morphisms:Properties}
\logpage{[ 7, 2, 0 ]}
\hyperdef{L}{X82BECE3A7EA231D1}{}
{
  

\subsection{\textcolor{Chapter }{IsMorphism}}
\logpage{[ 7, 2, 1 ]}\nobreak
\hyperdef{L}{X7F66120A814DC16B}{}
{\noindent\textcolor{FuncColor}{$\triangleright$\ \ \texttt{IsMorphism({\mdseries\slshape morph})\index{IsMorphism@\texttt{IsMorphism}}
\label{IsMorphism}
}\hfill{\scriptsize (property)}}\\
\textbf{\indent Returns:\ }
\texttt{true} or \texttt{false}



 Checks if the grid morphism \mbox{\texttt{\mdseries\slshape morph}} respects the fan structure. }

 

\subsection{\textcolor{Chapter }{IsProper}}
\logpage{[ 7, 2, 2 ]}\nobreak
\hyperdef{L}{X7B15A1848387B3C8}{}
{\noindent\textcolor{FuncColor}{$\triangleright$\ \ \texttt{IsProper({\mdseries\slshape morph})\index{IsProper@\texttt{IsProper}}
\label{IsProper}
}\hfill{\scriptsize (property)}}\\
\textbf{\indent Returns:\ }
\texttt{true} or \texttt{false}



 Checks if the defined morphism \mbox{\texttt{\mdseries\slshape morph}} is proper. }

 }

 
\section{\textcolor{Chapter }{Toric morphisms: Attributes}}\label{Morphisms:Attributes}
\logpage{[ 7, 3, 0 ]}
\hyperdef{L}{X79DB44C17CFE1F59}{}
{
  

\subsection{\textcolor{Chapter }{SourceObject}}
\logpage{[ 7, 3, 1 ]}\nobreak
\hyperdef{L}{X7912BF2F79064BB9}{}
{\noindent\textcolor{FuncColor}{$\triangleright$\ \ \texttt{SourceObject({\mdseries\slshape morph})\index{SourceObject@\texttt{SourceObject}}
\label{SourceObject}
}\hfill{\scriptsize (attribute)}}\\
\textbf{\indent Returns:\ }
a variety



 Returns the source object of the morphism \mbox{\texttt{\mdseries\slshape morph}}. This attribute is a must have. }

 

\subsection{\textcolor{Chapter }{UnderlyingGridMorphism}}
\logpage{[ 7, 3, 2 ]}\nobreak
\hyperdef{L}{X81558ABE8360E43D}{}
{\noindent\textcolor{FuncColor}{$\triangleright$\ \ \texttt{UnderlyingGridMorphism({\mdseries\slshape morph})\index{UnderlyingGridMorphism@\texttt{UnderlyingGridMorphism}}
\label{UnderlyingGridMorphism}
}\hfill{\scriptsize (attribute)}}\\
\textbf{\indent Returns:\ }
a map



 Returns the grid map which defines \mbox{\texttt{\mdseries\slshape morph}}. }

 

\subsection{\textcolor{Chapter }{ToricImageObject}}
\logpage{[ 7, 3, 3 ]}\nobreak
\hyperdef{L}{X7F82AD6D7A967E25}{}
{\noindent\textcolor{FuncColor}{$\triangleright$\ \ \texttt{ToricImageObject({\mdseries\slshape morph})\index{ToricImageObject@\texttt{ToricImageObject}}
\label{ToricImageObject}
}\hfill{\scriptsize (attribute)}}\\
\textbf{\indent Returns:\ }
a variety



 Returns the variety which is created by the fan which is the image of the fan
of the source of \mbox{\texttt{\mdseries\slshape morph}}. This is not an image in the usual sense, but a toric image. }

 

\subsection{\textcolor{Chapter }{RangeObject}}
\logpage{[ 7, 3, 4 ]}\nobreak
\hyperdef{L}{X7ABADFCD80C507FB}{}
{\noindent\textcolor{FuncColor}{$\triangleright$\ \ \texttt{RangeObject({\mdseries\slshape morph})\index{RangeObject@\texttt{RangeObject}}
\label{RangeObject}
}\hfill{\scriptsize (attribute)}}\\
\textbf{\indent Returns:\ }
a variety



 Returns the range of the morphism \mbox{\texttt{\mdseries\slshape morph}}. If no range is given (yes, this is possible), the method returns the image. }

 

\subsection{\textcolor{Chapter }{MorphismOnWeilDivisorGroup}}
\logpage{[ 7, 3, 5 ]}\nobreak
\hyperdef{L}{X81B87EE486EFA0A4}{}
{\noindent\textcolor{FuncColor}{$\triangleright$\ \ \texttt{MorphismOnWeilDivisorGroup({\mdseries\slshape morph})\index{MorphismOnWeilDivisorGroup@\texttt{MorphismOnWeilDivisorGroup}}
\label{MorphismOnWeilDivisorGroup}
}\hfill{\scriptsize (attribute)}}\\
\textbf{\indent Returns:\ }
a morphism



 Returns the associated morphism between the divisor group of the range of \mbox{\texttt{\mdseries\slshape morph}} and the divisor group of the source. }

 

\subsection{\textcolor{Chapter }{ClassGroup (for toric morphisms)}}
\logpage{[ 7, 3, 6 ]}\nobreak
\hyperdef{L}{X7F542C0880E1598D}{}
{\noindent\textcolor{FuncColor}{$\triangleright$\ \ \texttt{ClassGroup({\mdseries\slshape morph})\index{ClassGroup@\texttt{ClassGroup}!for toric morphisms}
\label{ClassGroup:for toric morphisms}
}\hfill{\scriptsize (attribute)}}\\
\textbf{\indent Returns:\ }
a morphism



 Returns the associated morphism between the class groups of source and range
of the morphism \mbox{\texttt{\mdseries\slshape morph}} }

 

\subsection{\textcolor{Chapter }{MorphismOnCartierDivisorGroup}}
\logpage{[ 7, 3, 7 ]}\nobreak
\hyperdef{L}{X782D88777AA3E1F4}{}
{\noindent\textcolor{FuncColor}{$\triangleright$\ \ \texttt{MorphismOnCartierDivisorGroup({\mdseries\slshape morph})\index{MorphismOnCartierDivisorGroup@\texttt{MorphismOnCartierDivisorGroup}}
\label{MorphismOnCartierDivisorGroup}
}\hfill{\scriptsize (attribute)}}\\
\textbf{\indent Returns:\ }
a morphism



 Returns the associated morphism between the Cartier divisor groups of source
and range of the morphism \mbox{\texttt{\mdseries\slshape morph}} }

 

\subsection{\textcolor{Chapter }{PicardGroup (for toric morphisms)}}
\logpage{[ 7, 3, 8 ]}\nobreak
\hyperdef{L}{X804B025C873DF32C}{}
{\noindent\textcolor{FuncColor}{$\triangleright$\ \ \texttt{PicardGroup({\mdseries\slshape morph})\index{PicardGroup@\texttt{PicardGroup}!for toric morphisms}
\label{PicardGroup:for toric morphisms}
}\hfill{\scriptsize (attribute)}}\\
\textbf{\indent Returns:\ }
a morphism



 Returns the associated morphism between the class groups of source and range
of the morphism \mbox{\texttt{\mdseries\slshape morph}} }

 }

 
\section{\textcolor{Chapter }{Toric morphisms: Methods}}\label{Morphisms:Methods}
\logpage{[ 7, 4, 0 ]}
\hyperdef{L}{X7DE931AA84720706}{}
{
  

\subsection{\textcolor{Chapter }{UnderlyingListList}}
\logpage{[ 7, 4, 1 ]}\nobreak
\hyperdef{L}{X7FE514BB782DD216}{}
{\noindent\textcolor{FuncColor}{$\triangleright$\ \ \texttt{UnderlyingListList({\mdseries\slshape morph})\index{UnderlyingListList@\texttt{UnderlyingListList}}
\label{UnderlyingListList}
}\hfill{\scriptsize (attribute)}}\\
\textbf{\indent Returns:\ }
a list



 Returns a list of list which represents the grid homomorphism. }

 }

 
\section{\textcolor{Chapter }{Toric morphisms: Constructors}}\label{Morphisms:Constructors}
\logpage{[ 7, 5, 0 ]}
\hyperdef{L}{X7D94D43D8463F158}{}
{
  

\subsection{\textcolor{Chapter }{ToricMorphism (for a source and a matrix)}}
\logpage{[ 7, 5, 1 ]}\nobreak
\hyperdef{L}{X7B53B2C67B98043E}{}
{\noindent\textcolor{FuncColor}{$\triangleright$\ \ \texttt{ToricMorphism({\mdseries\slshape vari, lis})\index{ToricMorphism@\texttt{ToricMorphism}!for a source and a matrix}
\label{ToricMorphism:for a source and a matrix}
}\hfill{\scriptsize (operation)}}\\
\textbf{\indent Returns:\ }
a morphism



 Returns the toric morphism with source \mbox{\texttt{\mdseries\slshape vari}} which is represented by the matrix \mbox{\texttt{\mdseries\slshape lis}}. The range is set to the image. }

 

\subsection{\textcolor{Chapter }{ToricMorphism (for a source, matrix and target)}}
\logpage{[ 7, 5, 2 ]}\nobreak
\hyperdef{L}{X7AC924D486FADC47}{}
{\noindent\textcolor{FuncColor}{$\triangleright$\ \ \texttt{ToricMorphism({\mdseries\slshape vari, lis, vari2})\index{ToricMorphism@\texttt{ToricMorphism}!for a source, matrix and target}
\label{ToricMorphism:for a source, matrix and target}
}\hfill{\scriptsize (operation)}}\\
\textbf{\indent Returns:\ }
a morphism



 Returns the toric morphism with source \mbox{\texttt{\mdseries\slshape vari}} and range \mbox{\texttt{\mdseries\slshape vari2}} which is represented by the matrix \mbox{\texttt{\mdseries\slshape lis}}. }

 }

 
\section{\textcolor{Chapter }{Toric morphisms: Examples}}\label{Morphisms:Examples}
\logpage{[ 7, 6, 0 ]}
\hyperdef{L}{X83CE579A7B2021DE}{}
{
  
\subsection{\textcolor{Chapter }{Morphism between toric varieties and their class groups}}\label{MorphismExample}
\logpage{[ 7, 6, 1 ]}
\hyperdef{L}{X7D1DD8EA8098C432}{}
{
  
\begin{Verbatim}[commandchars=!@E,fontsize=\small,frame=single,label=Example]
  !gapprompt@gap>E !gapinput@P1 := Polytope([[0],[1]]);E
  <A polytope in |R^1>
  !gapprompt@gap>E !gapinput@P2 := Polytope([[0,0],[0,1],[1,0]]);E
  <A polytope in |R^2>
  !gapprompt@gap>E !gapinput@P1 := ToricVariety( P1 );E
  <A projective toric variety of dimension 1>
  !gapprompt@gap>E !gapinput@P2 := ToricVariety( P2 );E
  <A projective toric variety of dimension 2>
  !gapprompt@gap>E !gapinput@P1P2 := P1*P2;E
  <A projective toric variety of dimension 3
   which is a product of 2 toric varieties>
  !gapprompt@gap>E !gapinput@ClassGroup( P1 );E
  <A non-torsion left module presented by 1 relation for 2 generators>
  !gapprompt@gap>E !gapinput@Display(ByASmallerPresentation(last));E
  Z^(1 x 1)
  !gapprompt@gap>E !gapinput@ClassGroup( P2 );E
  <A non-torsion left module presented by 2 relations for 3 generators>
  !gapprompt@gap>E !gapinput@Display(ByASmallerPresentation(last));E
  Z^(1 x 1)
  !gapprompt@gap>E !gapinput@ClassGroup( P1P2 );E
  <A free left module of rank 2 on free generators>
  !gapprompt@gap>E !gapinput@Display( last );E
  Z^(1 x 2)
  !gapprompt@gap>E !gapinput@PicardGroup( P1P2 );E
  <A free left module of rank 2 on free generators>
  !gapprompt@gap>E !gapinput@P1P2;E
  <A projective smooth toric variety of dimension 3 
   which is a product of 2 toric varieties>
  !gapprompt@gap>E !gapinput@P2P1:=P2*P1;E
  <A projective toric variety of dimension 3 
   which is a product of 2 toric varieties>
  !gapprompt@gap>E !gapinput@M := [[0,0,1],[1,0,0],[0,1,0]];E
  [ [ 0, 0, 1 ], [ 1, 0, 0 ], [ 0, 1, 0 ] ]
  !gapprompt@gap>E !gapinput@M := ToricMorphism(P1P2,M,P2P1);E
  <A "homomorphism" of right objects>
  !gapprompt@gap>E !gapinput@IsMorphism(M);E
  true
  !gapprompt@gap>E !gapinput@ClassGroup(M);E
  <A homomorphism of left modules>
  !gapprompt@gap>E !gapinput@Display(last);E
  [ [  0,  1 ],
    [  1,  0 ] ]
  
  the map is currently represented by the above 2 x 2 matrix
  !gapprompt@gap>E !gapinput@ByASmallerPresentation(ClassGroup(M));E
  <A non-zero homomorphism of left modules>
  !gapprompt@gap>E !gapinput@Display(last);E
  [ [  0,  1 ],
    [  1,  0 ] ]
  
  the map is currently represented by the above 2 x 2 matrix
\end{Verbatim}
}

 }

  }

   
\chapter{\textcolor{Chapter }{Toric divisors}}\label{Divisors}
\logpage{[ 8, 0, 0 ]}
\hyperdef{L}{X7FDB76897833225A}{}
{
  
\section{\textcolor{Chapter }{Toric divisors: Category and Representations}}\label{Divisors:Category}
\logpage{[ 8, 1, 0 ]}
\hyperdef{L}{X7F19CA2285A48A77}{}
{
  

\subsection{\textcolor{Chapter }{IsToricDivisor}}
\logpage{[ 8, 1, 1 ]}\nobreak
\hyperdef{L}{X8664662078263B5A}{}
{\noindent\textcolor{FuncColor}{$\triangleright$\ \ \texttt{IsToricDivisor({\mdseries\slshape M})\index{IsToricDivisor@\texttt{IsToricDivisor}}
\label{IsToricDivisor}
}\hfill{\scriptsize (Category)}}\\
\textbf{\indent Returns:\ }
\texttt{true} or \texttt{false}



 The \textsf{GAP} category of torus invariant Weil divisors. }

 }

 
\section{\textcolor{Chapter }{Toric divisors: Properties}}\label{Divisors:Properties}
\logpage{[ 8, 2, 0 ]}
\hyperdef{L}{X7E508A068393A363}{}
{
  

\subsection{\textcolor{Chapter }{IsCartier}}
\logpage{[ 8, 2, 1 ]}\nobreak
\hyperdef{L}{X7E721696878A61F4}{}
{\noindent\textcolor{FuncColor}{$\triangleright$\ \ \texttt{IsCartier({\mdseries\slshape divi})\index{IsCartier@\texttt{IsCartier}}
\label{IsCartier}
}\hfill{\scriptsize (property)}}\\
\textbf{\indent Returns:\ }
\texttt{true} or \texttt{false}



 Checks if the torus invariant Weil divisor \mbox{\texttt{\mdseries\slshape divi}} is Cartier i.e. if it is locally principal. }

 

\subsection{\textcolor{Chapter }{IsPrincipal}}
\logpage{[ 8, 2, 2 ]}\nobreak
\hyperdef{L}{X84DE6ECE85F7D2F2}{}
{\noindent\textcolor{FuncColor}{$\triangleright$\ \ \texttt{IsPrincipal({\mdseries\slshape divi})\index{IsPrincipal@\texttt{IsPrincipal}}
\label{IsPrincipal}
}\hfill{\scriptsize (property)}}\\
\textbf{\indent Returns:\ }
\texttt{true} or \texttt{false}



 Checks if the torus invariant Weil divisor \mbox{\texttt{\mdseries\slshape divi}} is principal which in the toric invariant case means that it is the divisor of
a character. }

 

\subsection{\textcolor{Chapter }{IsPrimedivisor}}
\logpage{[ 8, 2, 3 ]}\nobreak
\hyperdef{L}{X844EFF227C868438}{}
{\noindent\textcolor{FuncColor}{$\triangleright$\ \ \texttt{IsPrimedivisor({\mdseries\slshape divi})\index{IsPrimedivisor@\texttt{IsPrimedivisor}}
\label{IsPrimedivisor}
}\hfill{\scriptsize (property)}}\\
\textbf{\indent Returns:\ }
\texttt{true} or \texttt{false}



 Checks if the Weil divisor \mbox{\texttt{\mdseries\slshape divi}} represents a prime divisor, i.e. if it is a standard generator of the divisor
group. }

 

\subsection{\textcolor{Chapter }{IsBasepointFree}}
\logpage{[ 8, 2, 4 ]}\nobreak
\hyperdef{L}{X7AB6B5C77DFB740B}{}
{\noindent\textcolor{FuncColor}{$\triangleright$\ \ \texttt{IsBasepointFree({\mdseries\slshape divi})\index{IsBasepointFree@\texttt{IsBasepointFree}}
\label{IsBasepointFree}
}\hfill{\scriptsize (property)}}\\
\textbf{\indent Returns:\ }
\texttt{true} or \texttt{false}



 Checks if the divisor \mbox{\texttt{\mdseries\slshape divi}} is basepoint free. What else? }

 

\subsection{\textcolor{Chapter }{IsAmple}}
\logpage{[ 8, 2, 5 ]}\nobreak
\hyperdef{L}{X7F5062AA844AEC55}{}
{\noindent\textcolor{FuncColor}{$\triangleright$\ \ \texttt{IsAmple({\mdseries\slshape divi})\index{IsAmple@\texttt{IsAmple}}
\label{IsAmple}
}\hfill{\scriptsize (property)}}\\
\textbf{\indent Returns:\ }
\texttt{true} or \texttt{false}



 Checks if the divisor \mbox{\texttt{\mdseries\slshape divi}} is ample, i.e. if it is colored red, yellow and green. }

 

\subsection{\textcolor{Chapter }{IsVeryAmple}}
\logpage{[ 8, 2, 6 ]}\nobreak
\hyperdef{L}{X80A58559802BB02E}{}
{\noindent\textcolor{FuncColor}{$\triangleright$\ \ \texttt{IsVeryAmple({\mdseries\slshape divi})\index{IsVeryAmple@\texttt{IsVeryAmple}}
\label{IsVeryAmple}
}\hfill{\scriptsize (property)}}\\
\textbf{\indent Returns:\ }
\texttt{true} or \texttt{false}



 Checks if the divisor \mbox{\texttt{\mdseries\slshape divi}} is very ample. }

 }

 
\section{\textcolor{Chapter }{Toric divisors: Attributes}}\label{Divisors:Attributes}
\logpage{[ 8, 3, 0 ]}
\hyperdef{L}{X853500FD794001B8}{}
{
  

\subsection{\textcolor{Chapter }{CartierData}}
\logpage{[ 8, 3, 1 ]}\nobreak
\hyperdef{L}{X7F546017860701EA}{}
{\noindent\textcolor{FuncColor}{$\triangleright$\ \ \texttt{CartierData({\mdseries\slshape divi})\index{CartierData@\texttt{CartierData}}
\label{CartierData}
}\hfill{\scriptsize (attribute)}}\\
\textbf{\indent Returns:\ }
a list



 Returns the Cartier data of the divisor \mbox{\texttt{\mdseries\slshape divi}}, if it is Cartier, and fails otherwise. }

 

\subsection{\textcolor{Chapter }{CharacterOfPrincipalDivisor}}
\logpage{[ 8, 3, 2 ]}\nobreak
\hyperdef{L}{X796415858436B32D}{}
{\noindent\textcolor{FuncColor}{$\triangleright$\ \ \texttt{CharacterOfPrincipalDivisor({\mdseries\slshape divi})\index{CharacterOfPrincipalDivisor@\texttt{CharacterOfPrincipalDivisor}}
\label{CharacterOfPrincipalDivisor}
}\hfill{\scriptsize (attribute)}}\\
\textbf{\indent Returns:\ }
an element



 Returns the character corresponding to principal divisor \mbox{\texttt{\mdseries\slshape divi}}. }

 

\subsection{\textcolor{Chapter }{ToricVarietyOfDivisor}}
\logpage{[ 8, 3, 3 ]}\nobreak
\hyperdef{L}{X80FA3ADA81640191}{}
{\noindent\textcolor{FuncColor}{$\triangleright$\ \ \texttt{ToricVarietyOfDivisor({\mdseries\slshape divi})\index{ToricVarietyOfDivisor@\texttt{ToricVarietyOfDivisor}}
\label{ToricVarietyOfDivisor}
}\hfill{\scriptsize (attribute)}}\\
\textbf{\indent Returns:\ }
a variety



 Returns the closure of the torus orbit corresponding to the prime divisor \mbox{\texttt{\mdseries\slshape divi}}. Not implemented for other divisors. Maybe we should add the support here. Is
this even a toric variety? Exercise left to the reader. }

 

\subsection{\textcolor{Chapter }{ClassOfDivisor}}
\logpage{[ 8, 3, 4 ]}\nobreak
\hyperdef{L}{X7C3691B4816CF3E9}{}
{\noindent\textcolor{FuncColor}{$\triangleright$\ \ \texttt{ClassOfDivisor({\mdseries\slshape divi})\index{ClassOfDivisor@\texttt{ClassOfDivisor}}
\label{ClassOfDivisor}
}\hfill{\scriptsize (attribute)}}\\
\textbf{\indent Returns:\ }
an element



 Returns the class group element corresponding to the divisor \mbox{\texttt{\mdseries\slshape divi}}. }

 

\subsection{\textcolor{Chapter }{PolytopeOfDivisor}}
\logpage{[ 8, 3, 5 ]}\nobreak
\hyperdef{L}{X85ED82DC79E2F1CE}{}
{\noindent\textcolor{FuncColor}{$\triangleright$\ \ \texttt{PolytopeOfDivisor({\mdseries\slshape divi})\index{PolytopeOfDivisor@\texttt{PolytopeOfDivisor}}
\label{PolytopeOfDivisor}
}\hfill{\scriptsize (attribute)}}\\
\textbf{\indent Returns:\ }
a polytope



 Returns the polytope corresponding to the divisor \mbox{\texttt{\mdseries\slshape divi}}. }

 

\subsection{\textcolor{Chapter }{BasisOfGlobalSections}}
\logpage{[ 8, 3, 6 ]}\nobreak
\hyperdef{L}{X7A7853288166B329}{}
{\noindent\textcolor{FuncColor}{$\triangleright$\ \ \texttt{BasisOfGlobalSections({\mdseries\slshape divi})\index{BasisOfGlobalSections@\texttt{BasisOfGlobalSections}}
\label{BasisOfGlobalSections}
}\hfill{\scriptsize (attribute)}}\\
\textbf{\indent Returns:\ }
a list



 Returns a basis of the global section module of the quasi-coherent sheaf of
the divisor \mbox{\texttt{\mdseries\slshape divi}}. }

 

\subsection{\textcolor{Chapter }{IntegerForWhichIsSureVeryAmple}}
\logpage{[ 8, 3, 7 ]}\nobreak
\hyperdef{L}{X87DA4EEA824F4175}{}
{\noindent\textcolor{FuncColor}{$\triangleright$\ \ \texttt{IntegerForWhichIsSureVeryAmple({\mdseries\slshape divi})\index{IntegerForWhichIsSureVeryAmple@\texttt{IntegerForWhichIsSureVeryAmple}}
\label{IntegerForWhichIsSureVeryAmple}
}\hfill{\scriptsize (attribute)}}\\
\textbf{\indent Returns:\ }
an integer



 Returns an integer which, to be multiplied with the ample divisor \mbox{\texttt{\mdseries\slshape divi}}, someone gets a very ample divisor. }

 

\subsection{\textcolor{Chapter }{AmbientToricVariety (for toric divisors)}}
\logpage{[ 8, 3, 8 ]}\nobreak
\hyperdef{L}{X809666867CB2FC87}{}
{\noindent\textcolor{FuncColor}{$\triangleright$\ \ \texttt{AmbientToricVariety({\mdseries\slshape divi})\index{AmbientToricVariety@\texttt{AmbientToricVariety}!for toric divisors}
\label{AmbientToricVariety:for toric divisors}
}\hfill{\scriptsize (attribute)}}\\
\textbf{\indent Returns:\ }
a variety



 Returns the containing variety of the prime divisors of the divisor \mbox{\texttt{\mdseries\slshape divi}}. }

 

\subsection{\textcolor{Chapter }{UnderlyingGroupElement}}
\logpage{[ 8, 3, 9 ]}\nobreak
\hyperdef{L}{X7D42C40B7B4B06E0}{}
{\noindent\textcolor{FuncColor}{$\triangleright$\ \ \texttt{UnderlyingGroupElement({\mdseries\slshape divi})\index{UnderlyingGroupElement@\texttt{UnderlyingGroupElement}}
\label{UnderlyingGroupElement}
}\hfill{\scriptsize (attribute)}}\\
\textbf{\indent Returns:\ }
an element



 Returns an element which represents the divisor \mbox{\texttt{\mdseries\slshape divi}} in the Weil group. }

 

\subsection{\textcolor{Chapter }{UnderlyingToricVariety (for prime divisors)}}
\logpage{[ 8, 3, 10 ]}\nobreak
\hyperdef{L}{X81DF09097AEA7A0D}{}
{\noindent\textcolor{FuncColor}{$\triangleright$\ \ \texttt{UnderlyingToricVariety({\mdseries\slshape divi})\index{UnderlyingToricVariety@\texttt{UnderlyingToricVariety}!for prime divisors}
\label{UnderlyingToricVariety:for prime divisors}
}\hfill{\scriptsize (attribute)}}\\
\textbf{\indent Returns:\ }
a variety



 Returns the closure of the torus orbit corresponding to the prime divisor \mbox{\texttt{\mdseries\slshape divi}}. Not implemented for other divisors. Maybe we should add the support here. Is
this even a toric variety? Exercise left to the reader. }

 

\subsection{\textcolor{Chapter }{DegreeOfDivisor}}
\logpage{[ 8, 3, 11 ]}\nobreak
\hyperdef{L}{X7A346588871296FB}{}
{\noindent\textcolor{FuncColor}{$\triangleright$\ \ \texttt{DegreeOfDivisor({\mdseries\slshape divi})\index{DegreeOfDivisor@\texttt{DegreeOfDivisor}}
\label{DegreeOfDivisor}
}\hfill{\scriptsize (attribute)}}\\
\textbf{\indent Returns:\ }
an integer



 Returns the degree of the divisor \mbox{\texttt{\mdseries\slshape divi}}. }

 

\subsection{\textcolor{Chapter }{MonomsOfCoxRingOfDegree}}
\logpage{[ 8, 3, 12 ]}\nobreak
\hyperdef{L}{X838F54D8817F6066}{}
{\noindent\textcolor{FuncColor}{$\triangleright$\ \ \texttt{MonomsOfCoxRingOfDegree({\mdseries\slshape divi})\index{MonomsOfCoxRingOfDegree@\texttt{MonomsOfCoxRingOfDegree}}
\label{MonomsOfCoxRingOfDegree}
}\hfill{\scriptsize (attribute)}}\\
\textbf{\indent Returns:\ }
a list



 Returns the variety corresponding to the polytope of the divisor \mbox{\texttt{\mdseries\slshape divi}}. }

 

\subsection{\textcolor{Chapter }{CoxRingOfTargetOfDivisorMorphism}}
\logpage{[ 8, 3, 13 ]}\nobreak
\hyperdef{L}{X831C20A585952B08}{}
{\noindent\textcolor{FuncColor}{$\triangleright$\ \ \texttt{CoxRingOfTargetOfDivisorMorphism({\mdseries\slshape divi})\index{CoxRingOfTargetOfDivisorMorphism@\texttt{CoxRingOfTargetOfDivisorMorphism}}
\label{CoxRingOfTargetOfDivisorMorphism}
}\hfill{\scriptsize (attribute)}}\\
\textbf{\indent Returns:\ }
a ring



 A basepoint free divisor \mbox{\texttt{\mdseries\slshape divi}} defines a map from its ambient variety in a projective space. This method
returns the cox ring of such a projective space. }

 

\subsection{\textcolor{Chapter }{RingMorphismOfDivisor}}
\logpage{[ 8, 3, 14 ]}\nobreak
\hyperdef{L}{X786F05507F8026D5}{}
{\noindent\textcolor{FuncColor}{$\triangleright$\ \ \texttt{RingMorphismOfDivisor({\mdseries\slshape divi})\index{RingMorphismOfDivisor@\texttt{RingMorphismOfDivisor}}
\label{RingMorphismOfDivisor}
}\hfill{\scriptsize (attribute)}}\\
\textbf{\indent Returns:\ }
a ring



 A basepoint free divisor \mbox{\texttt{\mdseries\slshape divi}} defines a map from its ambient variety in a projective space. This method
returns the morphism between the cox ring of this projective space to the cox
ring of the ambient variety of \mbox{\texttt{\mdseries\slshape divi}}. }

 }

 
\section{\textcolor{Chapter }{Toric divisors: Methods}}\label{divisors:Methods}
\logpage{[ 8, 4, 0 ]}
\hyperdef{L}{X868C3EF185DDF025}{}
{
  

\subsection{\textcolor{Chapter }{VeryAmpleMultiple}}
\logpage{[ 8, 4, 1 ]}\nobreak
\hyperdef{L}{X79410AC7794A3EE7}{}
{\noindent\textcolor{FuncColor}{$\triangleright$\ \ \texttt{VeryAmpleMultiple({\mdseries\slshape divi})\index{VeryAmpleMultiple@\texttt{VeryAmpleMultiple}}
\label{VeryAmpleMultiple}
}\hfill{\scriptsize (operation)}}\\
\textbf{\indent Returns:\ }
a divisor



 Returns a very ample multiple of the ample divisor \mbox{\texttt{\mdseries\slshape divi}}. Will fail if divisor is not ample. }

 

\subsection{\textcolor{Chapter }{CharactersForClosedEmbedding}}
\logpage{[ 8, 4, 2 ]}\nobreak
\hyperdef{L}{X853616578245BF1A}{}
{\noindent\textcolor{FuncColor}{$\triangleright$\ \ \texttt{CharactersForClosedEmbedding({\mdseries\slshape divi})\index{CharactersForClosedEmbedding@\texttt{CharactersForClosedEmbedding}}
\label{CharactersForClosedEmbedding}
}\hfill{\scriptsize (operation)}}\\
\textbf{\indent Returns:\ }
a list



 Returns characters for closed embedding defined via the ample divisor \mbox{\texttt{\mdseries\slshape divi}}. Fails if divisor is not ample. }

 

\subsection{\textcolor{Chapter }{MonomsOfCoxRingOfDegree (for an homalg element)}}
\logpage{[ 8, 4, 3 ]}\nobreak
\hyperdef{L}{X7D890CF4845178FE}{}
{\noindent\textcolor{FuncColor}{$\triangleright$\ \ \texttt{MonomsOfCoxRingOfDegree({\mdseries\slshape vari, elem})\index{MonomsOfCoxRingOfDegree@\texttt{MonomsOfCoxRingOfDegree}!for an homalg element}
\label{MonomsOfCoxRingOfDegree:for an homalg element}
}\hfill{\scriptsize (operation)}}\\
\textbf{\indent Returns:\ }
a list



 Returns the monoms of the Cox ring of the variety \mbox{\texttt{\mdseries\slshape vari}} with degree to the class group element \mbox{\texttt{\mdseries\slshape elem}}. The variable \mbox{\texttt{\mdseries\slshape elem}} can also be a list. }

 

\subsection{\textcolor{Chapter }{DivisorOfGivenClass}}
\logpage{[ 8, 4, 4 ]}\nobreak
\hyperdef{L}{X7E66CB878743D6DF}{}
{\noindent\textcolor{FuncColor}{$\triangleright$\ \ \texttt{DivisorOfGivenClass({\mdseries\slshape vari, elem})\index{DivisorOfGivenClass@\texttt{DivisorOfGivenClass}}
\label{DivisorOfGivenClass}
}\hfill{\scriptsize (operation)}}\\
\textbf{\indent Returns:\ }
a list



 Computes a divisor of the variety \mbox{\texttt{\mdseries\slshape divi}} which is member of the divisor class presented by \mbox{\texttt{\mdseries\slshape elem}}. The variable \mbox{\texttt{\mdseries\slshape elem}} can be a homalg element or a list presenting an element. }

 

\subsection{\textcolor{Chapter }{AddDivisorToItsAmbientVariety}}
\logpage{[ 8, 4, 5 ]}\nobreak
\hyperdef{L}{X7A19D0127ECE5EE1}{}
{\noindent\textcolor{FuncColor}{$\triangleright$\ \ \texttt{AddDivisorToItsAmbientVariety({\mdseries\slshape divi})\index{AddDivisorToItsAmbientVariety@\texttt{AddDivisorToItsAmbientVariety}}
\label{AddDivisorToItsAmbientVariety}
}\hfill{\scriptsize (operation)}}\\
\textbf{\indent Returns:\ }




 Adds the divisor \mbox{\texttt{\mdseries\slshape divi}} to the Weil divisor list of its ambient variety. }

 

\subsection{\textcolor{Chapter }{Polytope (for toric divisors)}}
\logpage{[ 8, 4, 6 ]}\nobreak
\hyperdef{L}{X7BA1C5A27CC75C38}{}
{\noindent\textcolor{FuncColor}{$\triangleright$\ \ \texttt{Polytope({\mdseries\slshape divi})\index{Polytope@\texttt{Polytope}!for toric divisors}
\label{Polytope:for toric divisors}
}\hfill{\scriptsize (operation)}}\\
\textbf{\indent Returns:\ }
a polytope



 Returns the polytope of the divisor \mbox{\texttt{\mdseries\slshape divi}}. Another name for PolytopeOfDivisor for compatibility and shortness. }

 

\subsection{\textcolor{Chapter }{+}}
\logpage{[ 8, 4, 7 ]}\nobreak
\hyperdef{L}{X7F2703417F270341}{}
{\noindent\textcolor{FuncColor}{$\triangleright$\ \ \texttt{+({\mdseries\slshape divi1, divi2})\index{+@\texttt{+}}
\label{+}
}\hfill{\scriptsize (operation)}}\\
\textbf{\indent Returns:\ }
a divisor



 Returns the sum of the divisors \mbox{\texttt{\mdseries\slshape divi1}} and \mbox{\texttt{\mdseries\slshape divi2}}. }

 

\subsection{\textcolor{Chapter }{-}}
\logpage{[ 8, 4, 8 ]}\nobreak
\hyperdef{L}{X81B1391281B13912}{}
{\noindent\textcolor{FuncColor}{$\triangleright$\ \ \texttt{-({\mdseries\slshape divi1, divi2})\index{-@\texttt{-}}
\label{-}
}\hfill{\scriptsize (operation)}}\\
\textbf{\indent Returns:\ }
a divisor



 Returns the divisor \mbox{\texttt{\mdseries\slshape divi1}} minus \mbox{\texttt{\mdseries\slshape divi2}}. }

 

\subsection{\textcolor{Chapter }{* (for toric divisors)}}
\logpage{[ 8, 4, 9 ]}\nobreak
\hyperdef{L}{X7A14A08D79AABCC5}{}
{\noindent\textcolor{FuncColor}{$\triangleright$\ \ \texttt{*({\mdseries\slshape k, divi})\index{*@\texttt{*}!for toric divisors}
\label{*:for toric divisors}
}\hfill{\scriptsize (operation)}}\\
\textbf{\indent Returns:\ }
a divisor



 Returns \mbox{\texttt{\mdseries\slshape k}} times the divisor \mbox{\texttt{\mdseries\slshape divi}}. }

 }

 
\section{\textcolor{Chapter }{Toric divisors: Constructors}}\label{Divisors:Constructors}
\logpage{[ 8, 5, 0 ]}
\hyperdef{L}{X863E167D799CBE31}{}
{
  

\subsection{\textcolor{Chapter }{DivisorOfCharacter}}
\logpage{[ 8, 5, 1 ]}\nobreak
\hyperdef{L}{X7A1062647B359F8A}{}
{\noindent\textcolor{FuncColor}{$\triangleright$\ \ \texttt{DivisorOfCharacter({\mdseries\slshape elem, vari})\index{DivisorOfCharacter@\texttt{DivisorOfCharacter}}
\label{DivisorOfCharacter}
}\hfill{\scriptsize (operation)}}\\
\textbf{\indent Returns:\ }
a divisor



 Returns the divisor of the toric variety \mbox{\texttt{\mdseries\slshape vari}} which corresponds to the character \mbox{\texttt{\mdseries\slshape elem}}. }

 

\subsection{\textcolor{Chapter }{DivisorOfCharacter (for a list of integers)}}
\logpage{[ 8, 5, 2 ]}\nobreak
\hyperdef{L}{X7C2AF9FC7AEA18BE}{}
{\noindent\textcolor{FuncColor}{$\triangleright$\ \ \texttt{DivisorOfCharacter({\mdseries\slshape lis, vari})\index{DivisorOfCharacter@\texttt{DivisorOfCharacter}!for a list of integers}
\label{DivisorOfCharacter:for a list of integers}
}\hfill{\scriptsize (operation)}}\\
\textbf{\indent Returns:\ }
a divisor



 Returns the divisor of the toric variety \mbox{\texttt{\mdseries\slshape vari}} which corresponds to the character which is created by the list \mbox{\texttt{\mdseries\slshape lis}}. }

 

\subsection{\textcolor{Chapter }{CreateDivisor (for a homalg element)}}
\logpage{[ 8, 5, 3 ]}\nobreak
\hyperdef{L}{X7F24DB367B7BAAB4}{}
{\noindent\textcolor{FuncColor}{$\triangleright$\ \ \texttt{CreateDivisor({\mdseries\slshape elem, vari})\index{CreateDivisor@\texttt{CreateDivisor}!for a homalg element}
\label{CreateDivisor:for a homalg element}
}\hfill{\scriptsize (operation)}}\\
\textbf{\indent Returns:\ }
a divisor



 Returns the divisor of the toric variety \mbox{\texttt{\mdseries\slshape vari}} which corresponds to the Weil group element \mbox{\texttt{\mdseries\slshape elem}}. }

 

\subsection{\textcolor{Chapter }{CreateDivisor (for a list of integers)}}
\logpage{[ 8, 5, 4 ]}\nobreak
\hyperdef{L}{X87477D82832DDE3A}{}
{\noindent\textcolor{FuncColor}{$\triangleright$\ \ \texttt{CreateDivisor({\mdseries\slshape lis, vari})\index{CreateDivisor@\texttt{CreateDivisor}!for a list of integers}
\label{CreateDivisor:for a list of integers}
}\hfill{\scriptsize (operation)}}\\
\textbf{\indent Returns:\ }
a divisor



 Returns the divisor of the toric variety \mbox{\texttt{\mdseries\slshape vari}} which corresponds to the Weil group element which is created by the list \mbox{\texttt{\mdseries\slshape lis}}. }

 }

 
\section{\textcolor{Chapter }{Toric divisors: Examples}}\label{Divisors:Examples}
\logpage{[ 8, 6, 0 ]}
\hyperdef{L}{X7FE84D6F87076DA0}{}
{
  
\subsection{\textcolor{Chapter }{Divisors on a toric variety}}\label{DivisorsExampleSubsection}
\logpage{[ 8, 6, 1 ]}
\hyperdef{L}{X7A7080E77E93A36F}{}
{
  
\begin{Verbatim}[commandchars=!@J,fontsize=\small,frame=single,label=Example]
  !gapprompt@gap>J !gapinput@H7 := Fan( [[0,1],[1,0],[0,-1],[-1,7]],[[1,2],[2,3],[3,4],[4,1]] );J
  <A fan in |R^2>
  !gapprompt@gap>J !gapinput@H7 := ToricVariety( H7 );J
  <A toric variety of dimension 2>
  !gapprompt@gap>J !gapinput@P := TorusInvariantPrimeDivisors( H7 );J
  [ <A prime divisor of a toric variety with coordinates [ 1, 0, 0, 0 ]>, 
    <A prime divisor of a toric variety with coordinates [ 0, 1, 0, 0 ]>, 
    <A prime divisor of a toric variety with coordinates [ 0, 0, 1, 0 ]>, 
    <A prime divisor of a toric variety with coordinates [ 0, 0, 0, 1 ]> ]
  !gapprompt@gap>J !gapinput@D := P[3]+P[4];J
  <A divisor of a toric variety with coordinates [ 0, 0, 1, 1 ]>
  !gapprompt@gap>J !gapinput@IsBasepointFree(D);J
  true
  !gapprompt@gap>J !gapinput@IsAmple(D);J
  true
  !gapprompt@gap>J !gapinput@CoordinateRingOfTorus(H7,"x");J
  Q[x1,x1_,x2,x2_]/( x2*x2_-1, x1*x1_-1 )
  !gapprompt@gap>J !gapinput@Polytope(D);J
  <A polytope in |R^2>
  !gapprompt@gap>J !gapinput@CharactersForClosedEmbedding(D);J
  [ |[ 1 ]|, |[ x2 ]|, |[ x1 ]|, |[ x1*x2 ]|, |[ x1^2*x2 ]|, 
    |[ x1^3*x2 ]|, |[ x1^4*x2 ]|, |[ x1^5*x2 ]|, 
    |[ x1^6*x2 ]|, |[ x1^7*x2 ]|, |[ x1^8*x2 ]| ]
  !gapprompt@gap>J !gapinput@CoxRingOfTargetOfDivisorMorphism(D);J
  Q[x_1,x_2,x_3,x_4,x_5,x_6,x_7,x_8,x_9,x_10,x_11]
  (weights: [ 1, 1, 1, 1, 1, 1, 1, 1, 1, 1, 1 ])
  !gapprompt@gap>J !gapinput@RingMorphismOfDivisor(D);J
  <A "homomorphism" of rings>
  !gapprompt@gap>J !gapinput@Display(last);J
  Q[x_1,x_2,x_3,x_4]
  (weights: [ [ 0, 0, 1, -7 ], [ 0, 0, 0, 1 ], [ 0, 0, 1, 0 ], [ 0, 0, 0, 1 ] ])
    ^
    |
  [ x_3*x_4, x_1*x_4^8, x_2*x_3, x_1*x_2*x_4^7, x_1*x_2^2*x_4^6,
    x_1*x_2^3*x_4^5, x_1*x_2^4*x_4^4, x_1*x_2^5*x_4^3, 
    x_1*x_2^6*x_4^2, x_1*x_2^7*x_4, x_1*x_2^8 ]
    |
    |
  Q[x_1,x_2,x_3,x_4,x_5,x_6,x_7,x_8,x_9,x_10,x_11]
  (weights: [ 1, 1, 1, 1, 1, 1, 1, 1, 1, 1, 1 ])
  !gapprompt@gap>J !gapinput@ByASmallerPresentation(ClassGroup(H7));J
  <A free left module of rank 2 on free generators>
  !gapprompt@gap>J !gapinput@Display(RingMorphismOfDivisor(D));J
  Q[x_1,x_2,x_3,x_4]
  (weights: [ [ 1, -7 ], [ 0, 1 ], [ 1, 0 ], [ 0, 1 ] ])
    ^
    |
  [ x_3*x_4, x_1*x_4^8, x_2*x_3, x_1*x_2*x_4^7, x_1*x_2^2*x_4^6, 
    x_1*x_2^3*x_4^5, x_1*x_2^4*x_4^4, x_1*x_2^5*x_4^3, 
    x_1*x_2^6*x_4^2, x_1*x_2^7*x_4, x_1*x_2^8 ]
    |
    |
  Q[x_1,x_2,x_3,x_4,x_5,x_6,x_7,x_8,x_9,x_10,x_11]
  (weights: [ 1, 1, 1, 1, 1, 1, 1, 1, 1, 1, 1 ])
  !gapprompt@gap>J !gapinput@MonomsOfCoxRingOfDegree(D);J
  [ x_3*x_4, x_1*x_4^8, x_2*x_3, x_1*x_2*x_4^7, x_1*x_2^2*x_4^6, 
    x_1*x_2^3*x_4^5, x_1*x_2^4*x_4^4, x_1*x_2^5*x_4^3, 
    x_1*x_2^6*x_4^2, x_1*x_2^7*x_4, x_1*x_2^8 ]
  !gapprompt@gap>J !gapinput@D2:=D-2*P[2];J
  <A divisor of a toric variety with coordinates [ 0, -2, 1, 1 ]>
  !gapprompt@gap>J !gapinput@IsBasepointFree(D2);J
  false
  !gapprompt@gap>J !gapinput@IsAmple(D2);J
  false
\end{Verbatim}
}

 }

  }

 \def\indexname{Index\logpage{[ "Ind", 0, 0 ]}
\hyperdef{L}{X83A0356F839C696F}{}
}

\cleardoublepage
\phantomsection
\addcontentsline{toc}{chapter}{Index}


\printindex

\newpage
\immediate\write\pagenrlog{["End"], \arabic{page}];}
\immediate\closeout\pagenrlog
\end{document}
