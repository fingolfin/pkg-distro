%%%%%%%%%%%%%%%%%%%%%%%%%%%%%%%%%%%%%%%%%%%%%%%%%%%%%%%%%%%%%%%%%%%%%%%%%
%%
%W  forms.tex             FORMAT documentation    B. Eick and C.R.B. Wright
%%
%%  10-31-11

%%%%%%%%%%%%%%%%%%%%%%%%%%%%%%%%%%%%%%%%%%%%%%%%%%%%%%%%%%%%%%%%%%%%%%%%%
\Chapter{Formations in GAP}

%\index{Formations in GAP}

A *formation* is a class ${\F}$ of groups closed under taking epimorphic
images and subdirect products. Closure under subdirect products is equivalent to the property that each finite group $G$  has a unique 
smallest normal subgroup $G^{\F}$ with factor group $G / G^{\F}$ in~$\F$. 
The subgroup $G^{\F}$ is called the *$\F$-residual* subgroup of $G$. Thus, 
for example, the derived subgroup of $G$ is its residual for the formation 
of abelian groups, and the residual for the formation of nilpotent groups 
is the last term of the descending central series. 

In {\FORMAT} a formation is described by a function that computes $G^{\F}$ 
for each (finite solvable) group $G$, and from that perspective $\F$ 
consists of the groups $G$ for which $G^{\F}$ is trivial. To define a 
formation that is not one of the standard examples provided (see below), 
one must give {\GAP} an identifier for the formation and also some method 
for computing residual subgroups. 

Some of the most interesting formations can also be described by ``local definition.'' For each prime $p$ 
let $\F(p)$ be a formation or the empty class, and let $\F$ be the class 
of all finite solvable groups $G$ such that for each prime $p$ and each $p$-chief factor $H/K$ of $G$ the group of automorphisms that $G$ 
induces on $H/K$ by conjugation belongs to $\F(p)$. Then $\F$ is a 
formation,  with  *local definition*  the set of $ \F(p)$s. 
The set of primes $p$ for which $\F(p)$ is not empty is called the 
*support* of~$\F$. A $p$-chief factor is *$\F$-central* in case $G$
induces an $\F(p)$-group on it or, equivalently, in case $G^{\F(p)}$
centralizes it. It is possible to define a formation in {\FORMAT} by
giving such a local definition. Indeed one can define a kind of
generalized formation by giving what is called a normal subgroup function 
or *screen*, which specifies arbitrary normal subgroups, not necessarily 
of form $G^{\F(p)}$, to test ``centrality.'' Section~"Other Applications" describes one such usage of general screens. Most applications of formation 
theory to solvable groups require local definition, as do the {\GAP} 
functions for computing $\F$-normalizers and $\F$-covering subgroups. 

\> Formation( <rec> ) O
\> Formation( <str> [, <primes> ] ) O

The definition of a formation in {\FORMAT} begins with the creation of a
record `rec', which must contain a `name' component and at least one of
the components `fResidual' or `fScreen'. The component `name' is a string, 
`fResidual' is a function that computes a normal subgroup of each group,
and `fScreen' is a function of two variables, a group and a prime, that
returns a normal subgroup of the input group.

In the second form the function `Formation' can be used to obtain a
formation from the supplied library of formations. The formations
provided are:

\beginitems
`Formation( "Nilpotent" )' & The formation of nilpotent groups,  

`Formation( "Supersolvable" )' & The formation of supersolvable groups,
 

`Formation( "Abelian" )' & The formation of abelian groups,
 

`Formation( "ElementaryAbelianProduct" )' & The formation of direct products of elementary abelian groups,
 

`Formation( "PNilpotent", prime )' & The formation of $p$-nilpotent groups 
for $p =$ `prime',


`Formation( "PiGroups", primes )' & The formation of $\pi$-groups for 
$\pi =$ the set `primes',


`Formation( "PLengthOne", prime )' & The formation of groups of $p$-length 1 
for $p =$ `prime'.
\enditems

\> IsFormation( <F> ) C
\> NameOfFormation( <F> ) A
\> ResidualFunctionOfFormation( <F> ) A

`IsFormation' returns `true' if and only if <F> is a {\GAP} formation. 
`NameOfFormation' returns the name of a formation and `ResidualFunctionOfFormation'
returns the  residual function of a formation. 

\> ScreenOfFormation( <F> ) A

If <F> is locally defined by some screen of $\F(p)$s, 
then `HasScreenOfFormation( <F> )' is `true', `ScreenOfFormation( <F> )' is a function of two variables, <group> and <prime>, and 
`ScreenOfFormation( <F> )(  <G>, <p> )' returns $G^{F(p)}$ if <p> is
in the support of <F> and  gives the empty list otherwise. 

\> SupportOfFormation( <F> ) A

The attribute `SupportOfFormation' is optional. It may be bound by
`SetSupportOfFormation'. If `SupportOfFormation' is not bound, then the support
of the formation is taken to be the set of all primes. In case the support of
<F> is a finite set of primes, then `SupportOfFormation( <F> )' is a list of
those primes, and `HasSupportOfFormation( <F> )' returns true. In case the
support of <F> is an infinite set but not the set of all primes, then the user
will need to make sure, perhaps with `ChangedSupport' or
`SetSupportOfFormation', that all primes dividing the orders of relevant groups
are considered.

\> ChangedSupport( <F>, <primes> ) O

This function may be used to change the support of a formation. Let <F>
be a formation and <primes> a list of primes. Then `ChangedSupport'
returns a formation with a new name whose support is the intersection 
of the support of <F> and <primes>.

\> IsIntegrated( <F> ) P

The local definition  is called *integrated* in case $\F(p)$ is contained in
$\F$  for each  prime~$p$. The optional property `IsIntegrated' makes sense only if `HasScreenOfFormation( <F> )' is `true'. Notice that
some of the functions described below will require that all of the attributes `HasScreenOfFormation(
<F> )', `HasIsIntegrated( <F> )' and `IsIntegrated( <F> )' are `true'. If
unbound, this property can be bound with `SetIsIntegrated', but it is up to the
user to determine whether such a setting is appropriate.
Section "Formation Examples" contains an example of such usage.

\> Integrated( <F> )  O

A local definition of a formation may always be replaced by an
integrated one without changing the formation itself, though the meaning
of ${\F}$-central may change.  Let <F> be a locally defined {\GAP} formation with
name `<name>'.  If <F> is already integrated, then `Integrated( <F> )' 
yields <F> itself. Otherwise, it yields a  formation `<name>Int' that is
abstractly the same as <F> but has integrated local definition.

\> `<F1> = <F2>' {formation!equality}
\> `<F1> \< <F2>' {formation!comparison}

Two {\GAP} formations <F1> and <F2> are considered to be equal in case they 
have the same name. The natural ordering on strings gives an ordering 
on formations. This ordering is useful for organizing key-dependent 
lists but has no mathematical significance.

\> Intersection( <F1>, <F2> ) O

The intersection of two {\GAP} formations <F1> and 
<F2> is again a formation. `Intersection' produces the new formation
`(<name1>And<name2>)', which has attribute `ResidualFunctionOfFormation' if
either <F1> or <F2> does, has `FScreen' whenever both formations have `FScreen', and is
integrated if both are. 

\> ProductOfFormations( <F1>, <F2> ) O

The product of two  formations <F1> and <F2> is the formation <F>
such that a finite group $G$ is a member of <F> if and only if
$G^{F2}$ is in <F1>. (Notice that the product of <F1> by <F2> is 
not necessarily equal to the product of <F2> by <F1>, and unless <F1> is normal subgroup-closed the product need not contain all extensions of a group in <F1> by a group in <F2>.) The function 
`ProductOfFormations( <F1>, <F2> )' yields the product `(<name1>By<name2>)' of the two  
formations. The product has the attribute `ResidualFunctionOfFormation' and has
the attribute  `ScreenOfFormation' whenever both <F1> and <F2> have this entry
or whenever both `HasScreenOfFormation( <F2> )' and `not HasSupportOfFormation(
<F1> )' are `true'. In these cases the property `IsIntegrated' will be inherited
if possible. 
