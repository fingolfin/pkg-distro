\newcommand{\QPAIntroPartNumber}{2}
\documentclass[usenames,dvipsnames]{beamer}

%\usetheme{ntnu}
\usetheme{Warsaw}

\usepackage{times}
\usepackage[T1]{fontenc}
\usepackage[all]{xy}
\usepackage{textpos}
\usepackage{tikz}

\newcommand{\defn}[1]{\textit{#1}}
\newcommand{\Q}{\mathbb{Q}}
\newcommand{\equivalence}{\simeq}
\newcommand{\VV}[2]{\begin{pmatrix} #1 & #2 \end{pmatrix}}
\newcommand{\vv}[2]{\left( \begin{smallmatrix} #1 & #2 \end{smallmatrix} \right)}
\newcommand{\into}{\hookrightarrow}
\newcommand{\iso}{\cong}
\newcommand{\dsum}{\oplus}
\DeclareMathOperator{\fmod}{mod}
\DeclareMathOperator{\Rep}{Rep}
\DeclareMathOperator{\Hom}{Hom}
\DeclareMathOperator{\coker}{coker}
\DeclareMathOperator{\im}{im}
\DeclareMathOperator{\rad}{rad}
\DeclareMathOperator{\soc}{soc}
\DeclareMathOperator{\Top}{top}

\title[Introduction to QPA, part \QPAIntroPartNumber]
      {Introduction to QPA}
\subtitle{Part \QPAIntroPartNumber}

\author{\O{}ystein~Skarts\ae{}terhagen \and \O{}yvind~Solberg}
\institute{
Department of Mathematical Sciences\\
Norwegian University of Science and Technology}

\date{Third GAP Days}

% If you have a file called "university-logo-filename.xxx", where xxx
% is a graphic format that can be processed by latex or pdflatex,
% resp., then you can add a logo as follows:

% \pgfdeclareimage[height=0.5cm]{university-logo}{university-logo-filename}
% \logo{\pgfuseimage{university-logo}}



% Delete this, if you do not want the table of contents to pop up at
% the beginning of each subsection:
% \AtBeginSubsection[]
% {
%   \begin{frame}<beamer>{Outline}
%     \tableofcontents[currentsection,currentsubsection]
%   \end{frame}
% }


\begin{document}

\begin{frame}
  \titlepage
\end{frame}

\begin{frame}{Outline}
  \tableofcontents
\end{frame}

\section{Basic functions}

\subsection{Special algebras}

\begin{frame}{Nakayama algebras}
\[
\vcenter{\xymatrix{
1 \ar[r]^{a_1} &
2 \ar[r]^{a_2} &
\cdots
  \ar[r]^{a_{n-1}} &
n
}}
\qquad\text{or}\qquad
\vcenter{\xymatrix@R=.8em@C=1.2em{
& 
1 \ar[r]^{a_1} &
2 \ar[dr]^{a_2} \\
n \ar[ur]^{a_n} &&&
3 \ar[dl]^{a_3} \\
&
\ar[ul]^{a^{n-1}} &
\ar@{.}[l]
}}
\]
\end{frame}

\begin{frame}[fragile]{A Nakayama algebra}
\[
A = kQ/\langle\rho\rangle
\qquad
Q\colon
\xymatrix{
1 \ar[r]^{a} &
2 \ar[r]^{b} &
3 \ar[r]^{c} &
4}
\qquad
\rho = \{ ab \}
\]
\onslide<2->
Indecomposable projective $A$-modules:
\begin{align*}
P1&\colon
\xymatrix{
k \ar[r]^1 &
k \ar[r] &
0 \ar[r] &
0}
&&\text{\onslide<3->{(length 2)}}
\\
P2&\colon
\xymatrix{
0 \ar[r] &
k \ar[r]^1 &
k \ar[r]^1 &
k}
&&\text{\onslide<3->{(length 3)}}
\\
P3&\colon
\xymatrix{
0 \ar[r] &
0 \ar[r] &
k \ar[r]^1 &
k}
&&\text{\onslide<3->{(length 2)}}
\\
P4&\colon
\xymatrix{
0 \ar[r] &
0 \ar[r] &
0 \ar[r] &
k}
&&\text{\onslide<3->{(length 1)}}
\end{align*}
\onslide<4->
\emph{Admissible sequence}: $(2,3,2,1)$
\onslide<5->
\begin{verbatim}
gap> NakayamaAlgebra([2,3,2,1], Rationals);
\end{verbatim}
\end{frame}

\begin{frame}[fragile]{Truncated path algebras}
\begin{itemize}
\item $kQ/I$, where $I$ generated by all paths of length $n$
\end{itemize}
\begin{verbatim}
gap> Q := Quiver(3, [[1,2,"a"],
                     [2,1,"b"],
                     [2,2,"c"]]);;
gap> A := TruncatedPathAlgebra(Rationals, Q, 3);
\end{verbatim}
\end{frame}


\subsection{Modules}

\begin{frame}[fragile]{Recall: Modules (representations) in QPA}
\[
Q \colon
\xymatrix{1 \ar[r]^a & 2 \ar[r]^b & 3}
\]
\[
M \colon
\xymatrix{
k   \ar[r]^{\left( \begin{smallmatrix} 2 & 0 \end{smallmatrix} \right)} &
k^2 \ar[r]^{\left( \begin{smallmatrix} 4 \\ -1 \end{smallmatrix} \right)} &
k
}
\]
\begin{verbatim}
gap> Q := Quiver(3, [[1,2,"a"],[2,3,"b"]]);;
gap> kQ := PathAlgebra(Rationals, Q);;
gap> M := RightModuleOverPathAlgebra
          (kQ, [1,2,1],
           [["a", [[2,0]]], ["b", [[4],[-1]]]]);
<[ 1, 2, 1 ]>
\end{verbatim}
\end{frame}

\begin{frame}{Module attributes}
\begin{overprint}
\onslide<1>
\[
M \colon
\xymatrix{
{\textcolor{OliveGreen}{k}^{\textcolor{blue}{1}}}
\ar[r]^{\textcolor{BrickRed}
        {\left( \begin{smallmatrix} 2 & 0 \end{smallmatrix}
         \right)}} &
{\textcolor{OliveGreen}{k}^{\textcolor{blue}{2}}}
\ar[r]^{\textcolor{BrickRed}
        {\left( \begin{smallmatrix} 4 \\ -1 \end{smallmatrix}
         \right)}} &
{\textcolor{OliveGreen}{k}^{\textcolor{blue}{1}}}
}
\]
\begin{itemize}
\item \texttt{RightActingAlgebra}: $kQ$
\item \texttt{LeftActingDomain}: \textcolor{OliveGreen}{$k$}
\item \texttt{DimensionVector}: 
      $(\textcolor{blue}{1}, \textcolor{blue}{2},
        \textcolor{blue}{1})$
\item \texttt{MatricesOfPathAlgebraModule}:
      $\left(
        \textcolor{BrickRed}{
          \begin{pmatrix} 2 & 0 \end{pmatrix}},
        \textcolor{BrickRed}{
          \begin{pmatrix} 4 \\ -1 \end{pmatrix}}
      \right)$
\item \texttt{Dimension}:
      $4 = \textcolor{blue}{1}
           + \textcolor{blue}{2}
           + \textcolor{blue}{1}$
\end{itemize}
\onslide<2>
\[
M \colon
\xymatrix{
k^1
\ar[r]^{\left( \begin{smallmatrix} 2 & 0 \end{smallmatrix}
        \right)} &
k^2
\ar[r]^{\left( \begin{smallmatrix} 4 \\ -1 \end{smallmatrix}
        \right)} &
k^2
}
\]
\begin{itemize}
\item \texttt{Basis}:
\\ \vspace{-3em}
\begin{gather*}
1 \to (0,0) \to 0 \\
0 \to (1,0) \to 0 \\
0 \to (0,1) \to 0 \\
0 \to (0,0) \to 1
\end{gather*}
\item \texttt{MinimalGeneratingSetOfModule}:
\begin{gather*}
1 \to (0,0) \to 0 \\
0 \to (0,0) \to 1
\end{gather*}
\end{itemize}
\end{overprint}
\end{frame}

\begin{frame}[fragile]{Submodules}{}
\begin{overprint}
\onslide<1>
\relax{\huge
\[
N \stackrel{i}{\into} M
\]}
\begin{itemize}
\item Categorical view of submodules
\item A submodule is given by an inclusion homomorphism
\item A submodule is not a subset
\end{itemize}
\onslide<2>
\relax{\huge
\[
\textcolor{BrickRed}{N} \stackrel{\textcolor{blue}{i}}{\into} M
\]}
\begin{itemize}
\item Categorical view of submodules
\item A submodule is given by an inclusion homomorphism
\item A submodule is not a subset
\item \texttt{SubRepresentation}: \textcolor{BrickRed}{$N$}
\item \texttt{SubRepresentationInclusion}: \textcolor{blue}{$i$}
\end{itemize}
% \onslide<3>
% \relax{\huge
% \[
% N \stackrel{i}{\into} M
% \]}
% \begin{verbatim}
% gap> kQ := PathAlgebra(Rationals, Quiver(2, [[1,2,"a"],[1,2,"b"]]));;
% <Rationals[<quiver with 2 vertices and 2 arrows>]>
% gap> M := RightModuleOverPathAlgebra(kQ, [2,2], [["a", [[1,0],[1,1]]]]);
% <[ 2, 2 ]>
% gap> SubRepresentation(M, [Basis(M)[1]-Basis(M)[2]]);
% <[ 1, 1 ]>
% gap> SubRepresentationInclusion(M, [Basis(M)[1]-Basis(M)[2]]);
% <<[ 1, 1 ]> ---> <[ 2, 2 ]>>
% \end{verbatim}
%% [show example from QPA intro]
\end{overprint}
\end{frame}


\begin{frame}{Direct sum}
\[
\xymatrix@R=.5em{
M_1 \ar@{^{(}->}[dr]^-{i_1} &&
M_2 \\
M_2 \ar@{^{(}->}[r]^-{i_2} &
{\rule[-10pt]{0pt}{30pt} M_1 \dsum M_2 \dsum M_3} 
\ar@{->>}[ur]^-{p_1}
\ar@{->>}[r]^-{p_2}
\ar@{->>}[dr]^-{p_3}
&
M_2 \\
M_3 \ar@{^{(}->}[ur]^-{i_3} &&
M_3
}
\]
\begin{itemize}
\item \texttt{DirectSumOfQPAModules}: $M_1 \dsum M_2 \dsum M_3$
\item \texttt{DirectSumInclusions}: $(i_1, i_2, i_3)$
\item \texttt{DirectSumProjections}: $(p_1, p_2, p_3)$
\end{itemize}
\end{frame}

\begin{frame}{Radical, socle and top}
\[
\xymatrix{
\rad M \ar@{^{(}->}[r]^i &
M \ar@{->>}[r]^p &
\Top M \\
& \soc M \ar@{^{(}->}[u]^j
}
\]
\begin{itemize}
\item \texttt{RadicalOfModule}: $\rad M$
\item \texttt{RadicalOfModuleInclusion}: $i$
\item \texttt{SocleOfModule}: $\soc M$
\item \texttt{SocleOfModuleInclusion}: $j$
\item \texttt{TopOfModule}: $\Top M$
\item \texttt{TopOfModuleInclusion}: $p$
\end{itemize}
\end{frame}

% socle, top

\begin{frame}{Modules: equality and isomorphism}
Three ways to compare modules $M$ and $N$:
\begin{itemize}
\item \texttt{IsIdenticalObj(M, N)}
      \onslide<2->{(not very interesting)}
\item \texttt{M = N}
\item \texttt{IsomorphicModules(M, N)}
\end{itemize}
\onslide<3>
For isomorphic modules:
\begin{itemize}
\item \texttt{IsomorphismOfModules(M, N)}\\
produces isomorphism $M \stackrel{\iso}{\to} N$
\end{itemize}
\end{frame}


\begin{frame}{Simple modules}
\[
Q\colon
\xymatrix{1 \ar[r]^a & 2 \ar[r]^b & 3}
\]
Simple $kQ$-modules:
\begin{gather*}
S_1\colon \xymatrix{k \ar[r] & 0 \ar[r] & 0} \\
S_2\colon \xymatrix{0 \ar[r] & k \ar[r] & 0} \\
S_3\colon \xymatrix{0 \ar[r] & 0 \ar[r] & k}
\end{gather*}
\pause
In QPA: \texttt{SimpleModules} gives $(S_1, S_2, S_3)$
\end{frame}

\begin{frame}{Indecomposable projective modules}
\[
Q\colon
\xymatrix{1 \ar[r]^a & 2 \ar[r]^b & 3}
\]
Indecomposable projective $kQ$-modules:
\begin{gather*}
P_1\colon \xymatrix{k \ar[r]^1 & k \ar[r]^1 & k} \\
P_2\colon \xymatrix{0 \ar[r]   & k \ar[r]^1 & k} \\
P_3\colon \xymatrix{0 \ar[r]   & 0 \ar[r]   & k}
\end{gather*}
\pause
In QPA: \texttt{IndecProjectiveModules} gives $(P_1, P_2, P_3)$
\end{frame}

\begin{frame}{Indecomposable injective modules}
\[
Q\colon
\xymatrix{1 \ar[r]^a & 2 \ar[r]^b & 3}
\]
Indecomposable injective $kQ$-modules:
\begin{gather*}
I_1\colon \xymatrix{k \ar[r]   & 0 \ar[r]   & 0} \\
I_2\colon \xymatrix{k \ar[r]^1 & k \ar[r]   & 0} \\
I_3\colon \xymatrix{k \ar[r]^1 & k \ar[r]^1 & k}
\end{gather*}
\pause
In QPA: \texttt{IndecInjectiveModules} gives $(I_1, I_2, I_3)$
\end{frame}


\subsection{Homomorphisms}

\begin{frame}[fragile]{Homomorphisms}
Recall:
\[
\xymatrix{
M \colon \ar@<-.2em>[d]^{h} &
0 \ar[r]     \ar[d] &
k \ar[r]^{5} \ar[d]_{\vv{3}{2}} &
k            \ar[d]^{1}
\\
N \colon &
k   \ar[r]_{\vv{0}{3}} &
k^2 \ar[r]^{\left( \begin{smallmatrix} 1 \\ 1 \end{smallmatrix} \right)} &
k
}
\]
\begin{verbatim}
gap> h := RightModuleHomOverAlgebra
          (M, N, [ [[0]], [[3,2]], [[1]] ]);
<<[ 0, 1, 1 ]> ---> <[ 1, 2, 1 ]>>
\end{verbatim}
\end{frame}

\begin{frame}{Hom spaces}
\begin{itemize}
\item \texttt{HomOverAlgebra(M,N)}
gives $k$-basis for $\Hom_A(M,N)$.
\item $k$-structure on homomorphisms in $\Hom_A(M,N)$:\\
use \texttt{f+g} and \texttt{scalar*f}
\end{itemize}
\end{frame}

\begin{frame}{Composition of homomorphisms}
\[
\xymatrix{
M_1 \ar[r]^f & M_2 \ar[r]^g & M_3
}
\]
\begin{center}
Composition: \texttt{f*g}
\end{center}
\end{frame}

\begin{frame}{Kernel, Cokernel, Image}
\begin{overprint}
\onslide<1>
\[
\xymatrix{
{\phantom{\ker f}} &
M \ar[r]^f &
N &
{\phantom{\coker f}} \\
&& {\phantom{\im f}}
}
\]
\onslide<2>
\[
\xymatrix{
\ker f \ar@{^{(}->}[r]^i &
M \ar[r]^f &
N &
{\phantom{\coker f}} \\
&& {\phantom{\im f}}
}
\]
\onslide<3>
\[
\xymatrix{
\ker f \ar@{^{(}->}[r]^i &
M \ar[r]^f &
N \ar@{->>}[r]^-p &
\coker f \\
&& {\phantom{\im f}}
}
\]
\onslide<4>
\[
\xymatrix{
\ker f \ar@{^{(}->}[r]^i &
M \ar[r]^f &
N \ar@{->>}[r]^-p &
\coker f \\
&& \im f \ar@{^{(}->}[u]^j
}
\]
\end{overprint}
\begin{itemize}
\onslide<2->
\item \texttt{Kernel}: $\ker f$
\item \texttt{KernelInclusion}: $i$
\onslide<3->
\item \texttt{CoKernel}: $\coker f$
\item \texttt{CoKernelProjection}: $p$
\onslide<4->
\item \texttt{Image}: $\im f$
\item \texttt{ImageInclusion}: $j$
\end{itemize}
\end{frame}

\section{Chain complexes}

\begin{frame}{Chain complexes}
\[
C \colon \cdots \to
 C_2 \xrightarrow{d_2}
 C_1 \xrightarrow{d_1}
 C_0 \xrightarrow{d_0}
 C_{-1} \xrightarrow{d_{-1}}
 C_{-2} \to \cdots
\]
\begin{itemize}
\pause
\item To represent a chain complex: Need infinite\\
list $(\ldots, d_2, d_1, d_0, d_{-1}, \ldots)$ of differentials.
\pause
\item Can not store all the differentials.
\pause
\item Need to describe them with finite data.
\end{itemize}
\end{frame}

\begin{frame}{Chain complexes in QPA}
Divide complex in three parts:
{\large
\[
\underbrace{
  \cdots \xrightarrow{d_{b+m+1}}
  \xrightarrow{d_{b+m}}
}_{\begin{array}{c}
   \text{``positive''} \\ \text{(infinite)}
   \end{array}}
\underbrace{
  \xrightarrow{d_{b+m-1}} \cdots
  \xrightarrow{d_b}
}_{\begin{array}{c}
   \text{``middle''} \\ \text{(finite)}
   \end{array}}
\underbrace{
  \xrightarrow{d_{b-1}} \xrightarrow{d_{b_2}}
  \cdots
}_{\begin{array}{c}
   \text{``negative''} \\ \text{(infinite)}
   \end{array}}
\]
}
\pause
\begin{itemize}
\item Middle part: List of differentials
\item Positive/negative part: Three possibilities
\end{itemize}
\end{frame}

\begin{frame}{Possibilities for the infinite parts}
Consider the positive part:
\[
\xymatrix{
\cdots
\ar[r]^{d_3} &
\ar[r]^{d_2} &
\ar[r]^{d_1} &
}
\]
(assuming it starts with $d_1$)
\end{frame}

\begin{frame}{Possibility 1: Repeating list}
\begin{itemize}
\item The same list $(r_1, \ldots, r_n)$ of differentials repeated
infinitely.
\end{itemize}
\vspace{1em}
\begin{overprint}
\onslide<2>
\[
\xymatrix@C=1.5em{
\cdots
\ar[r]^{d_9} &
\ar[r]^{d_8} &
\ar[r]^{d_7} &
\ar[r]^{d_6} &
\ar[r]^{d_5} &
\ar[r]^{d_4} &
\ar[r]^{d_3}_{r_3} &
\ar[r]^{d_2}_{r_2} &
\ar[r]^{d_1}_{r_1} &
}
\]
\onslide<3>
\[
\xymatrix@C=1.5em{
\cdots
\ar[r]^{d_9} &
\ar[r]^{d_8} &
\ar[r]^{d_7} &
\ar[r]^{d_6}_{r_3} &
\ar[r]^{d_5}_{r_2} &
\ar[r]^{d_4}_{r_1} &
\ar[r]^{d_3}_{r_3} &
\ar[r]^{d_2}_{r_2} &
\ar[r]^{d_1}_{r_1} &
}
\]
\onslide<4->
\[
\xymatrix@C=1.5em{
\cdots
\ar[r]^{d_9}_{r_3} &
\ar[r]^{d_8}_{r_2} &
\ar[r]^{d_7}_{r_1} &
\ar[r]^{d_6}_{r_3} &
\ar[r]^{d_5}_{r_2} &
\ar[r]^{d_4}_{r_1} &
\ar[r]^{d_3}_{r_3} &
\ar[r]^{d_2}_{r_2} &
\ar[r]^{d_1}_{r_1} &
}
\]
\end{overprint}
\onslide<5>
\begin{itemize}
\item Special case: Zero
\end{itemize}
\end{frame}

\begin{frame}{Possibility 2: Inductive function}
\begin{itemize}
\item Initial differential $d_1$
\item Function $f$ producing $d_{i+1}$ from $d_i$
\end{itemize}
\begin{overprint}
\onslide<2>
\[
\xymatrix@C=1em{
\cdots
\ar[rr] &
&
\ar[rr] &
&
\ar[rr] &
&
\ar[rr]^{d_1} &
&
}
\]
\onslide<3>
\[
\xymatrix@C=1em{
\cdots
\ar[rr] &
&
\ar[rr] &
&
\ar[rr]^{d_2} &
&
\ar[rr]^{d_1} &
\ar@(d,d)@{|->}[ll]^f &
}
\]
\onslide<4>
\[
\xymatrix@C=1em{
\cdots
\ar[rr] &
&
\ar[rr]^{d_3} &
&
\ar[rr]^{d_2} &
\ar@(d,d)[ll]^f &
\ar[rr]^{d_1} &
\ar@(d,d)@{|->}[ll]^f &
}
\]
\onslide<5->
\[
\xymatrix@C=1em{
\cdots
\ar[rr]^{d_4} &
&
\ar[rr]^{d_3} &
\ar@(d,d)[ll]^f &
\ar[rr]^{d_2} &
\ar@(d,d)[ll]^f &
\ar[rr]^{d_1} &
\ar@(d,d)@{|->}[ll]^f &
}
\]
\end{overprint}
\onslide<6>
\vspace{1em}
\begin{itemize}
\item Can convert to ``repeating list'' if repetition is detected
\end{itemize}
\end{frame}

\begin{frame}{Possibility 3: Positional function}
\begin{itemize}
\item Function $f$ producing $d_i$ from $i$.
\end{itemize}
\vspace{1em}
\begin{overprint}
\onslide<2>
\[
\xymatrix@C=1em@R=5em{
\cdots
\ar[rr] &&
\ar[rr] &&
\ar[rr] &&
\ar[rr] &&
\ar[rr] &&
\ar[rr] && \\
&&&
&&&
&&&
}
\]
\onslide<3>
\[
\xymatrix@C=1em@R=5em{
\cdots
\ar[rr]^{d_6} &&
\ar[rr] &&
\ar[rr] &&
\ar[rr] &&
\ar[rr] &&
\ar[rr] && \\
&&& 6 \ar@{|->}[ull]^f
&&&
&&&
}
\]
\onslide<4>
\[
\xymatrix@C=1em@R=5em{
\cdots
\ar[rr]^{d_6} &&
\ar[rr] &&
\ar[rr] &&
\ar[rr] &&
\ar[rr] &&
\ar[rr]^{d_1} && \\
&&& 6 \ar@{|->}[ull]^f
&&&
&&& 1 \ar@{|->}[urr]^f
}
\]
\onslide<5->
\[
\xymatrix@C=1em@R=5em{
\cdots
\ar[rr]^{d_6} &&
\ar[rr] &&
\ar[rr]^{d_4} &&
\ar[rr] &&
\ar[rr] &&
\ar[rr]^{d_1} && \\
&&& 6 \ar@{|->}[ull]^f
&&& 4 \ar@{|->}[ul]^f
&&& 1 \ar@{|->}[urr]^f
}
\]
\end{overprint}
\end{frame}

\begin{frame}{Creating a chain complex}{}
{\large
\[
\underbrace{
  \cdots \xrightarrow{d_{b+m+1}}
  \xrightarrow{d_{b+m}}
}_{\text{positive}}
\underbrace{
  \xrightarrow{d_{b+m-1}} \cdots
  \xrightarrow{d_b}
}_{\text{middle}}
\underbrace{
  \xrightarrow{d_{b-1}} \xrightarrow{d_{b_2}}
  \cdots
}_{\text{negative}}
\]
}
Must specify:
\begin{itemize}
\item Position $b$
\item Middle part: $(d_b, \ldots, d_{b+m-1})$
\item Positive part: Repeating list or inductive function or
positional function
\item Negative part: Repeating list or inductive function or
positional function
\end{itemize}
\end{frame}

\begin{frame}{Special complex constructors}
\begin{itemize}
\item \texttt{ZeroComplex}
\item \texttt{FiniteComplex}
\item \texttt{StalkComplex}
\end{itemize}
\end{frame}

\begin{frame}{Projective resolutions}
\[
\cdots \to P_2 \to P_1 \to P_0 \to M \to 0
\]
\begin{itemize}
\item \texttt{ProjectiveResolution}
\end{itemize}
\end{frame}


\end{document}
