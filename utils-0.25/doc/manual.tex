% generated by GAPDoc2LaTeX from XML source (Frank Luebeck)
\documentclass[a4paper,11pt]{report}

\usepackage{a4wide}
\sloppy
\pagestyle{myheadings}
\usepackage{amssymb}
\usepackage[latin1]{inputenc}
\usepackage{makeidx}
\makeindex
\usepackage{color}
\definecolor{FireBrick}{rgb}{0.5812,0.0074,0.0083}
\definecolor{RoyalBlue}{rgb}{0.0236,0.0894,0.6179}
\definecolor{RoyalGreen}{rgb}{0.0236,0.6179,0.0894}
\definecolor{RoyalRed}{rgb}{0.6179,0.0236,0.0894}
\definecolor{LightBlue}{rgb}{0.8544,0.9511,1.0000}
\definecolor{Black}{rgb}{0.0,0.0,0.0}

\definecolor{linkColor}{rgb}{0.0,0.0,0.554}
\definecolor{citeColor}{rgb}{0.0,0.0,0.554}
\definecolor{fileColor}{rgb}{0.0,0.0,0.554}
\definecolor{urlColor}{rgb}{0.0,0.0,0.554}
\definecolor{promptColor}{rgb}{0.0,0.0,0.589}
\definecolor{brkpromptColor}{rgb}{0.589,0.0,0.0}
\definecolor{gapinputColor}{rgb}{0.589,0.0,0.0}
\definecolor{gapoutputColor}{rgb}{0.0,0.0,0.0}

%%  for a long time these were red and blue by default,
%%  now black, but keep variables to overwrite
\definecolor{FuncColor}{rgb}{0.0,0.0,0.0}
%% strange name because of pdflatex bug:
\definecolor{Chapter }{rgb}{0.0,0.0,0.0}
\definecolor{DarkOlive}{rgb}{0.1047,0.2412,0.0064}


\usepackage{fancyvrb}

\usepackage{mathptmx,helvet}
\usepackage[T1]{fontenc}
\usepackage{textcomp}


\usepackage[
            pdftex=true,
            bookmarks=true,        
            a4paper=true,
            pdftitle={Written with GAPDoc},
            pdfcreator={LaTeX with hyperref package / GAPDoc},
            colorlinks=true,
            backref=page,
            breaklinks=true,
            linkcolor=linkColor,
            citecolor=citeColor,
            filecolor=fileColor,
            urlcolor=urlColor,
            pdfpagemode={UseNone}, 
           ]{hyperref}

\newcommand{\maintitlesize}{\fontsize{50}{55}\selectfont}

% write page numbers to a .pnr log file for online help
\newwrite\pagenrlog
\immediate\openout\pagenrlog =\jobname.pnr
\immediate\write\pagenrlog{PAGENRS := [}
\newcommand{\logpage}[1]{\protect\write\pagenrlog{#1, \thepage,}}
%% were never documented, give conflicts with some additional packages

\newcommand{\GAP}{\textsf{GAP}}

%% nicer description environments, allows long labels
\usepackage{enumitem}
\setdescription{style=nextline}

%% depth of toc
\setcounter{tocdepth}{1}





%% command for ColorPrompt style examples
\newcommand{\gapprompt}[1]{\color{promptColor}{\bfseries #1}}
\newcommand{\gapbrkprompt}[1]{\color{brkpromptColor}{\bfseries #1}}
\newcommand{\gapinput}[1]{\color{gapinputColor}{#1}}


\begin{document}

\logpage{[ 0, 0, 0 ]}
\begin{titlepage}
\mbox{}\vfill

\begin{center}{\maintitlesize \textbf{\textsf{Utils}\mbox{}}}\\
\vfill

\hypersetup{pdftitle=\textsf{Utils}}
\markright{\scriptsize \mbox{}\hfill \textsf{Utils} \hfill\mbox{}}
{\Huge \textbf{Utility functions in \textsf{GAP}\mbox{}}}\\
\vfill

{\Huge Version 0.25\mbox{}}\\[1cm]
{11/02/2016\mbox{}}\\[1cm]
\mbox{}\\[2cm]
{\Large \textbf{ Sebastian Gutsche,   \mbox{}}}\\
{\Large \textbf{ Stefan Kohl,   \mbox{}}}\\
{\Large \textbf{ Chris Wensley,   \mbox{}}}\\
{\Large \textbf{ and any other package author who transfers functions to this package. \mbox{}}}\\
\hypersetup{pdfauthor= Sebastian Gutsche,   ;  Stefan Kohl,   ;  Chris Wensley,   ;  and any other package author who transfers functions to this package. }
\end{center}\vfill

\mbox{}\\
{\mbox{}\\
\small \noindent \textbf{ Sebastian Gutsche,   }  Email: \href{mailto://gutsche@mathematik.uni-kl.de} {\texttt{gutsche@mathematik.uni-kl.de}}\\
  Homepage: \href{http://wwwb.math.rwth-aachen.de/~gutsche/} {\texttt{http://wwwb.math.rwth-aachen.de/\texttt{\symbol{126}}gutsche/}}}\\
{\mbox{}\\
\small \noindent \textbf{ Stefan Kohl,   }  Email: \href{mailto://stefan@mcs.st-and.ac.uk} {\texttt{stefan@mcs.st-and.ac.uk}}\\
  Homepage: \href{ http://www.gap-system.org/DevelopersPages/StefanKohl/ } {\texttt{ http://www.gap-system.org/DevelopersPages/StefanKohl/ }}}\\
{\mbox{}\\
\small \noindent \textbf{ Chris Wensley,   }  Email: \href{mailto://c.d.wensley@bangor.ac.uk} {\texttt{c.d.wensley@bangor.ac.uk}}\\
  Homepage: \href{http://pages.bangor.ac.uk/~mas023/} {\texttt{http://pages.bangor.ac.uk/\texttt{\symbol{126}}mas023/}}}\\
\end{titlepage}

\newpage\setcounter{page}{2}
{\small 
\section*{Abstract}
\logpage{[ 0, 0, 1 ]}
 The \textsf{Utils} package provides a space for utility functions in a variety of \textsf{GAP} packages to be collected together into a single package. In this way it is
hoped that they will become more visible to package authors. 

 Any package author who transfers a function to \textsf{Utils} will become an author of \textsf{Utils}. 

 The current version is 0.25, released 11th February 2016 for \textsf{GAP} 4.8. 

 This package relies on modifications to the library files oper.g and global.g.
It appears that these modifications will not appear in a GAP release until at
least version 4.8.3, so development of this package is suspended until further
notice. 

 Bug reports, suggestions and comments are, of course, welcome. Please contact
the last author at \href{mailto://c.d.wensley@bangor.ac.uk} {\texttt{c.d.wensley@bangor.ac.uk}} or submit an issue at the GitHub repository \href{http://github.com/gap-packages/utils/issues/} {\texttt{http://github.com/gap-packages/utils/issues/}}. \mbox{}}\\[1cm]
{\small 
\section*{Copyright}
\logpage{[ 0, 0, 2 ]}
 \index{License} {\copyright} 2015-2016 The \textsf{GAP} Group. \textsf{Utils} is free software; you can redistribute it and/or modify it under the terms of
the \href{http://www.fsf.org/licenses/gpl.html} {GNU General Public License} as published by the Free Software Foundation; either version 2 of the License,
or (at your option) any later version. \mbox{}}\\[1cm]
{\small 
\section*{Acknowledgements}
\logpage{[ 0, 0, 3 ]}
 This documentation was prepared with the \textsf{GAPDoc} package of Frank L{\"u}beck and Max Neunh{\"o}ffer. \mbox{}}\\[1cm]
\newpage

\def\contentsname{Contents\logpage{[ 0, 0, 4 ]}}

\tableofcontents
\newpage

           
\chapter{\textcolor{Chapter }{Introduction}}\label{chap-intro}
\logpage{[ 1, 0, 0 ]}
\hyperdef{L}{X7DFB63A97E67C0A1}{}
{
  The \textsf{Utils} package provides a space for utility functions from a variety of \textsf{GAP} packages to be collected together into a single package. In this way it is
hoped that they will become more visible to other package authors. Any package
author who transfers a function to \textsf{Utils} will become an author of \textsf{Utils}. 

 A function (or collection of functions) is suitable for transfer from a
package \textsf{Home} to \textsf{Utils} if the following conditions are satisfied. 
\begin{itemize}
\item  The function does not depend on the remaining functions in \textsf{Home} 
\item  The function does not do what can already be done with a \textsf{GAP} library function. 
\item  Documentation of the function and test examples are available. 
\item  The function is sufficiently non-specialised so that it might ber of use to
other authors. 
\end{itemize}
 

 The package is loaded with the command 
\begin{Verbatim}[commandchars=!@|,fontsize=\small,frame=single,label=Example]
  
  !gapprompt@gap>| !gapinput@LoadPackage( "utils" ); |
  
\end{Verbatim}
 

 The current version is 0.25 for \textsf{GAP} 4.8, released on 11th February 2016. 

 The package may be obtained as a compressed tar file \texttt{utils-0.25.tar.gz} by ftp from one of the following sites: 
\begin{itemize}
\item  any \textsf{GAP} archive, e.g. \href{http://www.gap-system.org/Packages/packages.html} {\texttt{http://www.gap-system.org/Packages/packages.html}}; 
\item  the Bangor site: \href{http://www.maths.bangor.ac.uk/chda/utils.html} {\texttt{http://www.maths.bangor.ac.uk/chda/utils.html}}; 
\end{itemize}
 \index{repository} The package also has a GitHub repository: \href{https://github.com/gap-packages/utils} {\texttt{https://github.com/gap-packages/utils}}. 

 Once the package is loaded, the manual \texttt{doc/manual.pdf} can be found in the documentation folder. The \texttt{html} versions, with or without MathJax, may be rebuilt as follows: 

 
\begin{Verbatim}[commandchars=!@|,fontsize=\small,frame=single,label=Example]
  
  !gapprompt@gap>| !gapinput@ReadPackage( "utils", "makedocrel.g" ); |
  
\end{Verbatim}
 

 It is possible to check that the package has been installed correctly by
running the test files: 

 
\begin{Verbatim}[commandchars=!@|,fontsize=\small,frame=single,label=Example]
  
  !gapprompt@gap>| !gapinput@ReadPackage( "utils", "tst/testall.g" );|
  #I  Testing .../pkg/utils/tst/lists.tst 
  ... 
  
\end{Verbatim}
 

 Additional information can be found on the \emph{Computational Higher-dimensional Discrete Algebra} website at \href{http://pages.bangor.ac.uk/~mas023/chda/} {\texttt{http://pages.bangor.ac.uk/\texttt{\symbol{126}}mas023/chda/}}. 
\section{\textcolor{Chapter }{The current transfer procedure}}\label{sect-current-procedure}
\logpage{[ 1, 1, 0 ]}
\hyperdef{L}{X7806D2388165CF6E}{}
{
  We consider here the process for transferring utility functions from an
existing package (the \textsf{Home} package, say) to \textsf{Utils}. 

 If the functions in \textsf{Home} all have names of the form \texttt{HOME{\textunderscore}FunctionName} then, in \textsf{Utils}, these functions are likely to be renamed as \texttt{FunctionName} or something similar. In this case the problem of duplicate declarations does
not arise. This is what has happened with transfers from the \textsf{AutoDoc} package. 

 The case where the function names are unchanged is more complicated. Initially
we tried out a process which allowed repeated declarations and installations
of the functions being transferred. This involved additions to the main
library files \texttt{global.g} and \texttt{oper.g}. Since there were misgivings about interfering in this way with basic
operations such as \texttt{BIND{\textunderscore}GLOBAL}, a simpler (but slightly less convenient) process has been adopted. 

 Using an alternative procedure, the following steps are proposed when making
transfers from \textsf{Home} to \textsf{Utils}. 
\begin{enumerate}
\item  (\textsf{Home}:) Offer functions for inclusion. This may be simply done by emailing a list
of functions. More usefully, email the declaration, implementation, test and
documentation files, e.g.: \texttt{home.gd}, \texttt{home.gi}, \texttt{home.tst} and \texttt{home.xml}, 
\item  (\textsf{Home}:) Declare that \textsc{m.n} is the last version of \textsf{Home} to contain these functions. 
\item  (\textsf{Utils}:) Add strings \mbox{\texttt{\mdseries\slshape "home"}} and \mbox{\texttt{\mdseries\slshape "m.n"}} to the list \texttt{UtilsPackageVersions} in the file \texttt{utils/lib/start.gd}. 
\begin{Verbatim}[commandchars=!@|,fontsize=\small,frame=single,label=Example]
  
  UtilsPackageVersions := 
    [ "autodoc",     "2016.01.31", 
      "resclasses",  "4.1.9", 
      "home",        "m.n",
      ...,           ... 
    ];
  
\end{Verbatim}
 
\item  (\textsf{Utils}:) Include the function declaration and implentation sections in suitable
files, enclosed within a conditional clause of the form: 
\begin{Verbatim}[commandchars=!@|,fontsize=\small,frame=single,label=Example]
  
  if OKtoReadFromUtils( "Home" ) then
  . . . . . . 
   <the code> 
  . . . . . . 
  fi;
  
\end{Verbatim}
 \index{OKtoReadFromUtils} The function \texttt{OKtoReadFromUtils} returns \texttt{true} only if there is an installed version of \textsf{Home} and this version is greater than \textsc{m.n}. So, at this stage, \emph{the copied code will not be read}. 
\item  (\textsf{Utils}:) Add the test and documentation material to the appropriate files. The
copied code can be tested by temporarily moving \textsf{Home} away from \textsf{GAP}'s package directory. 
\item  (\textsf{Utils}:) Release a new version of \textsf{Utils}. 
\item  (\textsf{Home}:) Edit out the declarations and implementations of all the transferred
functions. Edit or remove references to them in the manual and tests, and add
a note to the manual that these functions have been transferred. Add \textsf{Utils} to the list of \textsf{Home}'s required packages in \texttt{PackageInfo.g}. Release a new version of \textsf{Home}. 
\item  (\textsf{Utils}:) In due course, when the new version(s) of \textsf{Home} are well established, it may be possible to remove the conditional clauses
mentioned in item 4 above. 
\end{enumerate}
 

 A note on running the tests. As long as a function being transferred still
exists in the \textsf{Home} package, the code will not be read from \textsf{Utils}. So, when the tests are run, it is necessary to \texttt{LoadPackage( "home" );} before the function is called. The file \texttt{testall.g} makes sure that all the necessary packages are loaded before the individual
tests are called. }

 }

           
\chapter{\textcolor{Chapter }{Lists, Sets and Strings}}\label{chap-lists}
\logpage{[ 2, 0, 0 ]}
\hyperdef{L}{X7AE6EFC086C0EB3C}{}
{
  
\section{\textcolor{Chapter }{Functions for lists}}\label{sec-lists}
\logpage{[ 2, 1, 0 ]}
\hyperdef{L}{X7C3F1E7D878AAA65}{}
{
  

\subsection{\textcolor{Chapter }{DifferencesList}}
\logpage{[ 2, 1, 1 ]}\nobreak
\hyperdef{L}{X78B7C92681D2F13C}{}
{\noindent\textcolor{FuncColor}{$\triangleright$\ \ \texttt{DifferencesList({\mdseries\slshape L})\index{DifferencesList@\texttt{DifferencesList}}
\label{DifferencesList}
}\hfill{\scriptsize (function)}}\\
\noindent\textcolor{FuncColor}{$\triangleright$\ \ \texttt{QuotientsList({\mdseries\slshape L})\index{QuotientsList@\texttt{QuotientsList}}
\label{QuotientsList}
}\hfill{\scriptsize (function)}}\\
\noindent\textcolor{FuncColor}{$\triangleright$\ \ \texttt{FloatQuotientsList({\mdseries\slshape L})\index{FloatQuotientsList@\texttt{FloatQuotientsList}}
\label{FloatQuotientsList}
}\hfill{\scriptsize (function)}}\\


 These functions have transferred from package \textsf{ResClasses}. 

 They take a list $L$ of length $n$ and output the lists of length $n-1$ containing all the differences $L[i]-L[i-1]$ and all the quotients $L[i]/L[i-1]$ of consecutive entries in $L$. 

 In the quotient functions an error is returned if an entry is zero. 

 }

 
\begin{Verbatim}[commandchars=!@|,fontsize=\small,frame=single,label=Example]
  
  !gapprompt@gap>| !gapinput@L := [ 1, 3, 5, -1, -3, -5 ];;|
  !gapprompt@gap>| !gapinput@DifferencesList( L );        |
  [ 2, 2, -6, -2, -2 ]
  !gapprompt@gap>| !gapinput@QuotientsList( L );|
  [ 3, 5/3, -1/5, 3, 5/3 ]
  !gapprompt@gap>| !gapinput@FloatQuotientsList( L );|
  [ 3., 1.66667, -0.2, 3., 1.66667 ]
  !gapprompt@gap>| !gapinput@QuotientsList( [ 2, 1, 0, -1, -2 ] );|
  [ 1/2, 0, fail, 2 ]
  
\end{Verbatim}
 

\subsection{\textcolor{Chapter }{SearchCycle}}
\logpage{[ 2, 1, 2 ]}\nobreak
\hyperdef{L}{X86096E73858CFABD}{}
{\noindent\textcolor{FuncColor}{$\triangleright$\ \ \texttt{SearchCycle({\mdseries\slshape L})\index{SearchCycle@\texttt{SearchCycle}}
\label{SearchCycle}
}\hfill{\scriptsize (operation)}}\\


 This operation has transferred from package \textsf{RCWA}. 

 A utility function for detecting cycles in lists. But what \emph{is} the \emph{definition} of a cycle here? 

 }

 
\begin{Verbatim}[commandchars=!@|,fontsize=\small,frame=single,label=Example]
  
  !gapprompt@gap>| !gapinput@L := [1,2,3,4,2,3,4,5];|
  [ 1, 2, 3, 4, 2, 3, 4, 5 ]
  !gapprompt@gap>| !gapinput@SearchCycle(L);        |
  fail
  !gapprompt@gap>| !gapinput@L := [7,7,7];|
  [ 7, 7, 7 ]
  !gapprompt@gap>| !gapinput@SearchCycle(L);|
  [ 7 ]
  !gapprompt@gap>| !gapinput@L := [8,9,8,9];    |
  [ 8, 9, 8, 9 ]
  !gapprompt@gap>| !gapinput@SearchCycle(L);|
  fail
  !gapprompt@gap>| !gapinput@L := [8,9,8,9,8,9];|
  [ 8, 9, 8, 9, 8, 9 ]
  !gapprompt@gap>| !gapinput@SearchCycle(L);    |
  [ 8, 9 ]
  !gapprompt@gap>| !gapinput@L := [7,7];    |
  [ 7, 7 ]
  !gapprompt@gap>| !gapinput@SearchCycle(L);|
  Error, List Elements: <positions> must be a dense list of positive integers in
    mainpart := list{[ Int( n / 3 ) .. n ]}
   ; at /Applications/gap/my-dev/pkg/utils/lib/lists.gi:115 called from 
  <function "SearchCycle">( <arguments> )
   called from read-eval loop at line 23 of *stdin*
  you can replace <positions> via 'return <positions>;'
  !gapbrkprompt@brk>| !gapinput@|
\end{Verbatim}
 

\subsection{\textcolor{Chapter }{PrintListOneItemPerLine}}
\logpage{[ 2, 1, 3 ]}\nobreak
\hyperdef{L}{X80D711F78325FCE5}{}
{\noindent\textcolor{FuncColor}{$\triangleright$\ \ \texttt{PrintListOneItemPerLine({\mdseries\slshape L})\index{PrintListOneItemPerLine@\texttt{PrintListOneItemPerLine}}
\label{PrintListOneItemPerLine}
}\hfill{\scriptsize (operation)}}\\


 This operation has transferred from package \textsf{Gpd}. 

 Printing lists vertically, rather than horizantally, may be useful when the
entries are lengthy. 

 }

 
\begin{Verbatim}[commandchars=!@|,fontsize=\small,frame=single,label=Example]
  
  !gapprompt@gap>| !gapinput@PrintListOneItemPerLine( KnownPropertiesOfObject(L) );  |
  [ IsFinite,
    IsSmallList
    ]
  
\end{Verbatim}
 }

 
\section{\textcolor{Chapter }{Distinct and Common Representatives}}\logpage{[ 2, 2, 0 ]}
\hyperdef{L}{X82F443FF84B8FCE3}{}
{
 \index{distinct and common representatives} 

\subsection{\textcolor{Chapter }{DistinctRepresentatives}}
\logpage{[ 2, 2, 1 ]}\nobreak
\hyperdef{L}{X78105CAA847A888C}{}
{\noindent\textcolor{FuncColor}{$\triangleright$\ \ \texttt{DistinctRepresentatives({\mdseries\slshape list})\index{DistinctRepresentatives@\texttt{DistinctRepresentatives}}
\label{DistinctRepresentatives}
}\hfill{\scriptsize (operation)}}\\
\noindent\textcolor{FuncColor}{$\triangleright$\ \ \texttt{CommonRepresentatives({\mdseries\slshape list})\index{CommonRepresentatives@\texttt{CommonRepresentatives}}
\label{CommonRepresentatives}
}\hfill{\scriptsize (operation)}}\\
\noindent\textcolor{FuncColor}{$\triangleright$\ \ \texttt{CommonTransversal({\mdseries\slshape grp, subgrp})\index{CommonTransversal@\texttt{CommonTransversal}}
\label{CommonTransversal}
}\hfill{\scriptsize (operation)}}\\
\noindent\textcolor{FuncColor}{$\triangleright$\ \ \texttt{IsCommonTransversal({\mdseries\slshape grp, subgrp, list})\index{IsCommonTransversal@\texttt{IsCommonTransversal}}
\label{IsCommonTransversal}
}\hfill{\scriptsize (operation)}}\\


 These functions deal with lists of subsets of $[1 \ldots n]$ and construct systems of distinct and common representatives using simple,
non-recursive, combinatorial algorithms. 

 When $L$ is a set of $n$ subsets of $[1 \ldots n]$ and the Hall condition is satisfied (the union of any $k$ subsets has at least $k$ elements), a set of \texttt{DistinctRepresentatives} exists. 

 When $J,K$ are both lists of $n$ sets, the function \texttt{CommonRepresentatives} returns two lists: the set of representatives, and a permutation of the
subsets of the second list. It may also be used to provide a common
transversal for sets of left and right cosets of a subgroup $H$ of a group $G$, although a greedy algorithm is usually quicker. }

 

 
\begin{Verbatim}[commandchars=!@|,fontsize=\small,frame=single,label=Example]
  
  !gapprompt@gap>| !gapinput@J := [ [1,2,3], [3,4], [3,4], [1,2,4] ];;|
  !gapprompt@gap>| !gapinput@DistinctRepresentatives( J );|
  [ 1, 3, 4, 2 ]
  !gapprompt@gap>| !gapinput@K := [ [3,4], [1,2], [2,3], [2,3,4] ];;|
  !gapprompt@gap>| !gapinput@CommonRepresentatives( J, K );|
  [ [ 3, 3, 3, 1 ], [ 1, 3, 4, 2 ] ]
  !gapprompt@gap>| !gapinput@d16 := DihedralGroup( IsPermGroup, 16 );  SetName( d16, "d16" );|
  Group([ (1,2,3,4,5,6,7,8), (2,8)(3,7)(4,6) ])
  !gapprompt@gap>| !gapinput@c4 := Subgroup( d16, [ d16.1^2 ] );  SetName( c4, "c4" );|
  Group([ (1,3,5,7)(2,4,6,8) ])
  !gapprompt@gap>| !gapinput@RightCosets( d16, c4 );|
  [ RightCoset(c4,()), RightCoset(c4,(2,8)(3,7)(4,6)), RightCoset(c4,(1,8,7,6,5,
     4,3,2)), RightCoset(c4,(1,8)(2,7)(3,6)(4,5)) ]
  !gapprompt@gap>| !gapinput@trans := CommonTransversal( d16, c4 );|
  [ (), (2,8)(3,7)(4,6), (1,2,3,4,5,6,7,8), (1,2)(3,8)(4,7)(5,6) ]
  !gapprompt@gap>| !gapinput@IsCommonTransversal( d16, c4, trans );|
  true
  
\end{Verbatim}
 }

 
\section{\textcolor{Chapter }{Functions for strings}}\label{sec-strings}
\logpage{[ 2, 3, 0 ]}
\hyperdef{L}{X8033A2FE80FC2F2A}{}
{
  

\subsection{\textcolor{Chapter }{BlankFreeString}}
\logpage{[ 2, 3, 1 ]}\nobreak
\hyperdef{L}{X870C964E7804B266}{}
{\noindent\textcolor{FuncColor}{$\triangleright$\ \ \texttt{BlankFreeString({\mdseries\slshape obj})\index{BlankFreeString@\texttt{BlankFreeString}}
\label{BlankFreeString}
}\hfill{\scriptsize (function)}}\\


 This function has transferred from package \textsf{ResClasses}. 

 The result of \texttt{BlankFreeString( obj );} is a composite of the functions \texttt{String( obj )} and \texttt{RemoveCharacters( obj, " " );}. 

 }

 
\begin{Verbatim}[commandchars=!@|,fontsize=\small,frame=single,label=Example]
  
  !gapprompt@gap>| !gapinput@D12 := DihedralGroup( 12 );; |
  !gapprompt@gap>| !gapinput@BlankFreeString( D12 );|
  "Group([f1,f2,f3])"
  
\end{Verbatim}
 

\subsection{\textcolor{Chapter }{StringDotSuffix}}
\logpage{[ 2, 3, 2 ]}\nobreak
\hyperdef{L}{X79FC663C85A592EE}{}
{\noindent\textcolor{FuncColor}{$\triangleright$\ \ \texttt{StringDotSuffix({\mdseries\slshape str, suf})\index{StringDotSuffix@\texttt{StringDotSuffix}}
\label{StringDotSuffix}
}\hfill{\scriptsize (operation)}}\\


 These functions have transferred from package \textsf{AutoDoc}, and were originally named \texttt{AUTODOC{\textunderscore}GetSuffix} and \texttt{AUTODOC{\textunderscore}HasSuffix}. 

 When \texttt{StringDotSuffix} is given a string containing a "." it return its extension, i.e. the bit after
the last ".". 

 The function \texttt{StringEndsWithOtherString} predates the \textsf{GAP}4.8 function \texttt{EndsWith}, and is kept for consistency. }

 
\begin{Verbatim}[commandchars=!@|,fontsize=\small,frame=single,label=Example]
  
  !gapprompt@gap>| !gapinput@StringDotSuffix( "file.ext" );|
  "ext"
  !gapprompt@gap>| !gapinput@StringDotSuffix( "file.ext.bak" );|
  "bak"
  !gapprompt@gap>| !gapinput@StringDotSuffix( "file." );|
  ""
  !gapprompt@gap>| !gapinput@StringDotSuffix( "Hello" );   |
  fail
  !gapprompt@gap>| !gapinput@StringEndsWithOtherString( "file.ext", ".txt" );|
  false
  
\end{Verbatim}
 }

 }

           
\chapter{\textcolor{Chapter }{Number-theoretic functions}}\label{chap-number}
\logpage{[ 3, 0, 0 ]}
\hyperdef{L}{X86E71C1687F2D0AD}{}
{
  
\section{\textcolor{Chapter }{Functions for integers}}\label{sec-integers}
\logpage{[ 3, 1, 0 ]}
\hyperdef{L}{X7D33B5B17BF785CA}{}
{
  

\subsection{\textcolor{Chapter }{AllSmoothIntegers}}
\logpage{[ 3, 1, 1 ]}\nobreak
\hyperdef{L}{X802064F37D50F46A}{}
{\noindent\textcolor{FuncColor}{$\triangleright$\ \ \texttt{AllSmoothIntegers({\mdseries\slshape maxp, maxn})\index{AllSmoothIntegers@\texttt{AllSmoothIntegers}}
\label{AllSmoothIntegers}
}\hfill{\scriptsize (function)}}\\


 This function has transferred from package \textsf{RCWA}. 

 The function \texttt{AllSmoothIntegers(\mbox{\texttt{\mdseries\slshape maxp}},\mbox{\texttt{\mdseries\slshape maxn}})} returns a list of all integers less than or equal to \mbox{\texttt{\mdseries\slshape maxn}} which do not have prime divisors exceeding \mbox{\texttt{\mdseries\slshape maxp}}. 

 }

 
\begin{Verbatim}[commandchars=!@|,fontsize=\small,frame=single,label=Example]
  
  !gapprompt@gap>| !gapinput@AllSmoothIntegers( 7, 100 );|
  [ 1, 2, 3, 4, 5, 6, 7, 8, 9, 10, 12, 14, 15, 16, 18, 20, 21, 24, 25, 27, 28, 
    30, 32, 35, 36, 40, 42, 45, 48, 49, 50, 54, 56, 60, 63, 64, 70, 72, 75, 80, 
    81, 84, 90, 96, 98, 100 ]
  !gapprompt@gap>| !gapinput@Length(last);|
  46
  
\end{Verbatim}
 

\subsection{\textcolor{Chapter }{AllProducts}}
\logpage{[ 3, 1, 2 ]}\nobreak
\hyperdef{L}{X78BE6B8B878D250D}{}
{\noindent\textcolor{FuncColor}{$\triangleright$\ \ \texttt{AllProducts({\mdseries\slshape L, k})\index{AllProducts@\texttt{AllProducts}}
\label{AllProducts}
}\hfill{\scriptsize (function)}}\\


 This function has transferred from package \textsf{RCWA}. Note that it has been removed from version 4.0.0 of \textsf{RCWA} but will not yet be read from \textsf{Utils}. 

 The function \texttt{AllProducts(\mbox{\texttt{\mdseries\slshape L}},\mbox{\texttt{\mdseries\slshape k}})} returns the list of all products of \mbox{\texttt{\mdseries\slshape k}} entries of the list{\nobreakspace}\mbox{\texttt{\mdseries\slshape L}}. 

 }

 
\begin{Verbatim}[commandchars=!@|,fontsize=\small,frame=single,label=Example]
  
  !gapprompt@gap>| !gapinput@AllProducts([1..4],3); |
  [ 1, 2, 3, 4, 2, 4, 6, 8, 3, 6, 9, 12, 4, 8, 12, 16, 2, 4, 6, 8, 4, 8, 12, 
    16, 6, 12, 18, 24, 8, 16, 24, 32, 3, 6, 9, 12, 6, 12, 18, 24, 9, 18, 27, 
    36, 12, 24, 36, 48, 4, 8, 12, 16, 8, 16, 24, 32, 12, 24, 36, 48, 16, 32, 
    48, 64 ]
  !gapprompt@gap>| !gapinput@Set(last);            |
  [ 1, 2, 3, 4, 6, 8, 9, 12, 16, 18, 24, 27, 32, 36, 48, 64 ]
  
\end{Verbatim}
 

\subsection{\textcolor{Chapter }{RestrictedPartitionsWithoutRepetitions}}
\logpage{[ 3, 1, 3 ]}\nobreak
\hyperdef{L}{X845F46E579CEA43F}{}
{\noindent\textcolor{FuncColor}{$\triangleright$\ \ \texttt{RestrictedPartitionsWithoutRepetitions({\mdseries\slshape L, k})\index{RestrictedPartitionsWithoutRepetitions@\texttt{Restricted}\-\texttt{Partitions}\-\texttt{Without}\-\texttt{Repetitions}}
\label{RestrictedPartitionsWithoutRepetitions}
}\hfill{\scriptsize (function)}}\\


 This function has transferred from package \textsf{RCWA}. 

 Given a positive integer \mbox{\texttt{\mdseries\slshape n}} and a set of positive integers \mbox{\texttt{\mdseries\slshape S}}, this function returns a list of all partitions of \mbox{\texttt{\mdseries\slshape n}} into distinct elements of \mbox{\texttt{\mdseries\slshape S}}. The only difference to \texttt{RestrictedPartitions} is that no repetitions are allowed. 

 }

 
\begin{Verbatim}[commandchars=!@|,fontsize=\small,frame=single,label=Example]
  
  !gapprompt@gap>| !gapinput@RestrictedPartitions( 20, [4..10] );|
  [ [ 4, 4, 4, 4, 4 ], [ 5, 5, 5, 5 ], [ 6, 5, 5, 4 ], [ 6, 6, 4, 4 ], 
    [ 7, 5, 4, 4 ], [ 7, 7, 6 ], [ 8, 4, 4, 4 ], [ 8, 6, 6 ], [ 8, 7, 5 ], 
    [ 8, 8, 4 ], [ 9, 6, 5 ], [ 9, 7, 4 ], [ 10, 5, 5 ], [ 10, 6, 4 ], 
    [ 10, 10 ] ]
  !gapprompt@gap>| !gapinput@RestrictedPartitionsWithoutRepetitions( 20, [4..10] );|
  [ [ 10, 6, 4 ], [ 9, 7, 4 ], [ 9, 6, 5 ], [ 8, 7, 5 ] ]
  
\end{Verbatim}
 

\subsection{\textcolor{Chapter }{ExponentOfPrime}}
\logpage{[ 3, 1, 4 ]}\nobreak
\hyperdef{L}{X7C1AF2467FB55D79}{}
{\noindent\textcolor{FuncColor}{$\triangleright$\ \ \texttt{ExponentOfPrime({\mdseries\slshape n, p})\index{ExponentOfPrime@\texttt{ExponentOfPrime}}
\label{ExponentOfPrime}
}\hfill{\scriptsize (function)}}\\


 This function has transferred from package \textsf{RCWA}. 

 The function \texttt{ExponentOfPrime(\mbox{\texttt{\mdseries\slshape n}},\mbox{\texttt{\mdseries\slshape p}})} returns the exponent of the prime \mbox{\texttt{\mdseries\slshape p}} in the prime factorization of \mbox{\texttt{\mdseries\slshape n}}. 

 }

 
\begin{Verbatim}[commandchars=!@|,fontsize=\small,frame=single,label=Example]
  
  !gapprompt@gap>| !gapinput@ExponentOfPrime( 13577531, 11 ); |
  3
  
\end{Verbatim}
 

\subsection{\textcolor{Chapter }{NextProbablyPrimeInt}}
\logpage{[ 3, 1, 5 ]}\nobreak
\hyperdef{L}{X81708BF4858505E8}{}
{\noindent\textcolor{FuncColor}{$\triangleright$\ \ \texttt{NextProbablyPrimeInt({\mdseries\slshape n})\index{NextProbablyPrimeInt@\texttt{NextProbablyPrimeInt}}
\label{NextProbablyPrimeInt}
}\hfill{\scriptsize (function)}}\\


 This function has transferred from package \textsf{ResClasses}. 

 The function \texttt{NextProbablyPrimeInt(\mbox{\texttt{\mdseries\slshape n}})} does the same as \texttt{NextPrimeInt(\mbox{\texttt{\mdseries\slshape n}})} except that for reasons of performance it tests numbers only for \texttt{IsProbablyPrimeInt(\mbox{\texttt{\mdseries\slshape n}})} instead of \texttt{IsPrimeInt(\mbox{\texttt{\mdseries\slshape n}})}. For large \mbox{\texttt{\mdseries\slshape n}}, this function is much faster than \texttt{NextPrimeInt(\mbox{\texttt{\mdseries\slshape n}})} 

 }

 
\begin{Verbatim}[commandchars=!@|,fontsize=\small,frame=single,label=Example]
  
  !gapprompt@gap>| !gapinput@n := 2^251;|
  3618502788666131106986593281521497120414687020801267626233049500247285301248
  !gapprompt@gap>| !gapinput@time;      |
  0
  !gapprompt@gap>| !gapinput@NextProbablyPrimeInt( n );|
  3618502788666131106986593281521497120414687020801267626233049500247285301313
  !gapprompt@gap>| !gapinput@time;                     |
  1
  !gapprompt@gap>| !gapinput@NextPrimeInt( n );        |
  3618502788666131106986593281521497120414687020801267626233049500247285301313
  !gapprompt@gap>| !gapinput@time;             |
  12346
  
\end{Verbatim}
 }

 }

           
\chapter{\textcolor{Chapter }{Groups and homomorphisms}}\label{chap-groups}
\logpage{[ 4, 0, 0 ]}
\hyperdef{L}{X8171DAF2833FF728}{}
{
  
\section{\textcolor{Chapter }{Functions for groups}}\label{sec-groups}
\logpage{[ 4, 1, 0 ]}
\hyperdef{L}{X7E21E6D285E6B12C}{}
{
  

\subsection{\textcolor{Chapter }{IsCommuting}}
\logpage{[ 4, 1, 1 ]}\nobreak
\hyperdef{L}{X803A050C7A183CCC}{}
{\noindent\textcolor{FuncColor}{$\triangleright$\ \ \texttt{IsCommuting({\mdseries\slshape a, b})\index{IsCommuting@\texttt{IsCommuting}}
\label{IsCommuting}
}\hfill{\scriptsize (operation)}}\\


 This operation has transferred from package \textsf{ResClasses}. 

 This operation tests whether two group elements commute. 

 }

 
\begin{Verbatim}[commandchars=!@|,fontsize=\small,frame=single,label=Example]
  
  !gapprompt@gap>| !gapinput@D12 := DihedralGroup( 12 );  SetName( D12, "D12" ); |
  <pc group of size 12 with 3 generators>
  !gapprompt@gap>| !gapinput@a := D12.1;;  b := D12.2;;  |
  !gapprompt@gap>| !gapinput@IsCommuting( a, b );|
  false
  
\end{Verbatim}
 

\subsection{\textcolor{Chapter }{ListOfPowers}}
\logpage{[ 4, 1, 2 ]}\nobreak
\hyperdef{L}{X87A8F01286548037}{}
{\noindent\textcolor{FuncColor}{$\triangleright$\ \ \texttt{ListOfPowers({\mdseries\slshape g, exp})\index{ListOfPowers@\texttt{ListOfPowers}}
\label{ListOfPowers}
}\hfill{\scriptsize (operation)}}\\


 This operation has transferred from package \textsf{RCWA}. 

 The operation \texttt{ListOfPowers(g,exp)} returns the list $[g,g^2,...,g^{exp}]$ of powers of the element $g$. 

 }

 
\begin{Verbatim}[commandchars=!@|,fontsize=\small,frame=single,label=Example]
  
  !gapprompt@gap>| !gapinput@ListOfPowers( D12.2, 6 );|
  [ f2, f3, f2*f3, f3^2, f2*f3^2, <identity> of ... ]
  
\end{Verbatim}
 

\subsection{\textcolor{Chapter }{GeneratorsAndInverses}}
\logpage{[ 4, 1, 3 ]}\nobreak
\hyperdef{L}{X820B71307E41BEE5}{}
{\noindent\textcolor{FuncColor}{$\triangleright$\ \ \texttt{GeneratorsAndInverses({\mdseries\slshape G})\index{GeneratorsAndInverses@\texttt{GeneratorsAndInverses}}
\label{GeneratorsAndInverses}
}\hfill{\scriptsize (operation)}}\\


 This operation has transferred from package \textsf{RCWA}. 

 This operation returns a list containing the generators of $G$ followed by the inverses of these generators. 

 }

 
\begin{Verbatim}[commandchars=!@|,fontsize=\small,frame=single,label=Example]
  
  !gapprompt@gap>| !gapinput@GeneratorsAndInverses( D12 );|
  [ f1, f2, f3, f1, f2*f3^2, f3^2 ]
  
\end{Verbatim}
 

\subsection{\textcolor{Chapter }{UpperFittingSeries}}
\logpage{[ 4, 1, 4 ]}\nobreak
\hyperdef{L}{X84CF95227F9D562F}{}
{\noindent\textcolor{FuncColor}{$\triangleright$\ \ \texttt{UpperFittingSeries({\mdseries\slshape G})\index{UpperFittingSeries@\texttt{UpperFittingSeries}}
\label{UpperFittingSeries}
}\hfill{\scriptsize (attribute)}}\\
\noindent\textcolor{FuncColor}{$\triangleright$\ \ \texttt{LowerFittingSeries({\mdseries\slshape G})\index{LowerFittingSeries@\texttt{LowerFittingSeries}}
\label{LowerFittingSeries}
}\hfill{\scriptsize (attribute)}}\\
\noindent\textcolor{FuncColor}{$\triangleright$\ \ \texttt{FittingLength({\mdseries\slshape G})\index{FittingLength@\texttt{FittingLength}}
\label{FittingLength}
}\hfill{\scriptsize (attribute)}}\\


 These operations have transferred from package \textsf{ResClasses}. Note that these three functions have been removed from version 4.0.1 of \textsf{ResClasses}, but will not yet be read by \textsf{Utils}. 

 The upper and lower Fitting series and the Fitting length of a solvable group
are described here: \href{https://en.wikipedia.org/wiki/Fitting_length} {\texttt{https://en.wikipedia.org/wiki/Fitting{\textunderscore}length}}. 

 }

 
\begin{Verbatim}[commandchars=!@|,fontsize=\small,frame=single,label=Example]
  
  !gapprompt@gap>| !gapinput@UpperFittingSeries( D12 );|
  [ Group([  ]), Group([ f3, f2*f3 ]), Group([ f3, f2*f3, f1 ]) ]
  !gapprompt@gap>| !gapinput@LowerFittingSeries( D12 );|
  [ D12, Group([ f3 ]), Group([  ]) ]
  !gapprompt@gap>| !gapinput@FittingLength( D12 );|
  2
  
\end{Verbatim}
 }

 
\section{\textcolor{Chapter }{Functions for words in finitely presented groups}}\label{sec-words}
\logpage{[ 4, 2, 0 ]}
\hyperdef{L}{X7ECF4E3A7FC30BD7}{}
{
  

\subsection{\textcolor{Chapter }{ReducedWordByOrdersOfGenerators}}
\logpage{[ 4, 2, 1 ]}\nobreak
\hyperdef{L}{X7A0A224D84C52539}{}
{\noindent\textcolor{FuncColor}{$\triangleright$\ \ \texttt{ReducedWordByOrdersOfGenerators({\mdseries\slshape w, gensords})\index{ReducedWordByOrdersOfGenerators@\texttt{ReducedWordByOrdersOfGenerators}}
\label{ReducedWordByOrdersOfGenerators}
}\hfill{\scriptsize (attribute)}}\\
\noindent\textcolor{FuncColor}{$\triangleright$\ \ \texttt{NormalizedRelator({\mdseries\slshape w, gensords})\index{NormalizedRelator@\texttt{NormalizedRelator}}
\label{NormalizedRelator}
}\hfill{\scriptsize (attribute)}}\\


 These operations have transferred from package \textsf{RCWA}. 

 Some description needed here. 

 }

 
\begin{Verbatim}[commandchars=!@|,fontsize=\small,frame=single,label=Example]
  
  !gapprompt@gap>| !gapinput@## some examples needed here |
  
\end{Verbatim}
 }

 
\section{\textcolor{Chapter }{Functions for group homomorphisms}}\label{sec-homomorphisms}
\logpage{[ 4, 3, 0 ]}
\hyperdef{L}{X80A512877F515DE7}{}
{
  

\subsection{\textcolor{Chapter }{EpimorphismByGenerators}}
\logpage{[ 4, 3, 1 ]}\nobreak
\hyperdef{L}{X80C9A0B583FEA7B9}{}
{\noindent\textcolor{FuncColor}{$\triangleright$\ \ \texttt{EpimorphismByGenerators({\mdseries\slshape G, H})\index{EpimorphismByGenerators@\texttt{EpimorphismByGenerators}}
\label{EpimorphismByGenerators}
}\hfill{\scriptsize (attribute)}}\\


 These operations have transferred from package \textsf{RCWA}. 

 This function maps the generators of $G$ to those of $H$. It is not checked that this map is a group homomorphism! 

 }

 
\begin{Verbatim}[commandchars=!@|,fontsize=\small,frame=single,label=Example]
  
  !gapprompt@gap>| !gapinput@G:=Group((1,2,3),(3,4,5));;|
  !gapprompt@gap>| !gapinput@H:=Group((6,7),(8,9));;    |
  !gapprompt@gap>| !gapinput@e:=EpimorphismByGenerators(G,H);|
  [ (1,2,3), (3,4,5) ] -> [ (6,7), (8,9) ]
  !gapprompt@gap>| !gapinput@IsGroupHomomorphism(e);|
  true
  
\end{Verbatim}
 }

 }

           
\chapter{\textcolor{Chapter }{Records}}\label{chap-record}
\logpage{[ 5, 0, 0 ]}
\hyperdef{L}{X7AA1073C7E943DD7}{}
{
  
\section{\textcolor{Chapter }{Functions for records}}\label{sec-records}
\logpage{[ 5, 1, 0 ]}
\hyperdef{L}{X82B3D1D583CDF0E5}{}
{
  

\subsection{\textcolor{Chapter }{SetIfMissing}}
\logpage{[ 5, 1, 1 ]}\nobreak
\hyperdef{L}{X8571EDE27B16E868}{}
{\noindent\textcolor{FuncColor}{$\triangleright$\ \ \texttt{SetIfMissing({\mdseries\slshape rec, name, val})\index{SetIfMissing@\texttt{SetIfMissing}}
\label{SetIfMissing}
}\hfill{\scriptsize (function)}}\\


 This function has transferred from package \textsf{AutoDoc}, where it was called \texttt{AUTODOC{\textunderscore}WriteOnce}. It writes into a record provided the position is not yet bound. 

 }

 
\begin{Verbatim}[commandchars=!@|,fontsize=\small,frame=single,label=Example]
  
  !gapprompt@gap>| !gapinput@r := rec( a := 1, b := 2 );;                                      |
  !gapprompt@gap>| !gapinput@SetIfMissing( r, "c", 3 );|
  !gapprompt@gap>| !gapinput@RecNames( r );|
  [ "b", "c", "a" ]
  !gapprompt@gap>| !gapinput@SetIfMissing( r, "c", 4 );|
  !gapprompt@gap>| !gapinput@r;|
  rec( a := 1, b := 2, c := 3 )
  
\end{Verbatim}
 

\subsection{\textcolor{Chapter }{AssignGlobals}}
\logpage{[ 5, 1, 2 ]}\nobreak
\hyperdef{L}{X84D82EB579B2ACCD}{}
{\noindent\textcolor{FuncColor}{$\triangleright$\ \ \texttt{AssignGlobals({\mdseries\slshape rec})\index{AssignGlobals@\texttt{AssignGlobals}}
\label{AssignGlobals}
}\hfill{\scriptsize (function)}}\\


 This function has transferred from package \textsf{RCWA}. 

 This function assigns the record components of \mbox{\texttt{\mdseries\slshape rec}} to global variables with the same names. 

 }

 
\begin{Verbatim}[commandchars=!@|,fontsize=\small,frame=single,label=Example]
  
  !gapprompt@gap>| !gapinput@AssignGlobals( r );|
  The following global variables have been assigned:
  [ "b", "c", "a" ]
  !gapprompt@gap>| !gapinput@[a,b,c];|
  [ 1, 2, 3 ]
  
\end{Verbatim}
 }

 }

            
\chapter{\textcolor{Chapter }{Various other functions}}\label{chap-others}
\logpage{[ 6, 0, 0 ]}
\hyperdef{L}{X83EFC3178180D918}{}
{
  
\section{\textcolor{Chapter }{DownloadFile, SendEmail and EmailLogFile}}\label{sec:SendEmailAndEmailLogFile}
\logpage{[ 6, 1, 0 ]}
\hyperdef{L}{X81DD0163859E9EF5}{}
{
  

\subsection{\textcolor{Chapter }{DownloadFile}}
\logpage{[ 6, 1, 1 ]}\nobreak
\hyperdef{L}{X7C1543408472A82B}{}
{\noindent\textcolor{FuncColor}{$\triangleright$\ \ \texttt{DownloadFile({\mdseries\slshape url})\index{DownloadFile@\texttt{DownloadFile}}
\label{DownloadFile}
}\hfill{\scriptsize (function)}}\\
\textbf{\indent Returns:\ }
 the contents of the file with URL \mbox{\texttt{\mdseries\slshape url}} in the form of a string if that file exists and the download was successful,
and \texttt{fail} otherwise. 



 This function needs the \textsf{IO} package, and as most system-related functions it works only under UNIX. Also
the computer must of course be connected to the Internet. }

 

\subsection{\textcolor{Chapter }{SendEmail}}
\logpage{[ 6, 1, 2 ]}\nobreak
\hyperdef{L}{X7F27919085001EA8}{}
{\noindent\textcolor{FuncColor}{$\triangleright$\ \ \texttt{SendEmail({\mdseries\slshape sendto, copyto, subject, text})\index{SendEmail@\texttt{SendEmail}}
\label{SendEmail}
}\hfill{\scriptsize (function)}}\\
\textbf{\indent Returns:\ }
 zero if everything worked correctly, and a system error number otherwise. 



 Sends an e-mail with subject \mbox{\texttt{\mdseries\slshape subject}} and body \mbox{\texttt{\mdseries\slshape text}} to the addresses in the list \mbox{\texttt{\mdseries\slshape sendto}}, and copies it to those in the list \mbox{\texttt{\mdseries\slshape copyto}}. The first two arguments must be lists of strings, and the latter two must be
strings. 

 As most system-related functions, \texttt{SendEmail} works only under UNIX. Also the computer must of course be connected to the
Internet. }

 

\subsection{\textcolor{Chapter }{EmailLogFile}}
\logpage{[ 6, 1, 3 ]}\nobreak
\hyperdef{L}{X8000E5CF7DF4339C}{}
{\noindent\textcolor{FuncColor}{$\triangleright$\ \ \texttt{EmailLogFile({\mdseries\slshape addresses})\index{EmailLogFile@\texttt{EmailLogFile}}
\label{EmailLogFile}
}\hfill{\scriptsize (function)}}\\
\textbf{\indent Returns:\ }
 zero if everything worked correctly, and a system error number otherwise. 



 Sends the current log file by e-mail to \mbox{\texttt{\mdseries\slshape addresses}}, if \textsf{GAP} is in logging mode and one is working under UNIX, and does nothing otherwise.
The argument \mbox{\texttt{\mdseries\slshape addresses}} must be either a list of e-mail addresses or a single e-mail address. Long log
files are abbreviated, i.e. if the log file is larger than 64KB, then any
output is truncated at 1KB, and if the log file is still longer than 64KB
afterwards, it is truncated at{\nobreakspace}64KB. }

 }

 
\section{\textcolor{Chapter }{A simple caching facility}}\label{sec:cache}
\logpage{[ 6, 2, 0 ]}
\hyperdef{L}{X840A069D87506078}{}
{
  

\subsection{\textcolor{Chapter }{SetupCache}}
\logpage{[ 6, 2, 1 ]}\nobreak
\hyperdef{L}{X8049726581FED930}{}
{\noindent\textcolor{FuncColor}{$\triangleright$\ \ \texttt{SetupCache({\mdseries\slshape name, size})\index{SetupCache@\texttt{SetupCache}}
\label{SetupCache}
}\hfill{\scriptsize (function)}}\\
\noindent\textcolor{FuncColor}{$\triangleright$\ \ \texttt{PutIntoCache({\mdseries\slshape name, key, value})\index{PutIntoCache@\texttt{PutIntoCache}}
\label{PutIntoCache}
}\hfill{\scriptsize (function)}}\\
\noindent\textcolor{FuncColor}{$\triangleright$\ \ \texttt{FetchFromCache({\mdseries\slshape name, key})\index{FetchFromCache@\texttt{FetchFromCache}}
\label{FetchFromCache}
}\hfill{\scriptsize (function)}}\\


 These functions have transferred from package \textsf{ResClasses}. 

 The function \texttt{SetupCache} creates an empty cache named \texttt{name} for at most \texttt{size} values. 

 The function \texttt{PutIntoCache} puts the entry \texttt{value} with key \texttt{key} into the cache named \texttt{name}. 

 The function \texttt{FetchFromCache} picks the entry with key \texttt{key} from the cache named \texttt{name}, and returns fail if no such entry exists. }

 

 
\begin{Verbatim}[commandchars=@|A,fontsize=\small,frame=single,label=Example]
  
  @gapprompt|gap>A @gapinput|## examples needed! A
  
\end{Verbatim}
 }

 
\section{\textcolor{Chapter }{Operations on folders}}\label{sec:folder-ops}
\logpage{[ 6, 3, 0 ]}
\hyperdef{L}{X7CC93AD18473B735}{}
{
  

\subsection{\textcolor{Chapter }{FindMatchingFiles}}
\logpage{[ 6, 3, 1 ]}\nobreak
\hyperdef{L}{X87ACF0197B180DAB}{}
{\noindent\textcolor{FuncColor}{$\triangleright$\ \ \texttt{FindMatchingFiles({\mdseries\slshape pkg, dirs, extns})\index{FindMatchingFiles@\texttt{FindMatchingFiles}}
\label{FindMatchingFiles}
}\hfill{\scriptsize (function)}}\\
\noindent\textcolor{FuncColor}{$\triangleright$\ \ \texttt{CreateDirIfMissing({\mdseries\slshape str})\index{CreateDirIfMissing@\texttt{CreateDirIfMissing}}
\label{CreateDirIfMissing}
}\hfill{\scriptsize (function)}}\\


 These functions have transferred from package \textsf{AutoDoc}. 

 \texttt{FindMatchingFiles} scans the given (by name) subdirectories of a package directory for files with
one of the given extensions, and returns the corresponding filenames, as paths
relative to the package directory. 

 \texttt{CreateDirIfMissing} checks whether the given directory exists and, if not, attempts to create it.
In either case \texttt{true} is returned. 

 \emph{Warning:} this function relies on the undocumented library function \texttt{CreateDir}, so use it with caution. 

 }

 

 
\begin{Verbatim}[commandchars=!@|,fontsize=\small,frame=single,label=Example]
  
  !gapprompt@gap>| !gapinput@FindMatchingFiles( "utils", [ "/", "tst" ], [ "g", "log" ] );|
  [ "/PackageInfo.g", "/feb1.log", "/init.g", "/makedocrel.g", "/read.g", 
    "/start.g", "/test.log", "tst/testall.g" ]
  !gapprompt@gap>| !gapinput@CreateDirIfMissing( "/Applications/gap/temp/" );|
  true
  
\end{Verbatim}
 }

 
\section{\textcolor{Chapter }{File operations}}\label{sec:log2html}
\logpage{[ 6, 4, 0 ]}
\hyperdef{L}{X81A0A4FF842B039B}{}
{
  

\subsection{\textcolor{Chapter }{Log2HTML}}
\logpage{[ 6, 4, 1 ]}\nobreak
\hyperdef{L}{X7B7ECADF85F748BE}{}
{\noindent\textcolor{FuncColor}{$\triangleright$\ \ \texttt{Log2HTML({\mdseries\slshape filename})\index{Log2HTML@\texttt{Log2HTML}}
\label{Log2HTML}
}\hfill{\scriptsize (function)}}\\


 This function has transferred from package \textsf{RCWA}. 

 This function converts the \textsf{GAP} logfile logfilename to HTML. The extension of the input file must be \mbox{\texttt{\mdseries\slshape *.log}}. The name of the output file is the same as the one of the input file except
that the extension \mbox{\texttt{\mdseries\slshape *.log}} is replaced by \mbox{\texttt{\mdseries\slshape *.html}}. There is a sample CSS file in \texttt{utils/doc/gaplog.css}, which you can adjust to your taste. 

 }

 

 
\begin{Verbatim}[commandchars=!@|,fontsize=\small,frame=single,label=Example]
  
  !gapprompt@gap>| !gapinput@LogTo("feb1.log");|
  !gapprompt@gap>| !gapinput@FindMatchingFiles( "utils", [""], ["g"] ); |
  [ "/PackageInfo.g", "/init.g", "/makedocrel.g", "/read.g", "/start.g" ]
  !gapprompt@gap>| !gapinput@LogTo();|
  !gapprompt@gap>| !gapinput@FindMatchingFiles( "utils", [""], ["log"] );|
  [ "/feb1.log", "/test.log" ]
  !gapprompt@gap>| !gapinput@Log2HTML( "feb1.log" );|
  !gapprompt@gap>| !gapinput@FindMatchingFiles( "utils", [""], ["html"] );|
  [ "/feb1.html" ]
  
\end{Verbatim}
 }

 
\section{\textcolor{Chapter }{{\LaTeX} strings}}\label{sec:latex}
\logpage{[ 6, 5, 0 ]}
\hyperdef{L}{X84D2922D87EDE9E9}{}
{
  

\subsection{\textcolor{Chapter }{IntOrOnfinityToLaTeX}}
\logpage{[ 6, 5, 1 ]}\nobreak
\hyperdef{L}{X87DEB2B58266F858}{}
{\noindent\textcolor{FuncColor}{$\triangleright$\ \ \texttt{IntOrOnfinityToLaTeX({\mdseries\slshape n})\index{IntOrOnfinityToLaTeX@\texttt{IntOrOnfinityToLaTeX}}
\label{IntOrOnfinityToLaTeX}
}\hfill{\scriptsize (function)}}\\


 This function has transferred from package \textsf{ResClasses}. 

 \texttt{IntOrInfinityToLaTeX(n)} returns the {\LaTeX} string for \mbox{\texttt{\mdseries\slshape n}}. 

 }

 

 
\begin{Verbatim}[commandchars=!@|,fontsize=\small,frame=single,label=Example]
  
  !gapprompt@gap>| !gapinput@IntOrInfinityToLaTeX( 10^3 );|
  "1000"
  !gapprompt@gap>| !gapinput@IntOrInfinityToLaTeX( infinity );|
  "\\infty"
  
\end{Verbatim}
 

\subsection{\textcolor{Chapter }{LaTeXStringFactorsInt}}
\logpage{[ 6, 5, 2 ]}\nobreak
\hyperdef{L}{X7DC642B97CD02F4E}{}
{\noindent\textcolor{FuncColor}{$\triangleright$\ \ \texttt{LaTeXStringFactorsInt({\mdseries\slshape n})\index{LaTeXStringFactorsInt@\texttt{LaTeXStringFactorsInt}}
\label{LaTeXStringFactorsInt}
}\hfill{\scriptsize (function)}}\\


 This function has transferred from package \textsf{RCWA}. 

 It returns the prime factorization of the integer $n$ as a string in {\LaTeX} format. 

 }

 

 
\begin{Verbatim}[commandchars=!@|,fontsize=\small,frame=single,label=Example]
  
  !gapprompt@gap>| !gapinput@LaTeXStringFactorsInt( Factorial(12) );|
  "2^{10} \\cdot 3^5 \\cdot 5^2 \\cdot 7 \\cdot 11"
  
\end{Verbatim}
 }

 }

 \def\bibname{References\logpage{[ "Bib", 0, 0 ]}
\hyperdef{L}{X7A6F98FD85F02BFE}{}
}

\bibliographystyle{alpha}
\bibliography{manual}

\addcontentsline{toc}{chapter}{References}

\def\indexname{Index\logpage{[ "Ind", 0, 0 ]}
\hyperdef{L}{X83A0356F839C696F}{}
}

\cleardoublepage
\phantomsection
\addcontentsline{toc}{chapter}{Index}


\printindex

\newpage
\immediate\write\pagenrlog{["End"], \arabic{page}];}
\immediate\closeout\pagenrlog
\end{document}
